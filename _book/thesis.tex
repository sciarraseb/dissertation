% This is the Reed College LaTeX thesis template. Most of the work
% for the document class was done by Sam Noble (SN), as well as this
% template. Later comments etc. by Ben Salzberg (BTS). Additional
% restructuring and APA support by Jess Youngberg (JY).
% Your comments and suggestions are more than welcome; please email
% them to cus@reed.edu
%
% See https://www.reed.edu/cis/help/LaTeX/index.html for help. There are a
% great bunch of help pages there, with notes on
% getting started, bibtex, etc. Go there and read it if you're not
% already familiar with LaTeX.
%
% Any line that starts with a percent symbol is a comment.
% They won't show up in the document, and are useful for notes
% to yourself and explaining commands.
% Commenting also removes a line from the document;
% very handy for troubleshooting problems. -BTS

%%
%% Preamble
\documentclass[
12pt, % The default document font size, options: 10pt, 11pt, 12pt
twoside,
english]{guelphthesis}
%----------------------------------------------------------------------------------------
% PACKAGES
%----------------------------------------------------------------------------------------

\usepackage{tocloft} %needed for table of contents, list of figures, list of tables, list of appendices
\usepackage{graphicx,latexsym}
\usepackage{amsmath}
\usepackage{amssymb,amsthm}

\usepackage{longtable,booktabs,setspace}
\usepackage{lmodern}
\usepackage{float}
\usepackage{etoolbox}
\floatplacement{figure}{H}
% Thanks, @Xyv
\usepackage{calc}
% End of CII addition
\usepackage{rotating}
\usepackage{tocbibind} %includes list of figures, list of tables, and table of contents in table of contents
\usepackage{indentfirst} %needed so that first paragraph after each section titles has indent
\usepackage{lineno} %allows option for line numbering
\usepackage{draftwatermark} %for draft watermark
\SetWatermarkText{} %ensures draft is not printed when draft:false
%\usepackage[backend=biber]{biblatex}


% Syntax highlighting #22
  \usepackage{color}
  \usepackage{fancyvrb}
  \newcommand{\VerbBar}{|}
  \newcommand{\VERB}{\Verb[commandchars=\\\{\}]}
  \DefineVerbatimEnvironment{Highlighting}{Verbatim}{commandchars=\\\{\}}
  % Add ',fontsize=\small' for more characters per line
  \usepackage{framed}
  \definecolor{shadecolor}{RGB}{248,248,248}
  \newenvironment{Shaded}{\begin{snugshade}}{\end{snugshade}}
  \newcommand{\AlertTok}[1]{\textcolor[rgb]{0.94,0.16,0.16}{#1}}
  \newcommand{\AnnotationTok}[1]{\textcolor[rgb]{0.56,0.35,0.01}{\textbf{\textit{#1}}}}
  \newcommand{\AttributeTok}[1]{\textcolor[rgb]{0.77,0.63,0.00}{#1}}
  \newcommand{\BaseNTok}[1]{\textcolor[rgb]{0.00,0.00,0.81}{#1}}
  \newcommand{\BuiltInTok}[1]{#1}
  \newcommand{\CharTok}[1]{\textcolor[rgb]{0.31,0.60,0.02}{#1}}
  \newcommand{\CommentTok}[1]{\textcolor[rgb]{0.56,0.35,0.01}{\textit{#1}}}
  \newcommand{\CommentVarTok}[1]{\textcolor[rgb]{0.56,0.35,0.01}{\textbf{\textit{#1}}}}
  \newcommand{\ConstantTok}[1]{\textcolor[rgb]{0.00,0.00,0.00}{#1}}
  \newcommand{\ControlFlowTok}[1]{\textcolor[rgb]{0.13,0.29,0.53}{\textbf{#1}}}
  \newcommand{\DataTypeTok}[1]{\textcolor[rgb]{0.13,0.29,0.53}{#1}}
  \newcommand{\DecValTok}[1]{\textcolor[rgb]{0.00,0.00,0.81}{#1}}
  \newcommand{\DocumentationTok}[1]{\textcolor[rgb]{0.56,0.35,0.01}{\textbf{\textit{#1}}}}
  \newcommand{\ErrorTok}[1]{\textcolor[rgb]{0.64,0.00,0.00}{\textbf{#1}}}
  \newcommand{\ExtensionTok}[1]{#1}
  \newcommand{\FloatTok}[1]{\textcolor[rgb]{0.00,0.00,0.81}{#1}}
  \newcommand{\FunctionTok}[1]{\textcolor[rgb]{0.00,0.00,0.00}{#1}}
  \newcommand{\ImportTok}[1]{#1}
  \newcommand{\InformationTok}[1]{\textcolor[rgb]{0.56,0.35,0.01}{\textbf{\textit{#1}}}}
  \newcommand{\KeywordTok}[1]{\textcolor[rgb]{0.13,0.29,0.53}{\textbf{#1}}}
  \newcommand{\NormalTok}[1]{#1}
  \newcommand{\OperatorTok}[1]{\textcolor[rgb]{0.81,0.36,0.00}{\textbf{#1}}}
  \newcommand{\OtherTok}[1]{\textcolor[rgb]{0.56,0.35,0.01}{#1}}
  \newcommand{\PreprocessorTok}[1]{\textcolor[rgb]{0.56,0.35,0.01}{\textit{#1}}}
  \newcommand{\RegionMarkerTok}[1]{#1}
  \newcommand{\SpecialCharTok}[1]{\textcolor[rgb]{0.00,0.00,0.00}{#1}}
  \newcommand{\SpecialStringTok}[1]{\textcolor[rgb]{0.31,0.60,0.02}{#1}}
  \newcommand{\StringTok}[1]{\textcolor[rgb]{0.31,0.60,0.02}{#1}}
  \newcommand{\VariableTok}[1]{\textcolor[rgb]{0.00,0.00,0.00}{#1}}
  \newcommand{\VerbatimStringTok}[1]{\textcolor[rgb]{0.31,0.60,0.02}{#1}}
  \newcommand{\WarningTok}[1]{\textcolor[rgb]{0.56,0.35,0.01}{\textbf{\textit{#1}}}}



% To pass between YAML and LaTeX the dollar signs are added by CII
\title{Is Timing Everything? The Effects of Measurement Timing on the Performance of Nonlinear Longitudinal Models}
\author{Sebastian L.V. Sciarra}
\year{2023}
\date{February, 2023}
\advisor{David Stanley}
\institution{University of Guelph}
\degree{Doctorate of Philosophy}



\field{Industrial-Organizational Psychology}
\department{Psychology}



  \let\cleardoublepage\clearpage

% From {rticles}
%


\urlstyle{rm}

%----------------------------------------------------------------------------------------
% CUSTOM COMMANDS
%----------------------------------------------------------------------------------------
%numbers lines before equations
%taken from https://tex.stackexchange.com/questions/43648/why-doesnt-lineno-number-a-paragraph-when-it-is-followed-by-an-align-equation
\newcommand*\patchAmsMathEnvironmentForLineno[1]{%
  \expandafter\let\csname old#1\expandafter\endcsname\csname #1\endcsname
  \expandafter\let\csname oldend#1\expandafter\endcsname\csname end#1\endcsname
  \renewenvironment{#1}%
     {\linenomath\csname old#1\endcsname}%
     {\csname oldend#1\endcsname\endlinenomath}}%
\newcommand*\patchBothAmsMathEnvironmentsForLineno[1]{%
  \patchAmsMathEnvironmentForLineno{#1}%
  \patchAmsMathEnvironmentForLineno{#1*}}%
\AtBeginDocument{%
\patchBothAmsMathEnvironmentsForLineno{equation}%
\patchBothAmsMathEnvironmentsForLineno{align}%
\patchBothAmsMathEnvironmentsForLineno{flalign}%
\patchBothAmsMathEnvironmentsForLineno{alignat}%
\patchBothAmsMathEnvironmentsForLineno{gather}%
\patchBothAmsMathEnvironmentsForLineno{multline}%
}


%nest all the \frontmatter functions in \oldfrontmatter, which allows us to redefine \frontmatter as everything it was with one modification to the
%draft watermark
\let\oldfrontmatter\frontmatter
%set page numbering to bottom center for \frontmatter
\fancypagestyle{frontmatter}{%
 \fancyhf{}% clear all header and footer fields
  \renewcommand{\headrulewidth}{0pt}
  \fancyhead[R]{\roman{page}}% Roman page number in footer centre

  }

\renewcommand{\frontmatter}{
  \oldfrontmatter
  
   %set page number font to Arial if ArialFont: false in YAML header
  
   \pagestyle{frontmatter} % add this to center page numbers
}

%set page numbering to bottom center for \mainmatter
\fancypagestyle{mainmatter}{%
 \fancyhf{}% clear all header and footer fields
  \renewcommand{\headrulewidth}{0pt}
  \fancyfoot[C]{\arabic{page}}% Roman page number in footer centre

   \hypersetup{pdfpagemode={UseOutlines},
    bookmarksopen=true,
    hypertexnames=true,
    colorlinks = true,
    allcolors = blue,
    %linkcolor = blue,
    %urlcolor= blue,
    %anchorcolor = blue,
    pdfstartview={FitV},
    breaklinks=true,
    hyperindex = true,
    backref=page}

  \cleardoublepage

  


}

%nest all the \mainmatter functions in \oldmainmatter, which allows us to redefine \mainmatter as everything it was with one modification to the
%page numbering format
\newcommand{\setMainMatterLinespacing}{
 \setstretch{2} %default line spacing

  %change line spacing if specified in YAML header
        \setstretch{2}
  }

\let\oldmainmatter\mainmatter
\renewcommand{\mainmatter}{
  \oldmainmatter

  %change line spacing if specified in YAML header
  \setMainMatterLinespacing

      \linenumbers
  
  \pagestyle{mainmatter} % add this to center page numbers

}

%code below is important for linespacing to remain unaffected when kableExtra::landscape() is used andthe margin is specifically defined. Otherwise,
%linespacing for entire document goes to singlespacing for the text that follows the table.
\let\oldRestoreGeometry\restoregeometry
\renewcommand{\restoregeometry}{
  \oldRestoreGeometry

  %change line spacing if specified in YAML header
  \setMainMatterLinespacing
}

%change footnote and page number font to arial if desired

%----------------------------------------------------------------------------------------
%	TABLE OF CONTENTS, LIST OF FIGURES, & LIST OF TABLES
%----------------------------------------------------------------------------------------
%TABLE OF CONTENTS
\setlength{\cftbeforetoctitleskip}{0cm} %remove vertical space above table of contents

%two lines below ensure centered title for toc
%needed so that table of contents entry is not indented
\renewcommand{\contentsname}{Table of Contents} %change title for toc
\renewcommand{\cfttoctitlefont}{\hfill\fontsize{14}{14}\selectfont\bfseries\MakeUppercase}
\renewcommand{\cftaftertoctitle}{\hfill\hfill} %sometimes another \hfill is needed; depends on some setting in abovce code

%fonts for all entry level titles
\renewcommand\cftchapfont{\mdseries} %eliminate bolded chapter titles in toc
\renewcommand\cftsecfont{\mdseries} %eliminate bolded chapter titles in toc
\renewcommand\cftsubsecfont{\mdseries} %eliminate bolded chapter titles in toc
\renewcommand\cftsubsubsecfont{\mdseries} %eliminate bolded chapter titles in toc
\renewcommand\cftparafont{\mdseries} %eliminate bolded chapter titles in toc
\renewcommand\cftsubparafont{\mdseries} %eliminate bolded chapter titles in toc

%fonts for all entry level page numbers
\renewcommand{\cftchappagefont}{\mdseries} %remove bolding of page numbers for chapter headers in toc
\renewcommand\cftsecpagefont{\mdseries} %eliminate bolded chapter titles in toc
\renewcommand\cftsubsecpagefont{\mdseries} %eliminate bolded chapter titles in toc
\renewcommand\cftsubsubsecpagefont{\mdseries} %eliminate bolded chapter titles in toc
\renewcommand\cftparapagefont{\mdseries} %eliminate bolded chapter titles in toc
\renewcommand\cftsubparapagefont{\mdseries} %eliminate bolded chapter titles in toc

\renewcommand{\cftchapleader}{\cftdotfill{0.1}} %remove chapter bolding + modif dot spacing
\renewcommand{\cftdotsep}{0.1} %make dots in toc closer together

%spacing between toc items (should be all equal)
\setlength{\cftbeforechapskip}{0cm} %removes spacing before each chapter element
\renewcommand{\cftchapafterpnum}{\vskip6pt}
\renewcommand{\cftsecafterpnum}{\vskip6pt}
\renewcommand{\cftsubsecafterpnum}{\vskip6pt}
\renewcommand{\cftsubsubsecafterpnum}{\vskip6pt}
\renewcommand{\cftparaafterpnum}{\vskip6pt}
\renewcommand{\cftsubparaafterpnum}{\vskip6pt}

%remove header that appears in table of contents after first page
\renewcommand{\cftmarktoc}{}

%commands need to be redefined so that leading dots go all the way to the page numbers for all header levels (chap, sec, subsec, subsubsec, para, subpara
%%%general framework for commands below: cftXfillnum sets the format for the leading dots (\cftchapleader) and the page number (\cftchappagefont) such that leading dots proceed all the way to the page number with no spaces between dots and page number (\nobreak) at which wpoint paragraph mode ends (\par) and vertical spacing (defined  above) after item entry is inserted
%chapter (level 0)
\renewcommand{\cftchapfillnum}[1]{%
  {\cftchapleader}\nobreak
  {\cftchappagefont #1}\par\cftchapafterpnum
}

%sec (level 1)
\renewcommand{\cftsecfillnum}[1]{%
  {\cftsecleader}\nobreak
  {\cftsecpagefont #1}\par\cftsecafterpnum
}

%subsec (level 2)
\renewcommand{\cftsubsecfillnum}[1]{%
  {\cftsubsecleader}\nobreak
  {\cftsubsecpagefont #1}\par\cftsubsecafterpnum
}

%subsubsec (level 3)
\renewcommand{\cftsubsubsecfillnum}[1]{%
  {\cftsubsubsecleader}\nobreak
  {\cftsubsubsecpagefont #1}\par\cftsubsubsecafterpnum
}

%para (level 4)
\renewcommand{\cftparafillnum}[1]{%
  {\cftparaleader}\nobreak
  {\cftparapagefont #1}\par\cftparaafterpnum
}

%subpara (level 5)
\renewcommand{\cftsubparafillnum}[1]{%
  {\cftsubparaleader}\nobreak
  {\cftsubparapagefont #1}\par\cftsubparaafterpnum
}

%LIST OF TABLES
\renewcommand{\cfttabfont}{\mdseries} %set font for entries in lot
\renewcommand{\cfttabpagefont}{\mdseries} %set front for page numbers

\setlength{\cftbeforelottitleskip}{0cm} %remove vertical space above table of contents
\setlength{\cftafterlottitleskip}{0.5cm} %space between title for list of tables and list entries
%two lines below ensure centered title for toc
%needed so that table of contents entry is not indented
\renewcommand{\cftlottitlefont}{\hfill\fontsize{14}{14}\selectfont\bfseries\MakeUppercase}
\renewcommand{\cftafterlottitle}{\hfill} %sometimes another \hfill is needed; depends on some setting in abovce code

%commands need to be redefined so that leading dots go all the way to the page numbers for tables
%%%general framework for command below: cftfigfillnum sets the format for the leading dots (\cftfigleader) and the page number (\cftfigpagefont) such that leading dots proceed all the way to the page number with no spaces between dots and page number (\nobreak) at which point paragraph mode ends (\par) and vertical spacing (defined  below) after item entry is inserted
\setlength{\cftbeforetabskip}{0cm} %removes spacing before each chapter element
\renewcommand{\cfttabafterpnum}{\vskip6pt}

\renewcommand{\cfttabfillnum}[1]{%
  {\cfttableader}\nobreak
  {\cfttabpagefont #1}\par\cfttabafterpnum
}

%remove header that appears in list of tables after first page
\renewcommand{\cftmarklot}{}

%LIST OF FIGURES
\renewcommand{\cftfigfont}{\mdseries} %set font for entries in lof
\renewcommand{\cftfigpagefont}{\mdseries} %set front for page numbers

\setlength{\cftbeforeloftitleskip}{0cm} %remove vertical space above table of contents
\setlength{\cftafterloftitleskip}{0.5cm} %space between title for list of figures and list entries

%two lines below ensure centered title for toc
%needed so that table of contents entry is not indented
\renewcommand{\cftloftitlefont}{\hfill\fontsize{14}{14}\selectfont\bfseries\MakeUppercase}
\renewcommand{\cftafterloftitle}{\hfill} %sometimes another \hfill is needed; depends on some setting in abovce code

%commands need to be redefined so that leading dots go all the way to the page numbers for figures
%%%general framework for command below: cftfigfillnum sets the format for the leading dots (\cftfigleader) and the page number (\cftfigpagefont) such that leading dots proceed all the way to the page number with no spaces between dots and page number (\nobreak) at which wpoint paragraph mode ends (\par) and vertical spacing (defined  below) after item entry is inserted
\setlength{\cftbeforefigskip}{0cm} %removes spacing before each chapter element
\renewcommand{\cftfigafterpnum}{\vskip6pt}

\renewcommand{\cftfigfillnum}[1]{%
  {\cftfigleader}\nobreak
  {\cftfigpagefont #1}\par\cftfigafterpnum
}

%remove header that appears in list of figures after first page
\renewcommand{\cftmarklof}{}

%----------------------------------------------------------------------------------------
% LIST OF APPENDICES
%----------------------------------------------------------------------------------------
\newcommand{\listappname}{List of Appendices}
\newlistof[chapter]{app}{loa}{\listappname} %creates a new appendix counter that will be reset at the start of each \chapter

\setcounter{loadepth}{5} %loa will  go to depth of level 5
\setlength{\cftbeforeloatitleskip}{0cm} %remove vertical space above loa
\setlength{\cftafterloatitleskip}{0.5cm} %space between title for loa and list entries
\renewcommand{\cftmarkloa}{} %remove header titles

%two lines below ensure centered title for loa
%needed so that table of contents entry is not indented
\renewcommand{\cftloatitlefont}{\hfill\fontsize{14}{14}\selectfont\bfseries\MakeUppercase}
\renewcommand{\cftafterloatitle}{\hfill\hfill} %sometimes another \hfill is needed; depends on some setting in above code


%APPENDIX (level 0)
\renewcommand{\theapp}{\Alph{app}} %sets alphabetic counter for appendix
\renewcommand{\cftappfont}{\mdseries} %set font for level 0 entry in loa
\renewcommand{\cftapppagefont}{\mdseries} %set front for page numbers

\renewcommand{\cftapppresnum}{Appendix\space}
\renewcommand{\cftappaftersnum}{:\space}
\settowidth{\cftappnumwidth}{\cftapppresnum\theapp\cftappaftersnum\space}

\setlength{\cftbeforeappskip}{0cm} %removes vertical spacing before each chapter element
\renewcommand{\cftappafterpnum}{\vskip6pt}

%updates appendix counter, modifies chapter title such so that it is Appendix _letter_: #1
\newcommand{\app}[1]{%
  \refstepcounter{app}\pdfbookmark[-1]{\cftapppresnum\theapp\cftappaftersnum#1}{#1\theapp}%
  \chapter*{\fontsize{16}{16}\selectfont\bfseries\cftapppresnum\theapp\cftappaftersnum #1} %formats entry in document
  \addcontentsline{loa}{app}{{\cftapppresnum\theapp\cftappaftersnum}#1}%
  \par
}

% figure and table counting in appendix
\usepackage{chngcntr}


%leading dots for appendix (end immediately before page number)
\renewcommand{\cftappfillnum}[1]{%
 {\cftappleader}\nobreak{\cftapppagefont #1}\par\cftappafterpnum
}

%SECAPPENDIX (level 1; format A.1 : title)
\newlistentry[app]{secapp}{loa}{1}
\renewcommand{\thesecapp}{\theapp.\arabic{secapp}}
\renewcommand{\cftsecappfont}{\mdseries} %set font for level 1 entry in loa
\renewcommand{\cftsecapppagefont}{\mdseries} %set front for page numbers

\renewcommand{\cftsecapppresnum}{} %remove word 'Appendix'
\renewcommand{\cftsecappaftersnum}{\hspace{0.5cm}}  %replicate toc format for sub-level-0 headers \thesubappendix (i.e., A.1   title )

\setlength{\cftbeforesecappskip}{0cm} %removes vertical spacing before each chapter element
\renewcommand{\cftsecappafterpnum}{\vskip6pt}
\setlength{\cftsecappindent}{1.55em} %indentation in loa
\settowidth{\cftsecappnumwidth}{\cftsecapppresnum\thesecapp\cftsecappaftersnum\hspace{0.3cm}}

%updates appendix counter, modifies chapter title such so that it is Appendix _letter_: #1
\newcommand{\secapp}[1]{%
  \refstepcounter{secapp}\pdfbookmark[0]{#1}{#1\thesubapp}%
  \section*{\thesecapp\hspace{0.3cm} #1} %spacing between section number and title in text
  \addcontentsline{loa}{secapp}{{\thesecapp\cftsecappaftersnum}#1}%
  \par
}

%leading dots for appendix (end immediately before page number)
\renewcommand{\cftsecappfillnum}[1]{%
 {\cftsecappleader}\nobreak{\cftsecapppagefont #1}\par\cftsecappafterpnum
}


%SUBAPPENDIX (level 2; format A.1.1 : title)
\newlistentry[app]{subapp}{loa}{1}
\renewcommand{\thesubapp}{\thesecapp.\arabic{subapp}}
\renewcommand{\cftsubappfont}{\mdseries} %set font for level 2 entry in loa
\renewcommand{\cftsubapppagefont}{\mdseries} %set front for page numbers

\renewcommand{\cftsubapppresnum}{} %remove word 'Appendix'
\renewcommand{\cftsubappaftersnum}{\hspace{0.5cm}}  %replicate toc format for sub-level-0 headers \thesubappendix (i.e., A.1   title )

\setlength{\cftbeforesubappskip}{0cm} %removes vertical spacing before each chapter element
\renewcommand{\cftsubappafterpnum}{\vskip6pt}
\setlength{\cftsubappindent}{3.10em} %indentation in loa
%\renewcommand{\cftsubappnumwidth}{1.47cm}
\settowidth{\cftsubappnumwidth}{\thesubapp\cftsubappaftersnum\hspace{0.3cm}}

%updates appendix counter, modifies chapter title such so that it is Appendix _letter_: #1
\newcommand{\subapp}[1]{%
  \refstepcounter{subapp}\pdfbookmark[1]{#1}{#1\thesubapp}%
  \subsection*{\thesubapp\hspace{0.3cm} #1}%
  \addcontentsline{loa}{subapp}{{\thesubapp\cftsubappaftersnum}#1}%
  \par
}

%leading dots for appendix (end immediately before page number)
\renewcommand{\cftsubappfillnum}[1]{%
 {\cftsubappleader}\nobreak{\cftsubapppagefont #1}\par\cftsubappafterpnum
}


% SUBSUBAPPENDIX (level 3; format A.1.1.1  title)
\newlistentry[app]{subsubapp}{loa}{1}
\renewcommand{\thesubsubapp}{\thesubapp.\arabic{subsubapp}}
\renewcommand{\cftsubsubappfont}{\mdseries} %set font for level 3 entry in loa
\renewcommand{\cftsubsubapppagefont}{\mdseries} %set front for page numbers


\renewcommand{\cftsubsubapppresnum}{} %remove word 'Appendix'
\renewcommand{\cftsubsubappaftersnum}{\hspace{0.5cm}}  %space after subsubapp title

\setlength{\cftbeforesubsubappskip}{0cm} %removes vertical spacing before each chapter element
\renewcommand{\cftsubsubappafterpnum}{\vskip6pt}
\setlength{\cftsubsubappindent}{4.65em} %indentation in loa (1.55 *2)
\settowidth{\cftsubsubappnumwidth}{\thesubsubapp\cftsubsubappaftersnum\hspace{0.3cm}}

%updates appendix counter, modifies chapter title such so that it is Appendix _letter_: #1
\newcommand{\subsubapp}[1]{%
  \refstepcounter{subsubapp}\pdfbookmark[2]{#1}{#1\thesubsubapp}%
  \subsubsection*{\thesubsubapp\hspace{0.3cm} #1}%
  \addcontentsline{loa}{subsubapp}{{\thesubsubapp\cftsubsubappaftersnum}#1}%
  \par
}

%leading dots for appendix (end immediately before page number)
\renewcommand{\cftsubsubappfillnum}[1]{%
 {\cftsubsubappleader}\nobreak{\cftsubsubapppagefont #1}\par\cftsubsubappafterpnum
}

% PARA (level 4; format A.1.1.1.1  title)
\newlistentry[app]{paraapp}{loa}{1}
\renewcommand{\theparaapp}{\thesubsubapp.\arabic{paraapp}}
\renewcommand{\cftparaappfont}{\mdseries} %set font for level 4 entry in loa
\renewcommand{\cftparaapppagefont}{\mdseries} %set front for page numbers

\renewcommand{\cftparaapppresnum}{} %remove word 'Appendix'
\renewcommand{\cftparaappaftersnum}{\hspace{0.5cm}}  %space after paraapp title

\setlength{\cftbeforeparaappskip}{0cm} %removes vertical spacing before each chapter element
\renewcommand{\cftparaappafterpnum}{\vskip6pt}
\setlength{\cftparaappindent}{6.2em} %indentation in loa (1.55 *2)
\settowidth{\cftparaappnumwidth}{\theparaapp\cftparaappaftersnum\hspace{0.3cm}}

%updates appendix counter, modifies chapter title such so that it is Appendix _letter_: #1
\newcommand{\paraapp}[1]{%
  \refstepcounter{paraapp}\pdfbookmark[3]{#1}{#1\theparaapp}%
  \paragraph*{\theparaapp\hspace{0.3cm} #1}%
  \addcontentsline{loa}{paraapp}{{\theparaapp\cftparaappaftersnum}#1}%
  \par
}

%leading dots for appendix (end immediately before page number)
\renewcommand{\cftparaappfillnum}[1]{%
 {\cftparaappleader}\nobreak{\cftparaapppagefont #1}\par\cftparaappafterpnum
}

% SUBPARA (level 5; format A.1.1.1.1  title)
\newlistentry[app]{subparaapp}{loa}{1}
\renewcommand{\thesubparaapp}{\theparaapp.\arabic{subparaapp}}
\renewcommand{\cftsubparaappfont}{\mdseries} %set font for level 5 entry in loa
\renewcommand{\cftsubparaapppagefont}{\mdseries} %set front for page numbers

\renewcommand{\cftsubparaapppresnum}{} %remove word 'Appendix'
\renewcommand{\cftsubparaappaftersnum}{\hspace{0.5cm}}  %space after subparaapp title

\setlength{\cftbeforesubparaappskip}{0cm} %removes vertical spacing before each chapter element
\renewcommand{\cftsubparaappafterpnum}{\vskip6pt}
\setlength{\cftsubparaappindent}{7.75em} %indentation in loa (1.55 *2)
\settowidth{\cftsubparaappnumwidth}{\thesubparaapp\cftsubparaappaftersnum\hspace{0.3cm}}

%updates appendix counter, modifies chapter title such so that it is Appendix _letter_: #1
\newcommand{\subparaapp}[1]{%
  \refstepcounter{subparaapp}\pdfbookmark[4]{#1}{#1\thesubparaapp}%
  \paragraph*{\thesubparaapp\hspace{0.3cm} #1} %paragraph is used because subparagraph has weird numbering problem
  \addcontentsline{loa}{subparaapp}{{\thesubparaapp\cftsubparaappaftersnum}#1}%
  \par
}

%SUBSUBPARA (level 6; format A.1.1.1.1.1  title)
\newlistentry[app]{subsubparaapp}{loa}{1}
\renewcommand{\thesubsubparaapp}{\thesubparaapp.\arabic{subsubparaapp}}

\renewcommand{\cftsubsubparaapppresnum}{} %remove word 'Appendix'
\renewcommand{\cftsubsubparaappaftersnum}{\hspace{0.5cm}}  %space after subparaapp title

\setlength{\cftbeforesubsubparaappskip}{0cm} %removes vertical spacing before each chapter element
\renewcommand{\cftsubsubparaappafterpnum}{\vskip6pt}
\setlength{\cftsubsubparaappindent}{9.3em} %indentation in loa (1.55 *2)
\settowidth{\cftsubsubparaappnumwidth}{\thesubsubparaapp\cftsubsubparaappaftersnum\hspace{0.3cm}}

%updates appendix counter, modifies chapter title such so that it is Appendix _letter_: #1
\newcommand{\subsubparaapp}[1]{%
  \refstepcounter{subsubparaapp}\pdfbookmark[5]{#1}{#1\thesubsubparaapp}%
  \subparagraph*{\thesubsubparaapp\hspace{0.3cm} #1} %paragraph is used because subparagraph has weird numbering problem
  \addcontentsline{loa}{subsubparaapp}{{\thesubsubparaapp\cftsubsubparaappaftersnum}#1}%
  \par
}

%leading dots for appendix (end immediately before page number)
\renewcommand{\cftsubsubparaappfillnum}[1]{%
 {\cftsubsubparaappleader}\nobreak{\cftsubsubparaapppagefont #1}\par\cftsubsubparaappafterpnum
}

\newcommand{\listabbname}{List of Abbreviations}
\newlistof[chapter]{abb}{loab}{\listabbname} %creates a new appendix counter that will be reset at the start of each \chapter

\setlength{\cftbeforeloabtitleskip}{0cm} %remove vertical space above loab
\setlength{\cftafterloabtitleskip}{0.2cm} %space between title for loab and list entries

\renewcommand{\cftmarkloab}{} %remove header titles

%two lines below ensure centered title for loa
%needed so that table of contents entry is not indented
\renewcommand{\cftloabtitlefont}{\hfill\fontsize{14}{14}\selectfont\bfseries\MakeUppercase}
\renewcommand{\cftafterloabtitle}{\hfill\hfill} %sometimes another \hfill is needed; depends on some setting in above code



%----------------------------------------------------------------------------------------
% REFERENCES & HYPERLINKING
%----------------------------------------------------------------------------------------

\usepackage{hyperref}

\PassOptionsToPackage{backref=true}{biblatex}

\RequirePackage[autocite=inline, style = apa]{biblatex}
\addbibresource{bib/references.bib}


\DeclareSourcemap{\maps[datatype = bibtex]{\map{\step[fieldsource = journal, match = \regexp{\x{26}}, replace = \regexp{\{\\\x{26}\}}] }}}
\DeclareSourcemap{\maps[datatype = bibtex]{\map{\step[fieldsource = title, match = \regexp{\x{26}}, replace = \regexp{\{\\\x{26}\}}] }}}

\hypersetup{pdfpagemode={UseOutlines},
    bookmarksopen=true,
    backref=page}
\usepackage{hypernat}
%%adds escape character to ampersand characters in journal fields of .bib file
\DefineBibliographyStrings{english}{backrefpage={cited on p.},backrefpages={cited on pp.}}




\hypersetup{pdfpagemode={UseOutlines},
bookmarksopen=true, %allows bookmarks in pdf
hypertexnames=true, %enables counting when referencing to sections
colorlinks = true, % Set to true to enable coloring links, a nice option, false to turn them off
%citecolor = blue, % The color of citations
%linkcolor = blue, % The color of references to document elements (sections, figures, etc)
%urlcolor= blue,
%anchorcolor = blue, % The color of hyperlinks (URLs)
allcolors = blue,
pdfstartview={FitV},
breaklinks=true, backref=page
}


%example numbering
\newtheorem{theorem}{Theorem}[section]
\renewcommand{\thetheorem}{\theapp.\arabic{theorem}}
\newtheorem{example}{Example}
\renewcommand{\theexample}{\theapp.\arabic{example}}


%load additional latex packages needed within document
	\usepackage{booktabs}
\usepackage{longtable}
\usepackage{array}
\usepackage{multirow}
\usepackage{wrapfig}
\usepackage{float}
\usepackage{colortbl}
\usepackage{pdflscape}
\usepackage{tabu}
\usepackage{threeparttable}
\usepackage{threeparttablex}
\usepackage[normalem]{ulem}
\usepackage{makecell}
\usepackage{xcolor}

%----------------------------------------------------------------------------------------
% DOCUMENT OUTLINE
%----------------------------------------------------------------------------------------

% BEGIN DOCUMENT
\begin{document}
\frontmatter %pages will be numbered with roman numerals

  \maketitle

\setcounter{page}{2} %ensures abstract page number starts at roman numberal ii

\cleardoublepage
\thispagestyle{empty} %removes page number only for abstract page
  \begin{abstract}{2}{Despite the value that longitudinal research offers for understanding psychological processes, studies in organizational research rarely use longitudinal designs. One reason for the paucity of longitudinal designs may be the challenges they present for researchers. Three challenges of particular importance are that researchers have to determine 1) how many measurements to take, 2) how to space measurements, and 3) how to design studies when participants provide data with different response schedules (time unstructuredness). In systematically reviewing the simulation literature, I found that few studies comprehensively investigated the effects of measurement number, measurement spacing, and time structuredness (in addition to sample size) on model performance. As a consequence, researchers have little guidance when trying to conduct longitudinal research. To address these gaps in the literature, I conducted a series of simulation experiments. I found poor model performance across all measurement number/sample size pairings. That is, bias and precision were never concurrently optimized under any combination of manipulated variables. Bias was often low, however, with moderate measurement numbers and sample sizes. Although precision was frequently low, the greatest improvements in precision resulted from using either seven measurements with \(N \ge 200\) or nine measurements with \(N \le 100\). With time-unstructured data, model performance systematically decreased across all measurement number/sample size pairings when the model incorrectly assumed an identical response pattern across all participants (i.e., time-structured data). Fortunately, when models were equipped to handle heterogeneous response patterns using definition variables, the poor model performance observed across all measurement number/sample size pairings no longer appeared. Altogether, the results of the current simulation experiments provide guidelines for researchers interested in modelling nonlinear change.}  %[linespacing][abstract][

  \end{abstract}

% notice how yaml variables are indexed with dollar signs and then passed into second argument of preambleItem environments
  \cleardoublepage
  \begin{preambleItem}{2}{Dedication}{{[}To be completed after defence{]}}
  \end{preambleItem}
  \cleardoublepage
   \begin{preambleItem}{2}{Acknowledgements}{{[}To be completed after defence{]}}
  \end{preambleItem}


%move page numbers to top right for list of tables, figures, and tables
\fancypagestyle{plain}{%
  \fancyhf{}% clear all header and footer fields
  \renewcommand{\headrulewidth}{0pt}
  \fancyhead[R]{\thepage}

   }

%table of contents
  \cleardoublepage
  \hypersetup{linkcolor = black, pdfborder= 0 0 0} %remove red borders around toc items
  \setcounter{secnumdepth}{5}
  \setcounter{tocdepth}{5}
  \tableofcontents
  \newpage

%list of tables
  \cleardoublepage
  \listoftables
  \newpage

%list of figures
  \cleardoublepage
  \listoffigures
  \newpage


%list of appendices
  \cleardoublepage
  \phantomsection
  \addcontentsline{toc}{chapter}{\listappname}
  \listofapp

  \newpage

\mainmatter % here the regular arabic numbering starts

\hypertarget{if-you-are-creating-a-pdf-youll-need-to-write-your-preliminary-content}{%
\chapter{If you are creating a PDF you'll need to write your preliminary content}\label{if-you-are-creating-a-pdf-youll-need-to-write-your-preliminary-content}}

Placeholder

\hypertarget{the-need-to-conduct-longitudinal-research}{%
\section{The Need to Conduct Longitudinal Research}\label{the-need-to-conduct-longitudinal-research}}

\hypertarget{understanding-patterns-of-change-that-emerge-over-time}{%
\section{Understanding Patterns of Change That Emerge Over Time}\label{understanding-patterns-of-change-that-emerge-over-time}}

\hypertarget{challenges-involved-in-conducting-longitudinal-research}{%
\section{Challenges Involved in Conducting Longitudinal Research}\label{challenges-involved-in-conducting-longitudinal-research}}

\hypertarget{number-of-measurements}{%
\subsection{Number of Measurements}\label{number-of-measurements}}

\hypertarget{spacing-of-measurements}{%
\subsection{Spacing of Measurements}\label{spacing-of-measurements}}

\hypertarget{time-structuredness}{%
\subsection{Time Structuredness}\label{time-structuredness}}

\hypertarget{time-structured-data}{%
\subsubsection{Time-Structured Data}\label{time-structured-data}}

\hypertarget{time-unstructured-data}{%
\subsubsection{Time-Unstructured Data}\label{time-unstructured-data}}

\hypertarget{summary}{%
\subsection{Summary}\label{summary}}

\hypertarget{using-simulations-to-assess-modelling-accuracy}{%
\section{Using Simulations To Assess Modelling Accuracy}\label{using-simulations-to-assess-modelling-accuracy}}

\hypertarget{systematic-review-of-simulation-literature}{%
\section{Systematic Review of Simulation Literature}\label{systematic-review-of-simulation-literature}}

\hypertarget{systematic-review-methodology}{%
\subsection{Systematic Review Methodology}\label{systematic-review-methodology}}

\hypertarget{systematic-review-results}{%
\subsection{Systematic Review Results}\label{systematic-review-results}}

\hypertarget{modelling-change}{%
\section{Methods of Modelling Nonlinear Patterns of Change Over Time}\label{modelling-change}}

\hypertarget{multilevel-and-latent-variable-approach}{%
\section{Multilevel and Latent Variable Approach}\label{multilevel-and-latent-variable-approach}}

\hypertarget{next-steps}{%
\subsection{Next Steps}\label{next-steps}}

\hypertarget{overview-of-simulation-experiments}{%
\section{Overview of Simulation Experiments}\label{overview-of-simulation-experiments}}

\hypertarget{exp-1}{%
\chapter{Experiment 1}\label{exp-1}}

Placeholder

\hypertarget{methods}{%
\section{Methods}\label{methods}}

\hypertarget{data-generation}{%
\subsection{Overview of Data Generation}\label{data-generation}}

\hypertarget{function-used-to-generate-each-data-set}{%
\subsubsection{Function Used to Generate Each Data Set}\label{function-used-to-generate-each-data-set}}

\hypertarget{population-values-used-for-function-parameters}{%
\subsubsection{Population Values Used for Function Parameters}\label{population-values-used-for-function-parameters}}

\hypertarget{data-modelling}{%
\subsection{Modelling of Each Generated Data Set}\label{data-modelling}}

\hypertarget{variables-used-in-simulation-experiment}{%
\subsection{Variables Used in Simulation Experiment}\label{variables-used-in-simulation-experiment}}

\hypertarget{independent-variables}{%
\subsubsection{Independent Variables}\label{independent-variables}}

\hypertarget{spacing-measurements}{%
\paragraph{Spacing of Measurements}\label{spacing-measurements}}

\hypertarget{number-measurements}{%
\paragraph{Number of Measurements}\label{number-measurements}}

\hypertarget{population-values-set-for-the-fixed-effect-days-to-halfway-elevation-parameter-upbeta_fixed-nature-of-change}{%
\paragraph{\texorpdfstring{Population Values Set for The Fixed-Effect Days-to-Halfway Elevation Parameter \(\upbeta_{fixed}\) (Nature of Change)}{Population Values Set for The Fixed-Effect Days-to-Halfway Elevation Parameter \textbackslash upbeta\_\{fixed\} (Nature of Change)}}\label{population-values-set-for-the-fixed-effect-days-to-halfway-elevation-parameter-upbeta_fixed-nature-of-change}}

\hypertarget{constants}{%
\subsubsection{Constants}\label{constants}}

\hypertarget{dependent-variables}{%
\subsubsection{Dependent Variables}\label{dependent-variables}}

\hypertarget{convergence}{%
\paragraph{Convergence Success Rate}\label{convergence}}

\hypertarget{model-performance}{%
\paragraph{Model Performance}\label{model-performance}}

\hypertarget{bias-comp}{%
\subparagraph{Bias}\label{bias-comp}}

\hypertarget{pres-precision}{%
\subparagraph{Precision}\label{pres-precision}}

\hypertarget{analysis-visualization}{%
\subsection{Analysis of Data Modelling Output and Accompanying Visualizations}\label{analysis-visualization}}

\hypertarget{convergence-analysis}{%
\subsubsection{Analysis of Convergence Success Rate}\label{convergence-analysis}}

\hypertarget{bias-analysis}{%
\subsubsection{Analysis and Visualization of Bias}\label{bias-analysis}}

\hypertarget{precision-analysis}{%
\subsubsection{Analysis and Visualization of Precision}\label{precision-analysis}}

\hypertarget{effect-size-computation-for-precision}{%
\paragraph{Effect Size Computation for Precision}\label{effect-size-computation-for-precision}}

\hypertarget{results-and-discussion}{%
\section{Results and Discussion}\label{results-and-discussion}}

\hypertarget{framework-for-interpreting-results}{%
\subsection{Framework for Interpreting Results}\label{framework-for-interpreting-results}}

\hypertarget{pre-processing-of-data-and-model-convergence}{%
\subsection{Pre-Processing of Data and Model Convergence}\label{pre-processing-of-data-and-model-convergence}}

\hypertarget{concise-tab}{%
\subsection{Equal Spacing}\label{concise-tab}}

\hypertarget{nature-change-equal-exp1}{%
\subsubsection{Nature of Change That Leads to Highest Model Performance}\label{nature-change-equal-exp1}}

\hypertarget{bias-equal-exp1}{%
\subsubsection{Bias}\label{bias-equal-exp1}}

\hypertarget{precision-equal-exp1}{%
\subsubsection{Precision}\label{precision-equal-exp1}}

\hypertarget{qualitative-equal-exp1}{%
\subsubsection{Qualitative Description}\label{qualitative-equal-exp1}}

\hypertarget{summary-of-results-with-equal-spacing}{%
\subsubsection{Summary of Results With Equal Spacing}\label{summary-of-results-with-equal-spacing}}

\hypertarget{time-interval-increasing-spacing}{%
\subsection{Time-Interval Increasing Spacing}\label{time-interval-increasing-spacing}}

\hypertarget{nature-change-time-inc-exp1}{%
\subsubsection{Nature of Change That Leads to Highest Model Performance}\label{nature-change-time-inc-exp1}}

\hypertarget{bias-time-inc-exp1}{%
\subsubsection{Bias}\label{bias-time-inc-exp1}}

\hypertarget{precision-time-inc-exp1}{%
\subsubsection{Precision}\label{precision-time-inc-exp1}}

\hypertarget{qualitative-time-inc-exp1}{%
\subsubsection{Qualitative Description}\label{qualitative-time-inc-exp1}}

\hypertarget{summary-of-results-with-time-interval-increasing-spacing}{%
\subsubsection{Summary of Results With Time-Interval Increasing Spacing}\label{summary-of-results-with-time-interval-increasing-spacing}}

\hypertarget{time-interval-decreasing-spacing}{%
\subsection{Time-Interval Decreasing Spacing}\label{time-interval-decreasing-spacing}}

\hypertarget{nature-change-time-dec-exp1}{%
\subsubsection{Nature of Change That Leads to Highest Model Performance}\label{nature-change-time-dec-exp1}}

\hypertarget{bias-time-dec-exp1}{%
\subsubsection{Bias}\label{bias-time-dec-exp1}}

\hypertarget{precision-time-dec-exp1}{%
\subsubsection{Precision}\label{precision-time-dec-exp1}}

\hypertarget{qualitative-time-dec-exp1}{%
\subsubsection{Qualitative Description}\label{qualitative-time-dec-exp1}}

\hypertarget{summary-of-results-time-interval-decreasing-spacing}{%
\subsubsection{Summary of Results Time-Interval Decreasing Spacing}\label{summary-of-results-time-interval-decreasing-spacing}}

\hypertarget{middle-and-extreme-spacing}{%
\subsection{Middle-and-Extreme Spacing}\label{middle-and-extreme-spacing}}

\hypertarget{nature-change-mid-ext-exp1}{%
\subsubsection{Nature of Change That Leads to Highest Model Performance}\label{nature-change-mid-ext-exp1}}

\hypertarget{bias-mid-ext-exp1}{%
\subsubsection{Bias}\label{bias-mid-ext-exp1}}

\hypertarget{precision-mid-ext-exp1}{%
\subsubsection{Precision}\label{precision-mid-ext-exp1}}

\hypertarget{qualitative-mid-ext-exp1}{%
\subsubsection{Qualitative Description}\label{qualitative-mid-ext-exp1}}

\hypertarget{summary-of-results-with-middle-and-extreme-spacing}{%
\subsubsection{Summary of Results With Middle-and-Extreme Spacing}\label{summary-of-results-with-middle-and-extreme-spacing}}

\hypertarget{addressing-my-research-questions}{%
\subsection{Addressing My Research Questions}\label{addressing-my-research-questions}}

\hypertarget{meas-placing}{%
\subsubsection{Does Placing Measurements Near Periods of Change Increase Model Performance?}\label{meas-placing}}

\hypertarget{unknown}{%
\subsubsection{When the Nature of Change is Unknown, How Should Measurements be Spaced?}\label{unknown}}

\hypertarget{summary-of-experiment-1}{%
\section{Summary of Experiment 1}\label{summary-of-experiment-1}}

\hypertarget{Exp2}{%
\chapter{Experiment 2}\label{Exp2}}

Placeholder

\hypertarget{methods-1}{%
\section{Methods}\label{methods-1}}

\hypertarget{overview-of-data-generation}{%
\subsection{Overview of Data Generation}\label{overview-of-data-generation}}

\hypertarget{data-modelling-exp2}{%
\subsection{Modelling of Each Generated Data Set}\label{data-modelling-exp2}}

\hypertarget{variables-used-in-simulation-experiment-1}{%
\subsection{Variables Used in Simulation Experiment}\label{variables-used-in-simulation-experiment-1}}

\hypertarget{independent-variables-1}{%
\subsubsection{Independent Variables}\label{independent-variables-1}}

\hypertarget{spacing-of-measurements-1}{%
\paragraph{Spacing of Measurements}\label{spacing-of-measurements-1}}

\hypertarget{number-of-measurements-1}{%
\paragraph{Number of Measurements}\label{number-of-measurements-1}}

\hypertarget{sample-size}{%
\paragraph{Sample Size}\label{sample-size}}

\hypertarget{constants-exp2}{%
\subsubsection{Constants}\label{constants-exp2}}

\hypertarget{dependent-variables-1}{%
\subsubsection{Dependent Variables}\label{dependent-variables-1}}

\hypertarget{convergence-success-rate}{%
\paragraph{Convergence Success Rate}\label{convergence-success-rate}}

\hypertarget{model-performance-1}{%
\paragraph{Model Performance}\label{model-performance-1}}

\hypertarget{bias}{%
\subparagraph{Bias}\label{bias}}

\hypertarget{precision}{%
\subparagraph{Precision}\label{precision}}

\hypertarget{analysis-of-data-modelling-output-and-accompanying-visualizations}{%
\subsection{Analysis of Data Modelling Output and Accompanying Visualizations}\label{analysis-of-data-modelling-output-and-accompanying-visualizations}}

\hypertarget{results-and-discussion-1}{%
\section{Results and Discussion}\label{results-and-discussion-1}}

\hypertarget{framework-for-interpreting-results-1}{%
\subsection{Framework for Interpreting Results}\label{framework-for-interpreting-results-1}}

\hypertarget{pre-processing-of-data-and-model-convergence-1}{%
\subsection{Pre-Processing of Data and Model Convergence}\label{pre-processing-of-data-and-model-convergence-1}}

\hypertarget{concise-example}{%
\subsection{Equal Spacing}\label{concise-example}}

\hypertarget{bias-equal-exp2}{%
\subsubsection{Bias}\label{bias-equal-exp2}}

\hypertarget{precision-equal-exp2}{%
\subsubsection{Precision}\label{precision-equal-exp2}}

\hypertarget{qualitative-equal-exp2}{%
\subsubsection{Qualitative Description}\label{qualitative-equal-exp2}}

\hypertarget{summary-of-results-with-equal-spacing-1}{%
\subsubsection{Summary of Results With Equal Spacing}\label{summary-of-results-with-equal-spacing-1}}

\hypertarget{time-interval-increasing-spacing-1}{%
\subsection{Time-Interval Increasing Spacing}\label{time-interval-increasing-spacing-1}}

\hypertarget{bias-time-inc-exp2}{%
\subsubsection{Bias}\label{bias-time-inc-exp2}}

\hypertarget{precision-time-inc-exp2}{%
\subsubsection{Precision}\label{precision-time-inc-exp2}}

\hypertarget{qualitative-time-inc-exp2}{%
\subsubsection{Qualitative Description}\label{qualitative-time-inc-exp2}}

\hypertarget{summary-of-results-with-time-interval-increasing-spacing-1}{%
\subsubsection{Summary of Results With Time-Interval Increasing Spacing}\label{summary-of-results-with-time-interval-increasing-spacing-1}}

\hypertarget{time-interval-decreasing-spacing-1}{%
\subsection{Time-Interval Decreasing Spacing}\label{time-interval-decreasing-spacing-1}}

\hypertarget{bias-time-dec-exp2}{%
\subsubsection{Bias}\label{bias-time-dec-exp2}}

\hypertarget{precision-time-dec-exp2}{%
\subsubsection{Precision}\label{precision-time-dec-exp2}}

\hypertarget{qualitative-time-dec-exp2}{%
\subsubsection{Qualitative Description}\label{qualitative-time-dec-exp2}}

\hypertarget{summary-of-results-time-interval-decreasing-spacing-1}{%
\subsubsection{Summary of Results Time-Interval Decreasing Spacing}\label{summary-of-results-time-interval-decreasing-spacing-1}}

\hypertarget{middle-and-extreme-spacing-1}{%
\subsection{Middle-and-Extreme Spacing}\label{middle-and-extreme-spacing-1}}

\hypertarget{bias-mid-ext-exp2}{%
\subsubsection{Bias}\label{bias-mid-ext-exp2}}

\hypertarget{precision-mid-ext-exp2}{%
\subsubsection{Precision}\label{precision-mid-ext-exp2}}

\hypertarget{qualitative-mid-ext-exp2}{%
\subsubsection{Qualitative Description}\label{qualitative-mid-ext-exp2}}

\hypertarget{summary-of-results-with-middle-and-extreme-spacing-1}{%
\subsubsection{Summary of Results with Middle-and-Extreme Spacing}\label{summary-of-results-with-middle-and-extreme-spacing-1}}

\hypertarget{what-measurement-numbersample-size-pairings-should-be-used-with-each-spacing-schedule}{%
\section{What Measurement Number/Sample Size Pairings Should be Used With Each Spacing Schedule?}\label{what-measurement-numbersample-size-pairings-should-be-used-with-each-spacing-schedule}}

\hypertarget{summary-of-experiment-2}{%
\section{Summary of Experiment 2}\label{summary-of-experiment-2}}

\hypertarget{Exp3}{%
\chapter{Experiment 3}\label{Exp3}}

Placeholder

\hypertarget{methods-2}{%
\section{Methods}\label{methods-2}}

\hypertarget{variables-used-in-simulation-experiment-2}{%
\subsection{Variables Used in Simulation Experiment}\label{variables-used-in-simulation-experiment-2}}

\hypertarget{independent-variables-2}{%
\subsubsection{Independent Variables}\label{independent-variables-2}}

\hypertarget{number-of-measurements-2}{%
\paragraph{Number of Measurements}\label{number-of-measurements-2}}

\hypertarget{sample-size-1}{%
\paragraph{Sample Size}\label{sample-size-1}}

\hypertarget{time-structuredness-1}{%
\paragraph{Time Structuredness}\label{time-structuredness-1}}

\hypertarget{constants-1}{%
\subsubsection{Constants}\label{constants-1}}

\hypertarget{dependent-variables-2}{%
\subsubsection{Dependent Variables}\label{dependent-variables-2}}

\hypertarget{convergence-success-rate-1}{%
\paragraph{Convergence Success Rate}\label{convergence-success-rate-1}}

\hypertarget{model-performance-2}{%
\paragraph{Model Performance}\label{model-performance-2}}

\hypertarget{bias-1}{%
\subparagraph{Bias}\label{bias-1}}

\hypertarget{precision-1}{%
\subparagraph{Precision}\label{precision-1}}

\hypertarget{overview-of-data-generation-1}{%
\subsection{Overview of Data Generation}\label{overview-of-data-generation-1}}

\hypertarget{simulating-time-struc}{%
\paragraph{Simulation Procedure for Time Structuredness}\label{simulating-time-struc}}

\hypertarget{data-modelling-exp3}{%
\subsection{Modelling of Each Generated Data Set}\label{data-modelling-exp3}}

\hypertarget{analysis-of-data-modelling-output-and-accompanying-visualizations-1}{%
\subsection{Analysis of Data Modelling Output and Accompanying Visualizations}\label{analysis-of-data-modelling-output-and-accompanying-visualizations-1}}

\hypertarget{results-and-discussion-2}{%
\section{Results and Discussion}\label{results-and-discussion-2}}

\hypertarget{framework-for-interpreting-results-2}{%
\subsection{Framework for Interpreting Results}\label{framework-for-interpreting-results-2}}

\hypertarget{pre-processing-of-data-and-model-convergence-2}{%
\subsection{Pre-Processing of Data and Model Convergence}\label{pre-processing-of-data-and-model-convergence-2}}

\hypertarget{concise-example-exp3}{%
\subsection{Time-Structured Data}\label{concise-example-exp3}}

\hypertarget{bias-time-struc-exp3}{%
\paragraph{Bias}\label{bias-time-struc-exp3}}

\hypertarget{precision-time-struc-exp3}{%
\paragraph{Precision}\label{precision-time-struc-exp3}}

\hypertarget{qualitative-time-struc-exp3}{%
\paragraph{Qualitative Description}\label{qualitative-time-struc-exp3}}

\hypertarget{summary-of-results-for-time-structured-data}{%
\subsubsection{Summary of Results for Time-Structured Data}\label{summary-of-results-for-time-structured-data}}

\hypertarget{time-unstructured-data-characterized-by-a-fast-response-rate}{%
\subsection{Time-Unstructured Data Characterized by a Fast Response Rate}\label{time-unstructured-data-characterized-by-a-fast-response-rate}}

\hypertarget{bias-fast-exp3}{%
\subsubsection{Bias}\label{bias-fast-exp3}}

\hypertarget{precision-fast-exp3}{%
\subsubsection{Precision}\label{precision-fast-exp3}}

\hypertarget{qualitative-fast-exp3}{%
\subsubsection{Qualitative Description}\label{qualitative-fast-exp3}}

\hypertarget{summary-of-results-for-time-unstructured-characterized-by-a-fast-response-rate}{%
\subsubsection{Summary of Results for Time-Unstructured Characterized by a Fast Response Rate}\label{summary-of-results-for-time-unstructured-characterized-by-a-fast-response-rate}}

\hypertarget{time-unstructured-data-characterized-by-a-slow-response-rate}{%
\subsection{Time-Unstructured Data Characterized by a Slow Response Rate}\label{time-unstructured-data-characterized-by-a-slow-response-rate}}

\hypertarget{bias-slow-exp3}{%
\subsubsection{Bias}\label{bias-slow-exp3}}

\hypertarget{precision-slow-exp3}{%
\subsubsection{Precision}\label{precision-slow-exp3}}

\hypertarget{qualitative-slow-exp3}{%
\subsubsection{Qualitative Description}\label{qualitative-slow-exp3}}

\hypertarget{summary-of-results-time-unstructured-characterized-by-a-slow-response-rate}{%
\subsubsection{Summary of Results Time-Unstructured Characterized by a Slow Response Rate}\label{summary-of-results-time-unstructured-characterized-by-a-slow-response-rate}}

\hypertarget{how-does-time-structuredness-affect-model-performance}{%
\subsection{How Does Time Structuredness Affect Model Performance?}\label{how-does-time-structuredness-affect-model-performance}}

\hypertarget{def-variables}{%
\subsection{Eliminating the Bias Caused by Time Unstructuredness: Using Definition Variables}\label{def-variables}}

\hypertarget{summary-of-experiment-3}{%
\section{Summary of Experiment 3}\label{summary-of-experiment-3}}

\hypertarget{general-discussion}{%
\chapter{General Discussion}\label{general-discussion}}

Placeholder

\hypertarget{limitations-and-future-directions}{%
\section{Limitations and Future Directions}\label{limitations-and-future-directions}}

\hypertarget{cutoff-values-for-bias-and-precision}{%
\subsection{Cutoff Values for Bias and Precision}\label{cutoff-values-for-bias-and-precision}}

\hypertarget{external-validity-of-simulation-experiments}{%
\subsection{External Validity of Simulation Experiments}\label{external-validity-of-simulation-experiments}}

\hypertarget{simulations-with-other-longitudinal-analyses}{%
\subsection{Simulations With Other Longitudinal Analyses}\label{simulations-with-other-longitudinal-analyses}}

\hypertarget{nonlinear-patterns-and-longitudinal-research}{%
\section{Nonlinear Patterns and Longitudinal Research}\label{nonlinear-patterns-and-longitudinal-research}}

\hypertarget{a-new-perpective-on-longitudinal-designs-for-modelling-change}{%
\subsection{A New Perpective on Longitudinal Designs for Modelling Change}\label{a-new-perpective-on-longitudinal-designs-for-modelling-change}}

\hypertarget{why-is-it-important-to-model-nonlinear-patterns-of-change}{%
\subsection{Why is it Important to Model Nonlinear Patterns of Change?}\label{why-is-it-important-to-model-nonlinear-patterns-of-change}}

\hypertarget{suggestions-for-modelling-nonlinear-change}{%
\subsection{Suggestions for Modelling Nonlinear Change}\label{suggestions-for-modelling-nonlinear-change}}

\hypertarget{conclusion}{%
\section{Conclusion}\label{conclusion}}

\newpage
\renewcommand\bibname{References}
\phantomsection
\addcontentsline{toc}{chapter}{References}
\printbibliography

%change numbering for figures, tables, and equations for appendices
\renewcommand\thefigure{\theapp.\arabic{figure}} %change figure numbering for appendix such that it goes A.1, A.2, etc.
\counterwithin{figure}{app} %reset figure number counter for each appendix

\renewcommand\thetable{\theapp.\arabic{table}} %change figure numbering for appendix such that it goes A.1, A.2, etc.
\counterwithin{table}{app} %reset figure number counter for each appendix

%reset equation number number counter for each appendix
\renewcommand{\theequation}{\theapp.\arabic{equation}}
\counterwithin{equation}{app} %reset figure number counter for each appendix

\counterwithin{chunk}{app} %reset code chunk numbering

\appendix

\newpage

\app{Ergodicity and the Need to Conduct Longitudinal Research}

\label{ergodicity}

To understand why cross-sectional results are unlikely to agree with longitudinal results for any given analysis, a discussion of data structures is apropos. Consider an example where a researcher obtains data from 50 people measured over 100 time points such that each row contains a \(p\) person's data over the 100 time points and each column contains data from 50 people at a \(t\) time point. For didactic purposes, all data are assumed to be sampled from a normal distribution. To understand whether findings in any given cross-sectional data set yield the same findings in any given longitudinal data set, the researcher randomly samples one cross-sectional and one longitudinal data set and computes the mean and variance in each set. To conduct a cross-sectional analysis, the researcher randomly samples the data across the 50 people at a given time point and computes a mean of the scores at the sampled time point (\(\bar{X}_t\)) using Equation \ref{eq:cross-mean} shown below:
\begin{align}
\bar{X}_t = \frac{1}{P}\sum^P_{p = 1} x_p,
\label{eq:cross-mean}
\end{align}
\noindent where the scores of all \(P\) people are summed (\(x_p\)) and then divided by the number of people (\(P\)). To compute the variance of the scores at the sampled time point (\(S^2_t\)), the researcher uses Equation \ref{eq:cross-variance} shown below:
\begin{align}
S^2_t = \frac{1}{P}\sum^P_{p = 1} (x_p - \bar{X}_t)^2,
\label{eq:cross-variance}
\end{align}
\noindent where the sum of the squared differences between each person's score (\(x_p\)) and the average value at the given \(t\) time point (\(\bar{X}_t\)) is computed and then divided by the number of people (\(P\)). To conduct a longitudinal analysis, the researcher randomly samples one person's data across the 100 time points and also computes a mean and variance of the scores. To compute the mean across the \(t\) time points of the longitudinal data set (\(\bar{X}_p\)), the researcher uses Equation \ref{eq:long-mean} shown below:
\begin{align}
\bar{X}_p = \frac{1}{T}\sum^T_{t = 1} x_t,
\label{eq:long-mean}
\end{align}
\noindent where the scores at each \(t\) time point are summed (\(x_t\)) and then divided by the number of time points (\(T\)). The researcher also computes a variance of the sampled person's scores across all time points (\(S^2_p\)) using Equation \ref{eq:long-variance} shown below:
\begin{align}
S^2_p = \frac{1}{T}\sum^T_{t = 1} (x_t - \bar{X}_p)^2,
\label{eq:long-variance}
\end{align}
\noindent where the sum of squared differences between the score at each time point (\(x_t\)) and the average value of the \(p\) person's scores (\(\bar{X}_p\)) is computed and then divided by the number of time points (\(T\)).

If the researcher wants treat the mean and variance from the cross-sectional and longitudinal data sets as interchangeable, then two conditions outlined by ergodic theory must be satisfied \autocite{molenaar2004,molenaar2009}.\footnote{Note that ergodic theory is an entire mathematical discipline \parencite[for an introduction, see][]{petersen1983}. In the current context, the most important ergodic theorems are those proven by \citeauthor{birkhoff1931} \parentext{\cite*{birkhoff1931}, \cite[for a review, see][Chapter 3]{choe2005}}} First, a given cross-sectional mean and variance can only closely estimate the mean and variance of any given person's data (i.e., a longitudinal data set) to the extent that each person's data originate from a normal distribution with the same mean and variance. If each person's data originate from a different normal distribution, then computing the mean and variance at a given time point would, at best, describe the values of one person. When each person's data are generated from the same normal distribution, the condition of \emph{homogeneity} is met. Importantly, satisfying the condition of homogeneity does not guarantee that the mean and variance obtained from another cross-sectional data set will closely estimate the mean and variance of any given person (i.e., any given longitudinal data set). The mean and variance values computed from any given cross-sectional data set can only closely estimate the values of any given person to the extent that the cross-sectional mean and variance remain constant over time. If the mean and variance of observations remain constant over time, then the second condition of \emph{stationarity} is satisfied. Therefore, the researcher can only treat means and variances from cross-sectional and longitudinal data sets as interchangeable if each person's data are generated from the same normal distribution (homogeneity) and if the mean and variance remain constant over time (stationarity). When the conditions of homogeneity and stationarity are satisfied, a process is said to be \emph{ergodic}: Analyses of cross-sectional data sets will return the same values as analyses on longitudinal data sets.

Given that psychological studies almost never collect data from only one person, one potential reservation may be that the conditions required for ergodicity only hold when a longitudinal data set contains the data of one person. That is, if the researcher uses the full data set containing the data of 100 people sampled over 100 time points and computes 100 cross-sectional means and variances (Equation \ref{eq:cross-mean} and Equation \ref{eq:cross-variance}, respectively) and 100 longitudinal means and variances (Equation \ref{eq:long-mean} and Equation \ref{eq:long-variance}, respectively), wouldn't the average of the cross-sectional means and variances be the same as the average of the longitudinal means and variances? Although averaging the cross-sectional means returns the same value as averaging the longitudinal means, the average longitudinal variance remains different from the average cross-sectional variance \autocite[for several empirical examples, see][]{fisher2018}. Therefore, the conditions of ergodicity apply even with larger longitudinal and cross-sectional sample sizes.

The guaranteed differences in cross-sectional and longitudinal variance values that result from non-ergodic processes have far-reaching implications. Almost every analysis employed in organizational research---whether it be correlation, regression, factor analysis, mediation, etc.---analyzes variability, and so, when a process is non-ergodic, cross-sectional variability will differ from longitudinal variability, and the results obtained from applying any given analysis on each of the variabilities will differ as a consequence. Because variability is central to so many analyses, the non-equivalence of longitudinal and cross-sectional variances that results from a non-ergodic process explains why discussions of ergodicity often point out that ``for non-ergodic processes, an analysis of the structure of IEV {[}interindividual variability{]} will yield results that differ from results obtained in an analogous analysis of IAV {[}intraindividual variability{]}''\autocite[p.~202]{molenaar2004}.\footnote{It is important to note that a violation of one or both ergodic conditions (homogeneity and stationarity) does not mean that an analysis of cross-sectional variability yields results that have no relation to the results gained from applying the analysis on longitudinal variability (i.e., the causes of cross-sectional variability are independent from the causes of longitudinal variability). An analysis of cross-sectional variability can still give insight into temporal dynamics if the causes of non-ergodicity can be identified \parencites{voelkle2014}[for similar discussion, see][]{spector2019}. Thus, conceptualizing ergodicity on a continuum with non-erdogicity and ergodicity on opposite ends provides a more accurate perspective for understanding ergodicity \parencites{adolf2019}{medaglia2019}.}

With an understanding of the conditions required for ergodicity, a brief review of organizational phenomena finds that these conditions are regularly violated. Focusing only on homogeneity (each person's data are generated from the same distribution), several instances in organizational research violate this condition. As examples of homogeneity violations, employees show different patterns of absenteeism over five years \autocite{magee2016}, leadership development over the course of a seminar \autocite{day2011}, career stress over the course of 10 years \autocite{igic2017}, and job performance in response to organizational restructuring \autocite{miraglia2015}. With respect to stationarity (constant values for statistical parameters across people over time), several examples can be generated by realizing how calendar events affect psychological processes and behaviours throughout the year. As examples of stationarity violations, consider how salespeople, on average, undoubtedly sell more products during holidays, how employees, on average, take more sick days during the winter months, and how accountants, on average, experience more stress during tax season. With violations of ergodic conditions commonly occurring in organizational psychology, it becomes fitting to echo the commonly held sentiment that few, if any, psychological processes are ergodic \autocite{molenaar2004,molenaar2008,molenaar2009,fisher2018,curran2011,wang2015,hamaker2012}. As a result, longitudinal research is necessary for understanding psychological processes.

\app{Code Used to Run Monte Carlo Simulations for all Experiments}

\label{simulation-code}

The code used to compute the simulations of each experiment are shown in Code Block \ref{sim-code}. Note that the cell size is 1000 (i.e., \texttt{num\_iterations\ =\ 1000}).

\captionof{chunk}{Code Use to Run Monte Carlo Simulations for Each Simulation Experiment}\restoreparindent\label{sim-code}
\begin{Shaded}
\begin{Highlighting}[numbers=left,,]
\NormalTok{devtools}\SpecialCharTok{::}\FunctionTok{install\_github}\NormalTok{(}\AttributeTok{repo =} \StringTok{\textquotesingle{}sciarraseb/nonlinSims\textquotesingle{}}\NormalTok{, }\AttributeTok{force=}\NormalTok{T)}

\FunctionTok{library}\NormalTok{(easypackages)}
\NormalTok{packages }\OtherTok{\textless{}{-}} \FunctionTok{c}\NormalTok{(}\StringTok{\textquotesingle{}devtools\textquotesingle{}}\NormalTok{, }\StringTok{\textquotesingle{}nonlinSims\textquotesingle{}}\NormalTok{, }\StringTok{\textquotesingle{}parallel\textquotesingle{}}\NormalTok{, }\StringTok{\textquotesingle{}tidyverse\textquotesingle{}}\NormalTok{, }\StringTok{"OpenMx"}\NormalTok{, }\StringTok{"data.table"}\NormalTok{, }\StringTok{\textquotesingle{}progress\textquotesingle{}}\NormalTok{, }\StringTok{\textquotesingle{}tictoc\textquotesingle{}}\NormalTok{)}
\FunctionTok{libraries}\NormalTok{(packages)}

\NormalTok{time\_period }\OtherTok{\textless{}{-}} \DecValTok{360}

\CommentTok{\#Population values for parameters }
\CommentTok{\#fixed effects}
\NormalTok{sd\_scale }\OtherTok{\textless{}{-}} \DecValTok{1}
\NormalTok{common\_effect\_size }\OtherTok{\textless{}{-}} \FloatTok{0.32}
\NormalTok{theta\_fixed }\OtherTok{\textless{}{-}} \DecValTok{3}
\NormalTok{alpha\_fixed }\OtherTok{\textless{}{-}}\NormalTok{ theta\_fixed }\SpecialCharTok{+}\NormalTok{ common\_effect\_size}
\NormalTok{beta\_fixed }\OtherTok{\textless{}{-}} \DecValTok{180}
\NormalTok{gamma\_fixed }\OtherTok{\textless{}{-}} \DecValTok{20}

\CommentTok{\#random effects }
\NormalTok{sd\_theta }\OtherTok{\textless{}{-}} \FloatTok{0.05}
\NormalTok{sd\_alpha }\OtherTok{\textless{}{-}} \FloatTok{0.05}
\NormalTok{sd\_beta }\OtherTok{\textless{}{-}} \DecValTok{10}
\NormalTok{sd\_gamma }\OtherTok{\textless{}{-}} \DecValTok{4}
\NormalTok{sd\_error }\OtherTok{\textless{}{-}} \FloatTok{0.05}

\CommentTok{\#List containing population parameter values}
\NormalTok{pop\_params\_4l }\OtherTok{\textless{}{-}} \FunctionTok{generate\_four\_param\_pop\_curve}\NormalTok{(}
  \AttributeTok{theta\_fixed =}\NormalTok{  theta\_fixed, }\AttributeTok{alpha\_fixed =}\NormalTok{ alpha\_fixed, }
   \AttributeTok{beta\_fixed =}\NormalTok{ beta\_fixed, }\AttributeTok{gamma\_fixed =}\NormalTok{ gamma\_fixed, }
   \AttributeTok{sd\_theta =}\NormalTok{ sd\_theta, }\AttributeTok{sd\_alpha =}\NormalTok{ sd\_alpha, }
   \AttributeTok{sd\_beta =}\NormalTok{ sd\_beta, }\AttributeTok{sd\_gamma =}\NormalTok{ sd\_gamma, }\AttributeTok{sd\_error =}\NormalTok{ sd\_error}



\NormalTok{num\_iterations }\OtherTok{\textless{}{-}} \FloatTok{1e3} \CommentTok{\#n=1000 (cell size)}
\NormalTok{seed }\OtherTok{\textless{}{-}} \DecValTok{27} \CommentTok{\#ensures replicability }

\CommentTok{\# Experiment 1 (number measurements,  spacing, midpoint) {-}{-}{-}{-}{-}{-}{-}{-}}
\NormalTok{factor\_list\_exp\_1 }\OtherTok{\textless{}{-}} \FunctionTok{list}\NormalTok{(}\StringTok{\textquotesingle{}num\_measurements\textquotesingle{}} \OtherTok{=} \FunctionTok{seq}\NormalTok{(}\AttributeTok{from =} \DecValTok{5}\NormalTok{, }\AttributeTok{to =} \DecValTok{11}\NormalTok{, }\AttributeTok{by =} \DecValTok{2}\NormalTok{), }
                          \StringTok{\textquotesingle{}time\_structuredness\textquotesingle{}} \OtherTok{=} \FunctionTok{c}\NormalTok{(}\StringTok{\textquotesingle{}time\_structured\textquotesingle{}}\NormalTok{),}
                          \StringTok{\textquotesingle{}spacing\textquotesingle{}} \OtherTok{=} \FunctionTok{c}\NormalTok{(}\StringTok{\textquotesingle{}equal\textquotesingle{}}\NormalTok{, }\StringTok{\textquotesingle{}time\_inc\textquotesingle{}}\NormalTok{, }\StringTok{\textquotesingle{}time\_dec\textquotesingle{}}\NormalTok{, }\StringTok{\textquotesingle{}mid\_ext\textquotesingle{}}\NormalTok{), }
                          \StringTok{\textquotesingle{}midpoint\textquotesingle{}} \OtherTok{=} \FunctionTok{c}\NormalTok{(}\DecValTok{80}\NormalTok{, }\DecValTok{180}\NormalTok{, }\DecValTok{280}\NormalTok{),  }
                          \StringTok{\textquotesingle{}sample\_size\textquotesingle{}} \OtherTok{=} \FunctionTok{c}\NormalTok{(}\DecValTok{225}\NormalTok{))}

\FunctionTok{tic}\NormalTok{()}
\NormalTok{exp\_1\_data }\OtherTok{\textless{}{-}} \FunctionTok{run\_exp\_simulation}\NormalTok{(}\AttributeTok{factor\_list =}\NormalTok{ factor\_list\_exp\_1, }\AttributeTok{num\_iterations =}\NormalTok{ num\_iterations, }\AttributeTok{pop\_params =}\NormalTok{ pop\_params\_4l, }
                                 \AttributeTok{num\_cores =} \FunctionTok{detectCores}\NormalTok{()}\SpecialCharTok{{-}}\DecValTok{1}\NormalTok{, }\AttributeTok{seed =}\NormalTok{ seed)}
\FunctionTok{toc}\NormalTok{()}

\CommentTok{\#Average computation time is 1 iteration per second. As an example, Experiment has 48 cells x 1000 iterations/cell = 48 000 iterations and seconds/3600s/hour \textasciitilde{} 13.33 hours (simulations computed with 15 cores)}
\FunctionTok{write\_csv}\NormalTok{(}\AttributeTok{x =}\NormalTok{ exp\_1\_data, }\AttributeTok{file =} \StringTok{\textquotesingle{}\textasciitilde{}/Desktop/exp\_1\_data.csv\textquotesingle{}}\NormalTok{)}

\CommentTok{\# Experiment 2 (number measurements, spacing,  sample size) {-}{-}{-}}
\NormalTok{factor\_list\_exp\_2 }\OtherTok{\textless{}{-}} \FunctionTok{list}\NormalTok{(}\StringTok{\textquotesingle{}num\_measurements\textquotesingle{}} \OtherTok{=} \FunctionTok{seq}\NormalTok{(}\AttributeTok{from =} \DecValTok{5}\NormalTok{, }\AttributeTok{to =} \DecValTok{11}\NormalTok{, }\AttributeTok{by =} \DecValTok{2}\NormalTok{), }
                          \StringTok{\textquotesingle{}time\_structuredness\textquotesingle{}} \OtherTok{=} \FunctionTok{c}\NormalTok{(}\StringTok{\textquotesingle{}time\_structured\textquotesingle{}}\NormalTok{),}
                          \StringTok{\textquotesingle{}spacing\textquotesingle{}} \OtherTok{=} \FunctionTok{c}\NormalTok{(}\StringTok{\textquotesingle{}equal\textquotesingle{}}\NormalTok{, }\StringTok{\textquotesingle{}time\_inc\textquotesingle{}}\NormalTok{, }\StringTok{\textquotesingle{}time\_dec\textquotesingle{}}\NormalTok{, }\StringTok{\textquotesingle{}mid\_ext\textquotesingle{}}\NormalTok{),}
                          \StringTok{\textquotesingle{}midpoint\textquotesingle{}} \OtherTok{=} \DecValTok{180}\NormalTok{, }
                          \StringTok{\textquotesingle{}sample\_size\textquotesingle{}} \OtherTok{=} \FunctionTok{c}\NormalTok{(}\DecValTok{30}\NormalTok{, }\DecValTok{50}\NormalTok{, }\DecValTok{100}\NormalTok{, }\DecValTok{200}\NormalTok{, }\DecValTok{500}\NormalTok{, }\DecValTok{1000}\NormalTok{))}

\FunctionTok{tic}\NormalTok{()}
\NormalTok{exp\_2\_data }\OtherTok{\textless{}{-}} \FunctionTok{run\_exp\_simulation}\NormalTok{(}\AttributeTok{factor\_list =}\NormalTok{ factor\_list\_exp\_2, }\AttributeTok{num\_iterations =}\NormalTok{ num\_iterations, }\AttributeTok{pop\_params =}\NormalTok{ pop\_params\_4l, }
                                 \AttributeTok{num\_cores =} \FunctionTok{detectCores}\NormalTok{(), }\AttributeTok{seed =}\NormalTok{ seed)}
\FunctionTok{toc}\NormalTok{()}

\FunctionTok{write\_csv}\NormalTok{(}\AttributeTok{x =}\NormalTok{ exp\_2\_data, }\AttributeTok{file =} \StringTok{\textquotesingle{}Desktop/exp\_2\_data.csv\textquotesingle{}}\NormalTok{)}

\CommentTok{\# Experiment 3 (number measurements, sample size, time structuredness) {-}{-}{-}{-}{-}}
\NormalTok{factor\_list\_exp\_3 }\OtherTok{\textless{}{-}} \FunctionTok{list}\NormalTok{(}\StringTok{\textquotesingle{}num\_measurements\textquotesingle{}} \OtherTok{=} \FunctionTok{seq}\NormalTok{(}\AttributeTok{from =} \DecValTok{5}\NormalTok{, }\AttributeTok{to =} \DecValTok{11}\NormalTok{, }\AttributeTok{by =} \DecValTok{2}\NormalTok{), }
                          \StringTok{\textquotesingle{}time\_structuredness\textquotesingle{}} \OtherTok{=} \FunctionTok{c}\NormalTok{(}\StringTok{\textquotesingle{}time\_structured\textquotesingle{}}\NormalTok{, }\StringTok{\textquotesingle{}fast\_response\textquotesingle{}}\NormalTok{, }\StringTok{\textquotesingle{}slow\_response\textquotesingle{}}\NormalTok{),}
                          \StringTok{\textquotesingle{}spacing\textquotesingle{}} \OtherTok{=} \FunctionTok{c}\NormalTok{(}\StringTok{\textquotesingle{}equal\textquotesingle{}}\NormalTok{), }
                          \StringTok{\textquotesingle{}midpoint\textquotesingle{}} \OtherTok{=} \DecValTok{180}\NormalTok{, }
                          \StringTok{\textquotesingle{}sample\_size\textquotesingle{}} \OtherTok{=} \FunctionTok{c}\NormalTok{(}\DecValTok{30}\NormalTok{, }\DecValTok{50}\NormalTok{, }\DecValTok{100}\NormalTok{, }\DecValTok{200}\NormalTok{, }\DecValTok{500}\NormalTok{, }\DecValTok{1000}\NormalTok{))}
\FunctionTok{tic}\NormalTok{()}
\NormalTok{exp\_3\_data }\OtherTok{\textless{}{-}} \FunctionTok{run\_exp\_simulation}\NormalTok{(}\AttributeTok{factor\_list =}\NormalTok{ factor\_list\_exp\_3, }\AttributeTok{num\_iterations =}\NormalTok{ num\_iterations, }\AttributeTok{pop\_params =}\NormalTok{ pop\_params\_4l, }
                                 \AttributeTok{num\_cores =} \FunctionTok{detectCores}\NormalTok{(), }\AttributeTok{seed =}\NormalTok{ seed)}
\FunctionTok{toc}\NormalTok{()                }

\FunctionTok{write\_csv}\NormalTok{(}\AttributeTok{x =}\NormalTok{ exp\_3\_data, }\AttributeTok{file =} \StringTok{\textquotesingle{}\textasciitilde{}/Desktop/exp\_3\_data.csv\textquotesingle{}}\NormalTok{)}



\CommentTok{\# Experiment 3 (definition variables with slow response rate ) {-}{-}{-}{-}{-}{-}}
\NormalTok{factor\_list\_exp\_def }\OtherTok{\textless{}{-}} \FunctionTok{list}\NormalTok{(}\StringTok{\textquotesingle{}num\_measurements\textquotesingle{}} \OtherTok{=} \FunctionTok{seq}\NormalTok{(}\AttributeTok{from =} \DecValTok{5}\NormalTok{, }\AttributeTok{to =} \DecValTok{11}\NormalTok{, }\AttributeTok{by =} \DecValTok{2}\NormalTok{), }
                            \StringTok{\textquotesingle{}time\_structuredness\textquotesingle{}} \OtherTok{=} \FunctionTok{c}\NormalTok{(}\StringTok{\textquotesingle{}slow\_response\textquotesingle{}}\NormalTok{),}
                            \StringTok{\textquotesingle{}spacing\textquotesingle{}} \OtherTok{=} \FunctionTok{c}\NormalTok{(}\StringTok{\textquotesingle{}equal\textquotesingle{}}\NormalTok{), }
                            \StringTok{\textquotesingle{}midpoint\textquotesingle{}} \OtherTok{=} \DecValTok{180}\NormalTok{, }
                            \StringTok{\textquotesingle{}sample\_size\textquotesingle{}} \OtherTok{=} \FunctionTok{c}\NormalTok{(}\DecValTok{30}\NormalTok{, }\DecValTok{50}\NormalTok{, }\DecValTok{100}\NormalTok{, }\DecValTok{200}\NormalTok{, }\DecValTok{500}\NormalTok{, }\DecValTok{1000}\NormalTok{))}
\FunctionTok{tic}\NormalTok{()}
\NormalTok{exp\_3\_def\_data }\OtherTok{\textless{}{-}} \FunctionTok{run\_exp\_simulation}\NormalTok{(}\AttributeTok{factor\_list =}\NormalTok{ factor\_list\_exp\_def, }\AttributeTok{num\_iterations =}\NormalTok{ num\_iterations, }\AttributeTok{pop\_params =}\NormalTok{ pop\_params\_4l, }
                                     \AttributeTok{num\_cores =} \FunctionTok{detectCores}\NormalTok{() }\SpecialCharTok{{-}} \DecValTok{1}\NormalTok{, }\AttributeTok{seed =}\NormalTok{ seed, }\AttributeTok{definition =}\NormalTok{ T)}
\FunctionTok{toc}\NormalTok{()                }
\CommentTok{\#240734.993 sec elapsed (7 cores used; simulation time increased by roughly a magnitude of 8). }
\FunctionTok{write\_csv}\NormalTok{(}\AttributeTok{x =}\NormalTok{ exp\_3\_def\_data, }\AttributeTok{file =} \StringTok{\textquotesingle{}exp\_3\_def.csv\textquotesingle{}}\NormalTok{)}
\end{Highlighting}
\end{Shaded}
\app{Procedure for Generating Measurement Schedules Measurement Schedules}

\label{measurement-schedules}

Given that no procedure existed (to my knowledge) for creating measurement schedules, I devised a method for generating measurement schedules for the four spacing conditions (equal, time-interval increasing, time-interval decreasing, and middle-and-extreme spacing). The code I used to automate the generation of these schedules can be found within the code for the \texttt{compute\_measurement\_schedules()} function for the \texttt{nonlinSims} package (see \url{https://github.com/sciarraseb/nonlinSims}). For each measurement spacing conditions across all measurement number levels, a two-step procedure was employed to generate measurement schedules in Experiments 1 and 2. At a broad level, the first step computes values for setup variables and the second step computes the interval lengths.

\secapp{Procedure for Constructing Measurement Schedules With Equal Spacing}

Figure \ref{fig:equal_spacing_schedule} shows how the two-step procedure
is implemented to construct a measurement schedule with equal spacing
and five measurements. In the first step, the number of intervals (\(NI\))
is computed by subtracting one from the number of measurements (\(NM\)). With five measurements (\(NM = 5\)), there are four intervals (\(NI = 4\)). In the second step, interval lengths are calculated by dividing the length
of the measurement period (\(MP\)) by the number of intervals (\(NI\)),
yielding an interval length of 90 days (\(\frac{MP}{NI} = \frac{360}{4} = 90\)) for each interval and the following measurement days:
\begin{itemize}
\tightlist
\item
  \(m_1\) = day 0
\item
  \(m_2\) = day 90
\item
  \(m_3\) = day 180
\item
  \(m_4\) = day 270
\item
  \(m_5\) = day 360.
\end{itemize}
\secapp{Procedure for Constructing Measurement Schedules With Time-Interval Increasing Spacing}

\label{time-inc-proc}

Figure \ref{fig:time_inc_diagram} shows how the two-step procedure is implemented to construct a measurement schedule with time-interval increasing spacing and five measurements. In the first step, the number of intervals (\(NI\)) is computed by subtracting one from the number of measurements (\(NM\)). With five measurements (\(NM = 5\)), there are four intervals
\begin{apaFigure}
[landscape]
[samepage]
{Procedure for Computing Measurement Schedules With Equal Spacing}
{equal_spacing_schedule}
{0.90}
{Figures/equal_spacing_schedule}
{In Step 1, setup variables are calculated. With five measurements ($NM = 5$), there are four intervals ($NI = 4$). In Step 2, interval lengths are calculated by dividing the length of the measurement period ($MP$) by the number of intervals ($NI$), yielding an interval length of 90 days ($\frac{MP}{NI} = \frac{360}{4} = 90$) for each interval.}
\end{apaFigure}
\noindent (\(NI = 4\)). Because interval lengths increase over time, I decided that intervals would increase by an integer multiple of a constant length (\(c\)) after each measurement day (\(m_i\)) according to the function shown below in Equation \ref{eq:constant-length}:
\begin{align}
\text{Constant-length increment} = \sum^{NI - 1}_{x=0} xc, 
  \label{eq:constant-length} 
\end{align}
where \(x\) represents the integer multiple that increases by 1 after each measurement day. Importantly, to calculate the constant-length increment (\(c\)) by which interval lengths increase over time, it is important to realize that two terms contribute to the length of any interval: A shortest-interval length (\(s\)) and a constant-length value (\(c\)), as shown below in Equation \ref{eq:time-inc-interval-length}:
\begin{align}
\text{Interval length} = s + \sum^{NI - 1}_{x=0} xc. 
  \label{eq:time-inc-interval-length} 
\end{align}
\noindent Because the shortest-interval length (\(s\)) contributes to the length of each interval--in this example, four intervals---then the sum of these lengths can be subtracted from the measurement period length of 360 days (\(MP = 360\)). In the current example with five measurements, 240 days remain (\(r = 240\)) after subtracting the days needed for the shortest-interval lengths (see Equation \ref{eq:eq-remaining}).
\begin{align}
\text{Remaining days (} \textit{r})  = MP - (NI)s = 360 - (30)4 = 240 \text{ days}
  \label{eq:eq-remaining} 
\end{align}
\noindent Having computed the number of remaining days, the constant-length value (\(c\)) can then be obtained by dividing the number of remaining days by the number of constant-value interval lengths (\(c_i\)), as shown below in Equation \ref{eq:eq-constant}:
\begin{align}
\text{Constant-value interval length(} \textit{c}\text{)} = \frac{r}{\sum^{NI - 2}_{i = 0}i} = \frac{240}{3 + 2 + 1} = \text{40 days}
  \label{eq:eq-constant} 
\end{align}
\noindent Therefore, having computed the value for \(c\), the following interval lengths are obtained:
\begin{itemize}
\tightlist
\item
  \(i_{1} = s + 0(c) = 30 + 0(30)\) = 30 days
\item
  \(i_{2} = s + 1(c) = 30 + 1(40)\) = 70 days
\item
  \(i_{3} = s + 0(c) = 30 + 2(40)\) = 110 days
\item
  \(i_{4} = s + 0(c) = 30 + 3(40)\) = 150 days
\end{itemize}
\noindent and the following measurement days are obtained:
\begin{itemize}
\tightlist
\item
  \(m_1\) = day 0
\item
  \(m_2\) = day 30
\item
  \(m_3\) = day 100
\item
  \(m_4\) = day 210
\item
  \(m_5\) = day 360.
\end{itemize}
\secapp{Procedure for Constructing Measurement Schedules With Time-Interval Decreasing Spacing}

Figure \ref{fig:time_dec_diagram} shows how the two-step procedure
is implemented to construct a measurement schedule with time-interval decreasing spacing and five measurements. Because the procedure for calculating time-decreasing intervals simply requires that the order of time-interval increasing intervals are reversed, the procedure is, thus, essentially identical to the procedure shown in the \protect\hyperlink{time-inc-proc}{previous section}. Therefore, with five measurements, time-interval decreasing spacing produces the following intervals:
\begin{apaFigure}
[landscape]
[samepage]
{Procedure for Computing Measurement Schedules With Time-Interval Increasing Spacing}
{time_inc_diagram}
{0.65}
{Figures/time_inc_schedule}
{In Step 1, setup variables are calculated. With five measurements ($NM = 5$), there are four intervals ($NI = 4$). In Step 2, two components contribute to each interval length: A shortest-interval length ($s$) and a constant-length value ($c$), as shown in Equation \ref{eq:time-inc-interval-length}. Because the shortest-interval length ($s$) contributes to each interval, the sum of these lengths can be subtracted from the measurement period length of 360 days ($MP = 360$). In the current example with five measurements, 240 days remain ($r = 240$) after subtracting the days needed for the shortest-interval lengths (see Equation \ref{eq:eq-remaining}). To calculate the constant-length value ($c$), the remaining days ($r$) are divided by the number of constant-value interval lengths ($c_i$), as shown in Equation \ref{eq:eq-constant}.}
\end{apaFigure}
\begin{itemize}
\tightlist
\item
  \(i_{1} = s + 0(c) = 30 + 3(40)\) = 150 days
\item
  \(i_{2} = s + 0(c) = 30 + 2(40)\) = 110 days
\item
  \(i_{3} = s + 1(c) = 30 + 1(40)\) = 70 days
\item
  \(i_{4} = s + 0(c) = 30 + 0(30)\) = 30 days
\end{itemize}
\noindent and the following measurement days are obtained:
\begin{itemize}
\tightlist
\item
  \(m_1\) = day 0
\item
  \(m_2\) = day 150
\item
  \(m_3\) = day 260
\item
  \(m_4\) = day 330
\item
  \(m_5\) = day 360.
\end{itemize}
\secapp{Procedure for Constructing Measurement Schedules With Middle-and-Extreme Spacing}

Figure \ref{fig:mid_ext_diagram} shows how the two-step procedure is implemented to construct a measurement schedule with middle-and-extreme spacing and five measurements. In the first step, the number of intervals (\(NI\)) is computed by subtracting one from the number of measurements (\(NM\)). With five measurements (\(NM = 5\)), there are four intervals (\(NI = 4\)). Importantly, because middle-and-extreme spacing places measurements near the extremities and the middle of the measurement window, the number of measurements in both these sections must also be calculated. The number of extreme measurements is first calculated by dividing the number of measurements by 3 and taking the floor (i.e., rounded-down value {[}\(\lfloor x\rfloor\){]}) of this value and multiplying it by 2, as shown below in Equation \ref{eq:extreme}:
\begin{apaFigure}
[landscape]
[samepage]
{Procedure for Computing Measurement Schedules With Time-Interval Decreasing Spacing}
{time_dec_diagram}
{0.65}
{Figures/time_dec_schedule}
{In Step 1, setup variables are calculated. With five measurements ($NM = 5$), there are four intervals ($NI = 4$). In Step 2, two components contribute to each interval length: A shortest-interval length ($s$) and a constant-length value ($c$), as shown in Equation \ref{eq:time-inc-interval-length}. Because the shortest-interval length ($s$) contributes to each interval, the sum of these lengths can be subtracted from the measurement period length of 360 days ($MP = 360$). In the current example with five measurements, 240 days remain ($r = 240$) after subtracting the days needed for the shortest-interval lengths (see Equation \ref{eq:eq-remaining}). To calculate the constant-length value ($c$), the remaining days ($r$) are divided by the number of constant-value interval lengths ($c_i$), as shown in Equation \ref{eq:eq-constant}.}
\end{apaFigure}
\begin{align}
\text{Number of extreme measurements(} \textit{ex}\text{)} = 2\lfloor\frac{NM}{3}\rfloor = 2\lfloor\frac{5}{3}\rfloor = 2.
  \label{eq:extreme} 
\end{align}
\noindent The number of middle measurements can then be calculated by subtracting the number of extreme measurements (\(ex\)) from the number of measurements (\(NM\)), as shown below in Equation \ref{eq:middle}:
\begin{align}
\text{Number of middle measurements(} \textit{mi}\text{)} = NM - ex = 5 - 2 = 3.
  \label{eq:middle} 
\end{align}
In Step 2, interval lengths are calculated. For middle-and-extreme spacing, there are two types of interval lengths: 1) Intervals separating either two middle or two extreme measurements and 2) intervals separating one middle and one extreme measurement. Intervals separating two middle or two extreme measurements (\(w_i\)) are set to the shortest-interval length (\(s\)), which I set to be 30 days (\(w_i = s = 30\)). Intervals separating one middle and one extreme measurement (\(b_i\)) are set to the sum of two components: 1) A shortest-interval length (\(s\)) and a 2) constant-value interval length (\(c\)), as shown below in Equation \ref{eq:middle}:
\begin{align}
b_i = s + c.
  \label{eq:middle} 
\end{align}
\noindent To obtain the constant-value interval length (\(c\)), the sum of shortest-value interval lengths (\(s\)) is subtracted from the measurement period of 360 days (\(MP = 360\)). In the current example with five measurements, 240 days remain (\(r = 240\)) after subtracting the days needed for the shortest-interval lengths (see Equation \ref{eq:eq-remaining-mid}).
\begin{align}
\text{Remaining days (} \textit{r})  = MP - (NI)s = 360 - (30)4 = 240 \text{ days}
  \label{eq:eq-remaining-mid} 
\end{align}
\noindent Having computed the number of remaining days, the constant-length value (\(c\)) can then be obtained by dividing the number of remaining days by the number of intervals separating middle and extreme measurements, which will always be 2, as shown below in Equation \ref{eq:eq-constant-mid}:
\begin{align}
\text{Constant-value interval length(} \textit{c}\text{)} = \frac{r}{2} = \frac{240}{2} = \text{120 days}
  \label{eq:eq-constant-mid} 
\end{align}
\noindent Therefore, having computed the value for \(c\), the following interval lengths are obtained:
\begin{itemize}
\tightlist
\item
  \(b_{1} = s + c = 30 + 120\) = 150 days
\item
  \(w_{1} = s = 30\) = 30 days
\item
  \(w_{2} = s = 30\) = 30 days
\item
  \(b_{2} = s + c = 30 + 120\) = 150 days
\end{itemize}
\noindent and the following measurement days are obtained:
\begin{itemize}
\tightlist
\item
  \(m_1\) = day 0
\item
  \(m_2\) = day 150
\item
  \(m_3\) = day 180
\item
  \(m_4\) = day 21
\item
  \(m_5\) = day 360.
\end{itemize}
\begin{apaFigure}
[landscape]
[samepage]
[0cm]
{Procedure for Computing Measurement Schedules With Middle-and-Extreme Spacing}
{mid_ext_diagram}
{0.65}
{Figures/mid_ext_schedule}
{In Step 1, setup variables are calculated. With five measurements ($NM = 5$), there are four intervals ($NI = 4$). Importantly, because middle-and-extreme spacing places measurements near the extremities and the middle of the measurement window, the number of measurements in both these sections must also be calculated. The number of extreme measurements is first calculated by dividing the number of measurements by 3 and taking the taking the floor (i.e., rounded-down value [$\lfloor x\rfloor$]) of this value and multiplying it by 2 (see Equation \ref{eq:extreme}). The number of middle measurements can then be calculated by subtracting the number of extreme measurements ($ex$) from the number of measurements ($NM$; see Equation \ref{eq:middle}). In Step 2, interval lengths are calculated. For middle-and-extreme spacing, there are two types of interval lengths: 1) Intervals separating either two middle or two extreme measurements and 2) intervals separating one middle and one extreme measurement. Intervals separating two middle or two extreme measurements are set to the shortest-interval length ($s$), which I set to be 30 days ($w_i = s = 30$). Intervals separating one middle and one extreme measurement are set to the sum of two components: 1) A shortest-interval length ($s$) and a 2) constant-value interval length ($c$; see Equation \ref{eq:eq-constant-mid}). To obtain the constant-value interval length ($c$), the sum of shortest-value interval lengths ($s$) is subtracted from the measurement period of 360 days ($MP = 360$). In the current example with five measurements, 240 days remain ($r = 240$) after subtracting the days needed for the shortest-interval lengths (see Equation \ref{eq:eq-remaining-mid}). Having computed the number of remaining days, the constant-length value ($c$) can then be obtained by dividing the number of remaining days by the number of intervals separating middle and extreme measurements, which will always be 2 (see Equation \ref{eq:eq-constant-mid}).}
\end{apaFigure}
\app{Using Nonlinear Function in the Structural Equation Modelling Framework}

\label{structured-lgc}
\secapp{Nonlinear Latent Growth Curve Model Used to Analyze Each Generated Data Set}

The sections that follow will first review the framework used to build
latent growth curve models and then explain how nonlinear functions can
be modified to fit into this framework.

\subapp{Brief Review of the Latent Growth Curve Model}

The latent growth curve model proposed by \textcite{meredith1990} is briefly reviewed here \autocite[for a review, see][]{preacher2008}. Consider an example where
data are collected at five time points (\(T = 5\)) from \(p\) people (\(\mathbf{y_p} = [y_1, y_2, y_3, y_4, y_5]\)). A simple model to fit is
one where change over time is defined by a straight line and each person's pattern of change is some variation of this straight line. In
modelling parlance, an intercept-slope model is fit where both the intercept and slope are random effects whose values are allowed to vary for
each person.

To fit a random-effect intercept-slope model, a general linear pattern can first be specified in the \(\mathbf{\Uplambda}\) matrix
shown below in Equation \ref{eq:int-slope-mat}:
\begin{align}
\mathbf{\Uplambda} = 
\begin{bmatrix}
1 & 0 \\ 
1 & 1 \\ 
1 & 2 \\ 
1 & 3 \\
1 & 4 \\
\end{bmatrix}.
 \label{eq:int-slope-mat}
\end{align}
\noindent In each column of \(\mathbf{\Uplambda}\), the effect a parameter is specified over the five time points; that is, \(\mathbf{\Uplambda}\) is a matrix with two columns (one for the intercept and one for the slope parameter) and five rows (one for each time point).\footnote{The columns of $\mathbf{\Uplambda}$ are often called basis curves \parencite{blozis2004} or basis functions \parencites{meredith1990}{browne1993} because each column specifies a particular component of change.} The first column of \(\mathbf{\Uplambda}\) specifies the intercept parameter. Because the effect of the intercept parameter is constant over time, a column of 1s is used to represent its effect. The second column of \(\mathbf{\Uplambda}\) specifies the slope parameter. Because a linear pattern of growth is assumed, the second column contains a series of monotonically increasing integer numbers across the time points and begins with 0. \footnote{The set of numbers specified for the slope starts at zero because there is presumably no effect of any variable at the first time point.}

To specify the intercept and slope parameters as random effects that vary across people, a weight can be applied to each column of
\(\mathbf{\Uplambda}\) and each weight can vary across people. That is, a \(p\) person's pattern of change can be reproduced with a unique set of weights in \(\mathbf{\upiota_p}\) that determines the extent to which each basis column of \(\mathbf{\Uplambda}\) contributes to the person's observed change over time. By allowing the weights for the intercept and slope parameters to vary across people, variability can be estimated in these parameters. Discrepancies between the values predicted by \(\mathbf{\Uplambda\upiota_p}\) and a person's observed scores across all
five time points are stored in an error vector \(\mathbf{\mathcal{E}_p}\). Thus, a person's observed data (\(\mathbf{y_p}\)) is
reproduced using the function shown below in Equation \ref{eq:sem-framework}:
\begin{align}
 y_p = \mathbf{\Uplambda\upiota_p} + \mathbf{\mathcal{E}_p}.
 \label{eq:sem-framework}
\end{align}
\noindent Note that Equation \ref{eq:sem-framework} defines the general structural equation modelling framework.

\subapp{Fitting a Nonlinear Function in the Structural Equation Modelling Framework}

Unfortunately, the logistic function of Equation \ref{eq:logFunction-generation}---where each parameter is estimated as a fixed- and random-effect---cannot be directly used in a latent growth curve model because it violates the linear nature of the structural equation modelling framework (Equation \ref{eq:sem-framework}). Structural equation models only permit linear combinations---specifically, the products of matrix-vector and/or matrix-matrix multiplication---and so directly fitting a nonlinear function such as the logistic function in Equation \ref{eq:logFunction-generation} is not possible.

One solution to fitting the logistic function within the structural equation modelling framework is to implement the structured latent curve modelling approach \autocites{browne1991,browne1993}[for an excellent review, see][]{preacher2015}. Briefly, the structured latent curve modelling approach constructs a Taylor series approximation of a nonlinear function so that the nonlinear function can be fit into the structural equation modelling framework (Equation \ref{eq:sem-framework}). The sections that follow will present the structured latent curve modelling approach in four parts such that 1) Taylor series approximations will first be reviewed, 2) a Taylor series approximation will then be constructed for the logistic function, 3) the logistic Taylor series approximation will be modified and fit into the structural equation modelling framework, and 4) the process of parameter estimation will be reviewed.

\subsubapp{Taylor Series': Approximations of Linear Functions}

A Taylor series uses derivative information of a nonlinear function to
construct a linear function that is an approximation of the nonlinear function.\footnote{Linear functions are
defined as functions where no parameter exists within its own partial
derivative (at any order). For example, none of the parameters in the polynomial
equation of $y = a + bt + ct^2 + dt^3$ exist within their own partial
derivative: $\frac{\partial y}{\partial a} = 1$,
$\frac{\partial y}{\partial b} = t$,
$\frac{\partial y}{\partial c} = t^2$, and
$\frac{\partial y}{\partial d} = t^3$. Conversely, the logistic function
is nonlinear because $\upbeta$ and $\upgamma$ exist in their own
partial derivatives. For example, the derivative of the logistic function  $y = \uptheta + \frac{\upalpha - \uptheta}{1 + e^{\frac{\upbeta - t}{\upgamma}}} $with respect to $\upbeta$ is $\frac{(\uptheta - \upalpha) (e^{\frac{\upbeta - t}{\upgamma}})(\frac{1}{\upgamma})}{1 + (e^{\frac{\upbeta - t}{\upgamma}})^2}$and so is nonlinear because it contains $\upbeta$.} Equation \ref{eq:taylor} shows the general formula for a Taylor series such that
\begin{align}
P^N(f(x), a)= \sum^{N}_{n = 0} \frac{f^na}{n !}(x-a)^n,
\label{eq:taylor}
\end{align}
\noindent where \(N\) is the highest derivative order of the function \(f(a)\) that is taken beginning from a zero-value derivative order (\(n=0\)), \(a\) is the point where the Taylor series is derived (i.e., the point of derivation), and \(x\) is the point where the Taylor series is evaluated (i.e., the point of evaluation). As an example of a Taylor series, consider the second-order Taylor series of \(f(x) = \cos(x)\). Note that, across the continuum of \(x\) values (i.e., from \(-\infty\) to \(\infty\)), \(\cos(x)\) returns values between -1 and 1 in an oscillatory manner. Computing the second-order Taylor series of \(f(x) = \cos(x)\) yields the following function shown in Equation \ref{eq:example-taylor}:
\begin{align} 
P^2(\cos(x), a) &=  \frac{\frac{\partial^0 \cos(a)}{\partial a^0}}{0!}(x -a)^0 + \frac{\frac{\partial^1 \cos(a)}{\partial a^1}}{1!}(x -a)^1 + \frac{\frac{\partial^2 \cos(a)}{\partial a^2}}{2!} (x -a)^2 \nonumber \\ 
&=  \frac{\cos(0)}{0!}(x -0)^0 - \frac{\sin(0)}{1!}(x -0)^1 - \frac{\cos(0)}{2!}(x -0)^2  \nonumber \\ 
&=  \frac{1}{1}1 - \frac{0}{1}x - \frac{1}{2}x^2  \nonumber \\ 
P^2(\cos(x), 0) &=  1- \frac{1}{2}x^2. 
  \label{eq:example-taylor}
\end{align}
\noindent Importantly, the second-order Taylor series of \(\cos(x)\) shown in Equation \ref{eq:example-taylor} is linear, whereas the function \(\cos(x)\) is not linear. To show that the second-order Taylor series of \(1- \frac{1}{2}x^2\) is linear, we can reformulate it by adding placeholder parameters in front of each term (\(b\) and \(c\)), resulting in the following modified equation of Equation \ref{eq:taylor-modified}:
\begin{align}
P^2_{reform}(\cos(x), a) = b1- c\frac{1}{2}x^2. 
\label{eq:taylor-modified}
\end{align}
\noindent If the partial derivative of \(P^2(\cos(x), a)\) is taken with respect to \(b\) and \(c\), no parameter exists within its own partial derivative, meaning the function is linear (see Equations \ref{eq:taylor-b}--\ref{eq:taylor-c} below).
\begin{align}
\frac{\partial P^2_{reform}(\cos(x), a)}{\partial b} = 1 \text{ and} \label{eq:taylor-b}\\
\frac{\partial P^2_{reform}(\cos(x), c)}{\partial c} = -\frac{1}{2}x^2. \label{eq:taylor-c}
\end{align}
\noindent Conversely, the fourth-order partial derivative of \(\cos(x)\) contains itself (see Equation \ref{eq:cos-four}), and so is a nonlinear function.
\begin{align}
\frac{\partial^4 \cos(x)}{\partial x^4}  = \cos(x).
\label{eq:cos-four} 
\end{align}
\noindent Therefore, Taylor series' can generate linear versions of nonlinear functions by using local derivative information.

Although Taylor series' provide linear versions of nonlinear functions, it is important to emphasize that the linear versions are approximations. More specifically, the second-order Taylor series of \(\cos(x)\)
perfectly estimates \(\cos(x)\) when the point of evaluation \(x\) is set
equal to the point of derivation \(a\), but estimates \(\cos(x)\) with an
increasing amount of error as the difference between \(x\) and \(a\)
increases (see Example \ref{exm:taylor-estimates}). Thus, Taylor series are approximations because they are only locally accurate (i.e., near the point of derivation).
\begin{example}
\protect\hypertarget{exm:taylor-estimates}{}\label{exm:taylor-estimates}Estimates of Taylor series approximation of \(f(x) = \cos(x)\) as the difference between the point of evaluation \(\mathrm{x}\) and the point of derivation \(\mathrm{a}\) increases.

\textup{Taylor series approximation of $\cos(x)$ (specifically, the second-order Taylor series; $P^2[\cos(x), a]$) estimates values that are exactly equal to the values returned by $\cos(x)$ when the point of evaluation (\textit{x}) is set to the point of derivation (\textit{a}). The example below computes the value predicted by the Taylor series approximation of $P^2[\cos(x), a]$ and by $\cos(x)$ when \textit{x} = \textit{a} = 0.}
\begin{align*}
P^2(\cos(x=0), a=0) &= \cos(x=0) \nonumber \\ 
1- \frac{1}{2}x^2 &=  \cos(0) \nonumber \\ 
1- \frac{1}{2}0^2 &=  1 \nonumber \\ 
1- 0 &=  1 \nonumber \\ 
1 &=  1 \nonumber \\ 
\end{align*}
\vspace*{-25mm}

\textup{Taylor series approximation of $\cos(x)$ (specifically, the second-order Taylor series; $P^2[\cos(x), a]$) estimates a value that is approximately equal ($\thickapprox$) to the value returned by $f\cos(x)$ when the difference between the point of evaluation \textit{x} and the point of derivation \textit{a} is small. The example below computes the value predicted by the Taylor series approximation of $P^2[\cos(x), a]$ and by $\cos(x)$ when \textit{x} = 1 and  \textit{a} = 0.}
\begin{align*}
P^2(\cos(x = 1), 0) &\thickapprox \cos(x = 1) \nonumber \\ 
1- \frac{1}{2}x^2 &\thickapprox   \cos(1) \nonumber \\ 
1- \frac{1}{2}1^2 &\thickapprox   0.54 \nonumber \\ 
1- 0.5 &\thickapprox   0.54 \nonumber \\ 
0.5 &\thickapprox 0.54 \nonumber \\ 
\end{align*}
\vspace*{-25mm}

\textup{Taylor series approximation of $f\cos(x)$ (specifically, the second-order Taylor series; $P^2[\cos(x), a]$) estimates a a value that is clearly not equal ($\neq$) to the value returned by $f\cos(x)$ when the difference between the point of evaluation \textit{x} and the point of derivation \textit{a} is large. The example below computes the value predicted by the Taylor series approximation of $P^2[\cos(x), a]$ and by $\cos(x)$ when \textit{x} = 4 and  \textit{a} = 0.}
\begin{align*}
P^2(\cos(x = 4), 0) &\neq \cos(x = 4) \nonumber \\ 
1- \frac{1}{2}x^2 &\neq  \cos(4) \nonumber \\ 
1- \frac{1}{2}4^2 &\neq  -0.65 \nonumber \\ 
1- 16 &\neq  -0.65 \nonumber \\ 
0.5 &\neq  -0.65 \nonumber \\ 
\end{align*}
\vspace*{-25mm}

\noindent \hrulefill
\end{example}
Figure \ref{fig:taylor-vs-nonlin} provides a comprehensive visualization of the of the point conveyed in Example \ref{exm:taylor-estimates} about the accuracy of Taylor series approximations. In Figure \ref{fig:taylor-vs-nonlin}, the values returned by the nonlinear function of \(\cos(x)\) and its second-order Taylor series \(P^2[\cos(x)] = 1- \frac{1}{2}x^2\) are shown. The
second order Taylor series perfectly estimates \(\cos(x)\) when the point
of evaluation (\(x\)) equals the point of derivation (\(a\); \(x = a = 0\)),
but incurs an increasingly large amount of error as the difference
between the point of evaluation and the point of derivation increases.
For example, at \(x = 10\), \(\cos(10) = -0.84\), but the Taylor series
outputs a value of -49.50 (\(P^2[cos(50)] = 1- \frac{1}{2}10^2 = -49.50\)).
\begin{apaFigure}
[portrait]
[samepage]
[0cm]
{Estimation Accuracy of Taylor Series Approximation of Nonlinear Function (cos(x))}
{taylor-vs-nonlin}
{0.7}
{Figures/taylor_vs_nonlin}
{The second order Taylor series perfectly estimates $\cos(x)$ when the point of evaluation ($x$) equals the point of derivation ($a$; $x = a = 0$), but incurs an increasingly large amount of error as the difference between the point of evaluation and the point of derivation increases. For example, at $x = 10$, $\cos(x) = -0.84$, but the Taylor series outputs a value of -49.50 ($P^2[cos(50)] = 1- \frac{1}{2}10^2 = -49.50$).}
\end{apaFigure}
\subsubapp{Taylor Series of the Logistic Function}

Given that the Taylor series provides a linear version of a nonlinear function, the structured latent curve modelling approach uses Taylor series' to fit nonlinear functions into the linear nature of the structural equation modelling framework \autocite{browne1991,browne1993}. In the current simulations, the logistic function was used to generate data (see Equation \ref{eq:logistic}), and so a Taylor series approximation was constructed for the logistic function in the analysis. Note that, because the logistic function had four parameters (\(\uptheta\), \(\upalpha\), \(\upbeta\), \(\upgamma\)), partial derivatives were computed with respect to each of the parameters. Using a derivative order set to one
(\(n = 1\)), the following Taylor series was constructed for the logistic
function (Equation \ref{eq:logistic-approx}):
\begin{align}
 P^1(L(\Uptheta, t)) = L + \frac{\partial L}{\partial \uptheta}(x_{\uptheta}-a_{\uptheta})^1 + \frac{\partial L}{\partial \upalpha}(x_{\upalpha}-a_{\upalpha})^1 + \frac{\partial L}{\partial \upbeta}(x_{\upbeta}-a_{\upbeta})^1 + \frac{\partial L}{\partial \upgamma_{\upgamma}}(x_{\upgamma}-a_{\upgamma})^1, 
\label{eq:logistic-approx}
\end{align}
\noindent where \(\mathbf{L(\Uptheta, t)}\) represents the logistic function shown below in
Equation \ref{eq:logistic}:
\begin{align}
  \mathbf{L(\Uptheta, t)} = \uptheta + \frac{\upalpha - \uptheta}{{1 + e^\frac{\upbeta - t}{\upgamma}}} + \upepsilon, 
\label{eq:logistic}
\end{align}
\noindent with \(\mathbf{\Uptheta} = [\uptheta, \upalpha, \upbeta, \upgamma]\) and \(\mathbf{L(\Uptheta, t)}\) being a vector of scores across all \(\mathbf{t}\) time points. Because each parameter of the logistic function has a unique meaning (see section on \protect\hyperlink{data-generation}{data generation}), they are unlikely to have the same population value, and so the derivation (\(a\)) will, therefore, differ for each parameter. To set the derivation values (\(a\)), the mean values estimated by the structured latent growth curve model for each parameter (i.e., fixed-effect values) are used, meaning that each derivation value in Equation \ref{eq:logistic-approx} is replaced with a model estimate as shown below:
\begin{itemize}
\tightlist
\item
  \(a_{\uptheta} = \hat{\uptheta}\)
\item
  \(a_{\upalpha} = \hat{\upalpha}\)
\item
  \(a_{\upbeta} = \hat{\upbeta}\)
\item
  \(a_{\upgamma} = \hat{\upgamma}\)
\end{itemize}
\noindent where a caret (\(\hat{\phantom{\beta}}\)) denotes a parameter value that is estimated by the analysis. In order to compute curves for each \(p\) person, evaluation points for each parameter (\(x_{\uptheta}\), \(x_{\upalpha}\), \(x_{\upbeta}\), \(x_{\upgamma}\)) are set to the value computed for a given person (\(\uptheta_p\), \(\upalpha_p\), \(\upbeta_p\), \(\upgamma_p\)). Thus, each evaluation value in Equation \ref{eq:logistic-approx} is replaced with a person-specfic value as shown below:
\begin{itemize}
\tightlist
\item
  \(x_{\uptheta} = \uptheta_p\)
\item
  \(x_{\upalpha} = \upalpha_p\)
\item
  \(x_{\upbeta} = \upbeta_p\)
\item
  \(x_{\upgamma} = \upgamma_p\).
\end{itemize}
\noindent Substituting the above values for the derivation and evaluation values of \(x\) and \(a\) in the initial logistic Taylor series (Equation \ref{eq:logistic-approx}) yields the following function (Equation \ref{eq:taylor-full}):
\begin{align}
 P^1(L(\Uptheta, t)) = L(\Uptheta, t) + \frac{\partial L}{\partial \uptheta}(\uptheta_p-\hat{\uptheta})^1 + \frac{\partial L}{\partial \upalpha}(\upalpha_p-\hat{\upalpha})^1 + \frac{\partial L}{\partial \upbeta}(\upbeta_p-\hat{\upbeta})^1 + \frac{\partial L}{\partial \upgamma}(\upgamma_p-\hat{\upbeta})^1.
\label{eq:taylor-full}
\end{align}
Two important points about Equation \ref{eq:logistic-approx} deserve mentioning. First, the average population logistic curve (i.e., the fixed-effect parameter values) will have a perfect logistic function shape. In estimating the average population logistic curve, the evaluation values (\(x\)) are set equal to the derivation value counterparts (\(a\)); that is, each mean value estimated for a parameter (\(\hat{\uptheta}\), \(\hat{\upalpha}\), \(\hat{\upbeta}\), \(\hat{\upgamma}\)) replaces the corresponding derivation-evaluation pair in Equation \ref{eq:logistic-approx}. Second, it is possible that estimates of random-effect parameters (i.e., variability observed in a parameter's value across people) may be misleading. To compute the values for the random-effect parameters, the evaluation values (\(a\)) are set to the logistic function values needed to compute each \(p\) person's observed curve (\(\uptheta_p\), \(\upalpha_p\), \(\upbeta_p\), \(\upgamma_p\)). Because Taylor series approximations are only locally accurate, the curves computed for individuals can accommodate shapes that do not resemble a logistic (i.e., s-shaped) pattern (see Example \ref{exm:taylor-estimates}). Thus, estimates of random-effect parameters (i.e., variability observed in a parameter's value across people) can be influenced by curves that do not have a logistic shape and, therefore, may be misleading.

\subsubapp{Fitting the Logistic Taylor Series Into the Structual Equation Modelling Framework}

With the logistic Taylor series computed in Equation \ref{eq:taylor-full}, it can be fit into the structural equation modelling framework by transforming it from its scalar form (Equation \ref{eq:taylor-full}) into its matrix form (see Equation \ref{eq:taylor-final}). In transforming the scalar form of the logistic Taylor series into a matrix form, three steps will be completed, with each step transforming a component of the scalar form into a matrix representation. The paragraphs that follow detail each of these three steps.

First, the partial derivative information must be transformed into their matrix form. The matrix \(\mathbf{\Uplambda}\) shown below contains the partial derivative information presented in the scalar Taylor series function (see Equation \ref{eq:taylor-full}):\footnote{This is also known as a Jacobian matrix.}

\[ 
\mathbf{\Uplambda} = 
\begin{bmatrix}
\frac{\partial L(\Uptheta, t_1)}{\partial \uptheta} & \frac{\partial L(\Uptheta, t_1)}{\partial \upalpha}  &  \frac{\partial L(\Uptheta, t_1)}{\partial \upbeta} & \frac{\partial L(\Uptheta, t_1)}{\partial \upgamma}   \\ 
\frac{\partial L(\Uptheta, t_2)}{\partial \uptheta}  & \frac{\partial L(\Uptheta, t_2)}{\partial \upalpha} &  \frac{\partial L(\Uptheta, t_2)}{\partial \upbeta} & \frac{\partial L(\Uptheta, t_2)}{\partial \upgamma} & \\ 
\vdots & \vdots & \vdots & \vdots \\ 
\frac{\partial L(\Uptheta, t_n)}{\partial \uptheta} & \frac{\partial L(\Uptheta, t_n)}{\partial \upalpha}  & \frac{\partial L(\Uptheta, t_n)}{\partial \upbeta} & \frac{\partial L(\Uptheta, t_n)}{\partial \upgamma} \\
\end{bmatrix}.
\]

\noindent As in the structural equation modelling framework (see Equation \ref{eq:sem-framework}) where each column of \(\mathbf{\Uplambda}\) specifies a basis curve (i.e., loadings of a growth parameter onto all time points that specify the effect of the parameter over time), each column of \(\mathbf{\Uplambda}\) in the structured latent curve modelling approach similarly contains the loadings of a logistic function parameter across all the \(n\) time points, but the loading values are now determined by the partial derivative of the logistic function with respect to that parameter.

Second, the difference between the evaluation and derivation values (\(x - a\)) must be transformed into their matrix form. As a reminder, the difference between the evaluation and derivation values is needed so that person-specific curves can be computed. Thus, the difference between the evaluation and derivation values can be conceptualized as person-specific deviation. The vector \(\mathbf{\upiota_p}\) contains the person-specific deviations (e.g., \(\hat{\uptheta} - \uptheta_p\)) from each mean estimated parameter value as shown below:

\[ 
\mathbf{\upiota_p} = 
\begin{bmatrix}
\hat{\uptheta} - \uptheta_p   \\ 
\hat{\upalpha} - \upalpha_p   \\ 
\hat{\upbeta} - \upbeta_p \\ 
\hat{\upgamma_i} - \upgamma_p \\
\end{bmatrix},
\]

\noindent where a caret (\(\hat{\phantom{\beta}}\)) denotes the mean value estimated
for a given parameter and a subscript \(p\) indicates a parameter value
computed for a person.

With a matrix of logistic function loadings (\(\mathbf{\Uplambda}\)) and the vector of person-specific deviations (\(\mathbf{\upiota_p}\)), person-specific weights can be computed for each parameter. Specifically, person-specific weights can be computed by post-multiplying the matrix of loadings (\(\mathbf{\Uplambda}\)) by the vector of person-specific deviations (\(\mathbf{\upiota_p}\)), as shown below in Equation \ref{eq:person-weights}:
\begin{align}
 \text{Person-specific weights} = \mathbf{\Uplambda\upiota_p}.
 \label{eq:person-weights}
\end{align}
\noindent Importantly, to compute person-specific curves (\(\mathbf{y_p}\)), the average logistic curve must be added to Equation \ref{eq:person-weights}, as shown below in Equation \ref{eq:slcm-nonsem}:
\begin{align}
 \mathbf{y_p} = \mathbf{L(\Uptheta, t)} + \mathbf{\Uplambda\upiota_p} + \mathbf{\mathcal{E}_p}.
 \label{eq:slcm-nonsem}
\end{align}
\noindent Unfortunately, the logistic function (\(\mathbf{L(\Uptheta, t)}\)) in the above expression (Equation \ref{eq:slcm-nonsem}) is simply the original logistic function (see Equation \ref{eq:logistic}), and so Equation \ref{eq:slcm-nonsem} above is nonlinear. Because Equation \ref{eq:slcm-nonsem} is nonlinear, it cannot be inserted in the structural equation modelling framework, which requires a linear function (see Equation \ref{eq:sem-framework}). Thus, the logistic function term in Equation \ref{eq:slcm-nonsem} (\(\mathbf{L(\Uptheta, t)}\)) must be linearized so that the logistic Taylor series can be used in the structural equation modelling framework.

Third, and last, the logistic function component (\(\mathbf{L(\Uptheta, t)}\)) must be linearized. By taking advantage of some clever linear algebra, the logistic function component can be rewritten as the product of the partial derivative matrix (\(\mathbf{\Uplambda}\)) and a mean vector \autocites[\(\mathbf{\uptau}\);][]{shapiro1987,browne1993} as shown below in Equation \ref{eq:logistic-matrix-vector}:
\begin{align}
 \mathbf{L(\Uptheta, t)} = \mathbf{\Lambda\uptau}.
\label{eq:logistic-matrix-vector}
\end{align}
\noindent Importantly, the values of the mean vector \(\mathbf{\uptau}\) need to be determined so that a linear representation of the logistic function can be created. Example \ref{exm:tau-vector} below solves for the mean vector (\(\mathbf{\uptau}\)) and shows that the values obtained for the linear parameters (i.e., \(\uptheta\) and \(\upalpha\)) constitute the mean values estimated by the analysis (i.e., the fixed-effect values) and zeroes are obtained for the nonlinear parameters (i.e., \(\uptheta\) and \(\upalpha\)). Given that the vector \(\mathbf{\uptau}\) contains mean estimated values, it is often called the mean vector \autocite{blozis2004,preacher2015}.
\begin{example}
\protect\hypertarget{exm:tau-vector}{}\label{exm:tau-vector}Computation of mean vector \(\mathbf{\uptau}\).

\noindent \textup{Given the parameter estimates of $\hat{\uptheta} = 3.00$, $\hat{\upalpha} = 3.32$, $\hat{\upbeta} = 180.00$, and $\hat{\upgamma} = 20.00$ and $\mathbf{t}$ = [0, 1, 2, 3], $\mathbf{\uptau}$ = [3.00, 3.32, 0, 0], then }
\begin{align*}
\mathbf{L(\Uptheta, t)} &= \mathbf{\Lambda\uptau} \\ 
[3.00, 3.02, 3.30, 3.32] &= \begin{bmatrix}
1.00 & 0.00 & 0.00  & 0.00 \\ 
0.95  & 0.05 & -0.00 & 0.00 \\ 
0.05 & 0.95 & -0.00 & -0.00 \\ 
0.00 & 1.00  & 0.00 & 0.00 \\
\end{bmatrix} \mathbf{\uptau} \\ 
\begin{bmatrix}
1.00 & 0.00 & 0.00  & 0.00 \\ 
0.95  & 0.05 & -0.00 & 0.00 \\ 
0.05 & 0.95 & -0.00 & -0.00 \\ 
0.00 & 1.00  & 0.00 & 0.00 \\
\end{bmatrix}^{-1}
\begin{bmatrix} 
3.00 \\ 3.02 \\ 3.30 \\ 3.32
\end{bmatrix} &=  \mathbf{\Lambda\uptau} \\ 
 \mathbf{\uptau} &= [3.00, 3.32, 0, 0]\\
\end{align*}
\vspace*{-25mm}

\noindent \hrulefill
\end{example}
With \(\mathbf{L(\Uptheta, t)} = \mathbf{\Uplambda\uptau}\), Equation \ref{eq:slcm-nonsem} can be rewritten in a linear equation as shown below in Equation \ref{eq:taylor-linear}:
\begin{align}
 \mathbf{y_p} = \mathbf{\Uplambda\uptau} + \mathbf{\Uplambda\upiota_p} + \mathbf{\mathcal{E}_p}.
 \label{eq:taylor-linear}
 \end{align}
\noindent Two important points should be made about Equation \ref{eq:taylor-linear}. First, with some algebraic modification, it can be shown to have the exact same form as the general structural equation modelling framework (see Equation \ref{eq:sem-framework}) that expresses a person's score (\(y_p\)) as the sum of a loading matrix (\(\mathbf{\Uplambda}\)) post-multiplied by a vector of person-specific deviations (\(\upiota_p\)) and an error vector (\(\mathbf{\mathcal{E}_p}\)). To show the equivalence between Equation \ref{eq:taylor-linear} and Equation \ref{eq:sem-framework}, the mean vector \(\mathbf{\uptau}\) and vector of person-specific deviations \(\mathbf{\upiota_p}\) can be combined into a new vector \(\mathbf{s_p}\) that, like the product of \(\mathbf{\Lambda\uptau}\) (see Equation \ref{eq:logistic-matrix-vector}), also represents the person-specific weights applied to the basis curves in \(\mathbf{\Uplambda}\) such that

\[  
\mathbf{s_p} = \mathbf{\uptau + \upiota_p} =
\begin{bmatrix} 
\hat{\uptheta} + \hat{\uptheta} - \uptheta_p \\ 
\hat{\upalpha} + \hat{\upalpha} - \upalpha_p \\ 
0 + \hat{\upbeta} - \upbeta_p \\ 
0 + \hat{\upgamma} - \upgamma_p \\
\end{bmatrix},
\]

\noindent which allows Equation \ref{eq:taylor-linear} to be reexpressed in Equation \ref{eq:taylor-final} below and, thus, take on the exact same form as the general structural equation modelling framework (see Equation \ref{eq:sem-framework})
\begin{align}
\mathbf{y_p} = \mathbf{\Uplambda s_p} + \mathbf{\mathcal{E}_p}.
\label{eq:taylor-final}
\end{align}
Second, the logistic Taylor series shown in Equation \ref{eq:taylor-linear} reproduces the nonlinear logistic function. Because the expected value of the person-specific weights (\(\mathbf{s_p}\)) is the mean vector (\(\mathbf{\uptau}\);
\(\mathbb{E}[{\mathbf{s_p}}] = \mathbf{\uptau}\)), the expected set
of scores predicted across all people (\(\mathbb{E}[{\mathbf{y_p}}]\)) gives back the original expression for the logistic function matrix-vector product in Equation \ref{eq:logistic-matrix-vector} as shown below in Equation \ref{eq:expected-value}:
\begin{align}
 \mathbb{E}[{\mathbf{y_p}}] = \mathbf{\Uplambda\uptau} = \mathbf{L(\Uptheta, t)}. 
\label{eq:expected-value}
\end{align}
\noindent Therefore, the structured latent curve modelling approach
successfully reproduces the output of the nonlinear logistic function
(Equation \ref{eq:logistic}) with the linear function of Equation
\ref{eq:taylor-final}. Note that that no error term exists in Equation \ref{eq:expected-value} because the expected value of the error
values is zero (\(\mathbb{E}[{\mathbf{\mathcal{E}_p}}] = 0\)).

\subsubapp{Estimating Parameters in the Structured Latent Curve Modelling Approach}

To estimate the parameter values, the full-information maximum
likelihood shown in Equation \ref{eq:fiml-person} is computed for each
person (i.e., likelihood of observing a \(p\) person's data given the
estimated parameter values):
\begin{align}
\mathcal{L}_p = k_p \ln(2\pi) + \ln(|\mathbf{\Sigma_p}| + (\mathbf{y_p} - \mathbf{\upmu_p})^\top \mathbf{\Sigma_p}^{-1}(\mathbf{y_p} - \mathbf{\upmu_p}),
\label{eq:fiml-person}
\end{align}
\noindent where \(k_p\) is the number of non-missing values for a given
\(p\) person, \(\mathbf{\Sigma_p}\) is the model-implied covariance matrix
with rows and columns filtered at time points where person \(p\) has
missing data, \(\mathbf{y_p}\) is a vector containing the data points collected for a \(p\) person (i.e., filtered data), and
\(\mathbf{\upmu_p}\) is the model-implied mean vector that is filtered at
time points where person \(p\) has missing data. Note that, because all
my simulations assumed complete data across all times points, no filtering
procedures were executed \autocite[for a review of the filtering procedure, see][Chapter 5]{boker2020}. Thus, computing the above full-information
maximum likelihood in Equation \ref{eq:fiml-person} is equivalent to
computing the below likelihood function in Equation
\ref{eq:ml-estimation}:
\begin{align}
\mathcal{L}_p = k_p \ln(2\pi) + \ln(|\mathbf{\Sigma}| + (\mathbf{y_p} - \mathbf{\upmu})^\top \mathbf{\Sigma}^{-1}(\mathbf{y_p} - \mathbf{\upmu}),  
\label{eq:ml-estimation}
\end{align}
\noindent where \(\mathbf{\Sigma}\) is the model-implied covariance matrix,
\(\mathbf{y_p}\) contains the data collected from a \(p\) person, and
\(\mathbf{\upmu}\) is the model-implied mean vector. The model-implied
covariance matrix \(\mathbf{\Sigma}\) is computed using Equation
\ref{eq:covariance} below:
\begin{align}
\mathbf{\Sigma} = \mathbf{\Uplambda\Uppsi\Uplambda} + \mathbf{\Upomega}_{\mathcal{E}},   
\label{eq:covariance}
\end{align}
\noindent where \(\mathbf{\Uppsi}\) is the random-effect covariance matrix
and \(\mathbf{\Upomega}_{\mathcal{E}}\) contains the error variances at
each time point. The mean vector \(\mathbf{\upmu}\) is computed using
Equation \ref{eq:mean-structure} shown below:
\begin{align}
\mathbf{\upmu} = \mathbf{\Uplambda\uptau}. 
\label{eq:mean-structure}
\end{align}
\noindent Parameter estimation is conducted by finding values for the model-implied
covariance matrix \(\mathbf{\Sigma}\) and the model-implied mean vector
\(\mathbf{\upmu}\) that maximizes the sum of log-likelihoods across all \(P\) people
(see Equation \ref{eq:max-ll} below):
\begin{align}
\mathcal{L} = \underset{\mathbf{\Sigma},\mathbf{\upmu} }{\argmax} \sum^P_{p = 1} \mathcal{L}_p.
\label{eq:max-ll}
\end{align}
\noindent In OpenMx, the above problem is solved using the sequential
least squares quadratic program \autocite[for a review, see][]{kraft1994}.

\app{OpenMx Code for Structured Latent Growth Curve Model Used in Simulation Experiments}

\label{structured-lgc-code}

The code that I used to model logistic pattern of change (see \protect\hyperlink{data-generation}{data generation}) is shown in Code Block \ref{structured-model}. Note that, the code is largely excerpted from the \texttt{run\_exp\_simulations()} and \texttt{create\_logistic\_model\_ns()} functions from the \texttt{nonlinSims} package, and so readers interested in obtaining more information should consult the source code of this package. One important point to mention is that the model specified in Code Block \ref{structured-model} assumes time-structured data.

\captionof{chunk}{OpenMx Code for Structured Latent Growth Curve Model That Assumes Time-Structured Data}\restoreparindent\label{structured-model}
\begin{Shaded}
\begin{Highlighting}[numbers=left,,]
\CommentTok{\#Days on which measurements are assumed to be taken (note that model assumes time{-}structured data; that is, at each time point, participants provide data at the exact same moment). The measurement days obtained by finding the unique values in the \textasciigrave{}measurement\_day\textasciigrave{} column of the generated data set. }
\NormalTok{measurement\_days }\OtherTok{\textless{}{-}} \FunctionTok{unique}\NormalTok{(data}\SpecialCharTok{$}\NormalTok{measurement\_day) }

\CommentTok{\#Manifest variable names (i.e., names of columns containing data at each time point,}
\NormalTok{manifest\_vars }\OtherTok{\textless{}{-}}\NormalTok{ nonlinSims}\SpecialCharTok{:::}\FunctionTok{extract\_manifest\_var\_names}\NormalTok{(}\AttributeTok{data\_wide =}\NormalTok{ data\_wide)}

\CommentTok{\#Now convert data to wide format (needed for OpenMx)}
\NormalTok{data\_wide }\OtherTok{\textless{}{-}}\NormalTok{ data[ , }\FunctionTok{c}\NormalTok{(}\DecValTok{1}\SpecialCharTok{:}\DecValTok{3}\NormalTok{, }\DecValTok{5}\NormalTok{)] }\SpecialCharTok{\%\textgreater{}\%} 
    \FunctionTok{pivot\_wider}\NormalTok{(}\AttributeTok{names\_from =}\NormalTok{ measurement\_day, }\AttributeTok{values\_from =} \FunctionTok{c}\NormalTok{(obs\_score, actual\_measurement\_day))}
  
\CommentTok{\#Remove . from column names so that OpenMx does not run into error (this occurs because, with some spacing schedules, measurement days are not integer values.) }
\FunctionTok{names}\NormalTok{(data\_wide) }\OtherTok{\textless{}{-}} \FunctionTok{str\_replace}\NormalTok{(}\AttributeTok{string =} \FunctionTok{names}\NormalTok{(data\_wide), }\AttributeTok{pattern =} \StringTok{\textquotesingle{}}\SpecialCharTok{\textbackslash{}\textbackslash{}}\StringTok{.\textquotesingle{}}\NormalTok{, }\AttributeTok{replacement =} \StringTok{\textquotesingle{}\_\textquotesingle{}}\NormalTok{)}

\CommentTok{\#Latent variable names (theta = baseline, alpha = maximal elevation, beta = days{-}to{-}halfway elevation, gamma = triquarter{-}haflway elevation)}
\NormalTok{latent\_vars }\OtherTok{\textless{}{-}} \FunctionTok{c}\NormalTok{(}\StringTok{\textquotesingle{}theta\textquotesingle{}}\NormalTok{, }\StringTok{\textquotesingle{}alpha\textquotesingle{}}\NormalTok{, }\StringTok{\textquotesingle{}beta\textquotesingle{}}\NormalTok{, }\StringTok{\textquotesingle{}gamma\textquotesingle{}}\NormalTok{) }

\NormalTok{latent\_growth\_curve\_model }\OtherTok{\textless{}{-}} \FunctionTok{mxModel}\NormalTok{(}
  \AttributeTok{model =}\NormalTok{ model\_name,}
  \AttributeTok{type =} \StringTok{\textquotesingle{}RAM\textquotesingle{}}\NormalTok{, }\AttributeTok{independent =}\NormalTok{ T,}
  \FunctionTok{mxData}\NormalTok{(}\AttributeTok{observed =}\NormalTok{ data\_wide, }\AttributeTok{type =} \StringTok{\textquotesingle{}raw\textquotesingle{}}\NormalTok{),}
  
  \AttributeTok{manifestVars =}\NormalTok{ manifest\_vars,}
  \AttributeTok{latentVars =}\NormalTok{ latent\_vars,}
  
  \CommentTok{\#Residual variances; by using one label, they are assumed to all be equal (homogeneity of variance). That is, there is no complex error structure. }
  \FunctionTok{mxPath}\NormalTok{(}\AttributeTok{from =}\NormalTok{ manifest\_vars,}
         \AttributeTok{arrows=}\DecValTok{2}\NormalTok{, }\AttributeTok{free=}\ConstantTok{TRUE}\NormalTok{,  }\AttributeTok{labels=}\StringTok{\textquotesingle{}epsilon\textquotesingle{}}\NormalTok{, }\AttributeTok{values =} \DecValTok{1}\NormalTok{, }\AttributeTok{lbound =} \DecValTok{0}\NormalTok{),}
  
  \CommentTok{\#Latent variable covariances and variances (note that only the variances are estimated. )}
  \FunctionTok{mxPath}\NormalTok{(}\AttributeTok{from =}\NormalTok{ latent\_vars,}
         \AttributeTok{connect=}\StringTok{\textquotesingle{}unique.pairs\textquotesingle{}}\NormalTok{, }\AttributeTok{arrows=}\DecValTok{2}\NormalTok{,}
         \AttributeTok{free =} \FunctionTok{c}\NormalTok{(}\ConstantTok{TRUE}\NormalTok{,}\ConstantTok{FALSE}\NormalTok{, }\ConstantTok{FALSE}\NormalTok{, }\ConstantTok{FALSE}\NormalTok{, }
                  \ConstantTok{TRUE}\NormalTok{, }\ConstantTok{FALSE}\NormalTok{, }\ConstantTok{FALSE}\NormalTok{, }
                  \ConstantTok{TRUE}\NormalTok{, }\ConstantTok{FALSE}\NormalTok{, }
                  \ConstantTok{TRUE}\NormalTok{), }
         \AttributeTok{values=}\FunctionTok{c}\NormalTok{(}\DecValTok{1}\NormalTok{, }\ConstantTok{NA}\NormalTok{, }\ConstantTok{NA}\NormalTok{, }\ConstantTok{NA}\NormalTok{, }
                  \DecValTok{1}\NormalTok{, }\ConstantTok{NA}\NormalTok{, }\ConstantTok{NA}\NormalTok{, }
                  \DecValTok{1}\NormalTok{, }\ConstantTok{NA}\NormalTok{,}
                  \DecValTok{1}\NormalTok{),}
         \AttributeTok{labels=}\FunctionTok{c}\NormalTok{(}\StringTok{\textquotesingle{}theta\_rand\textquotesingle{}}\NormalTok{, }\StringTok{\textquotesingle{}NA(cov\_theta\_alpha)\textquotesingle{}}\NormalTok{, }\StringTok{\textquotesingle{}NA(cov\_theta\_beta)\textquotesingle{}}\NormalTok{, }
                  \StringTok{\textquotesingle{}NA(cov\_theta\_gamma)\textquotesingle{}}\NormalTok{,}
                  \StringTok{\textquotesingle{}alpha\_rand\textquotesingle{}}\NormalTok{,}\StringTok{\textquotesingle{}NA(cov\_alpha\_beta)\textquotesingle{}}\NormalTok{, }\StringTok{\textquotesingle{}NA(cov\_alpha\_gamma)\textquotesingle{}}\NormalTok{, }
                  \StringTok{\textquotesingle{}beta\_rand\textquotesingle{}}\NormalTok{, }\StringTok{\textquotesingle{}NA(cov\_beta\_gamma)\textquotesingle{}}\NormalTok{, }
                  \StringTok{\textquotesingle{}gamma\_rand\textquotesingle{}}\NormalTok{), }
         \AttributeTok{lbound =} \FunctionTok{c}\NormalTok{(}\FloatTok{1e{-}3}\NormalTok{, }\ConstantTok{NA}\NormalTok{, }\ConstantTok{NA}\NormalTok{, }\ConstantTok{NA}\NormalTok{, }
                    \FloatTok{1e{-}3}\NormalTok{, }\ConstantTok{NA}\NormalTok{, }\ConstantTok{NA}\NormalTok{, }
                    \DecValTok{1}\NormalTok{, }\ConstantTok{NA}\NormalTok{,}
                    \DecValTok{1}\NormalTok{), }
         \AttributeTok{ubound =} \FunctionTok{c}\NormalTok{(}\DecValTok{2}\NormalTok{, }\ConstantTok{NA}\NormalTok{, }\ConstantTok{NA}\NormalTok{, }\ConstantTok{NA}\NormalTok{, }
                    \DecValTok{2}\NormalTok{, }\ConstantTok{NA}\NormalTok{, }\ConstantTok{NA}\NormalTok{, }
                    \DecValTok{90}\SpecialCharTok{\^{}}\DecValTok{2}\NormalTok{, }\ConstantTok{NA}\NormalTok{, }
                    \DecValTok{45}\SpecialCharTok{\^{}}\DecValTok{2}\NormalTok{)),}
  
  \CommentTok{\# Latent variable means (linear parameters). Note that the parameters of beta and gamma do not have estimated means because they are nonlinear parameters (i.e., the logistic function\textquotesingle{}s first{-}order partial derivative with respect to each of those two parameters contains those two parameters. )}
  \FunctionTok{mxPath}\NormalTok{(}\AttributeTok{from =} \StringTok{\textquotesingle{}one\textquotesingle{}}\NormalTok{, }\AttributeTok{to =} \FunctionTok{c}\NormalTok{(}\StringTok{\textquotesingle{}theta\textquotesingle{}}\NormalTok{, }\StringTok{\textquotesingle{}alpha\textquotesingle{}}\NormalTok{), }\AttributeTok{free =} \FunctionTok{c}\NormalTok{(}\ConstantTok{TRUE}\NormalTok{, }\ConstantTok{TRUE}\NormalTok{), }\AttributeTok{arrows =} \DecValTok{1}\NormalTok{,}
         \AttributeTok{labels =} \FunctionTok{c}\NormalTok{(}\StringTok{\textquotesingle{}theta\_fixed\textquotesingle{}}\NormalTok{, }\StringTok{\textquotesingle{}alpha\_fixed\textquotesingle{}}\NormalTok{), }\AttributeTok{lbound =} \DecValTok{0}\NormalTok{, }\AttributeTok{ubound =} \DecValTok{7}\NormalTok{, }
         \AttributeTok{values =} \FunctionTok{c}\NormalTok{(}\DecValTok{1}\NormalTok{, }\DecValTok{1}\NormalTok{)),}
  
  \CommentTok{\#Functional constraints (needed to estimate mean values of fixed{-}effect parameters)}
  \FunctionTok{mxMatrix}\NormalTok{(}\AttributeTok{type =} \StringTok{\textquotesingle{}Full\textquotesingle{}}\NormalTok{, }\AttributeTok{nrow =} \FunctionTok{length}\NormalTok{(manifest\_vars), }\AttributeTok{ncol =} \DecValTok{1}\NormalTok{, }\AttributeTok{free =} \ConstantTok{TRUE}\NormalTok{, }
           \AttributeTok{labels =} \StringTok{\textquotesingle{}theta\_fixed\textquotesingle{}}\NormalTok{, }\AttributeTok{name =} \StringTok{\textquotesingle{}t\textquotesingle{}}\NormalTok{, }\AttributeTok{values =} \DecValTok{1}\NormalTok{, }\AttributeTok{lbound =} \DecValTok{0}\NormalTok{,  }\AttributeTok{ubound =} \DecValTok{7}\NormalTok{), }
  \FunctionTok{mxMatrix}\NormalTok{(}\AttributeTok{type =} \StringTok{\textquotesingle{}Full\textquotesingle{}}\NormalTok{, }\AttributeTok{nrow =} \FunctionTok{length}\NormalTok{(manifest\_vars), }\AttributeTok{ncol =} \DecValTok{1}\NormalTok{, }\AttributeTok{free =} \ConstantTok{TRUE}\NormalTok{, }
           \AttributeTok{labels =} \StringTok{\textquotesingle{}alpha\_fixed\textquotesingle{}}\NormalTok{, }\AttributeTok{name =} \StringTok{\textquotesingle{}a\textquotesingle{}}\NormalTok{, }\AttributeTok{values =} \DecValTok{1}\NormalTok{, }\AttributeTok{lbound =} \DecValTok{0}\NormalTok{,  }\AttributeTok{ubound =} \DecValTok{7}\NormalTok{), }
  \FunctionTok{mxMatrix}\NormalTok{(}\AttributeTok{type =} \StringTok{\textquotesingle{}Full\textquotesingle{}}\NormalTok{, }\AttributeTok{nrow =} \FunctionTok{length}\NormalTok{(manifest\_vars), }\AttributeTok{ncol =} \DecValTok{1}\NormalTok{, }\AttributeTok{free =} \ConstantTok{TRUE}\NormalTok{, }
           \AttributeTok{labels =} \StringTok{\textquotesingle{}beta\_fixed\textquotesingle{}}\NormalTok{, }\AttributeTok{name =} \StringTok{\textquotesingle{}b\textquotesingle{}}\NormalTok{, }\AttributeTok{values =} \DecValTok{1}\NormalTok{, }\AttributeTok{lbound =} \DecValTok{1}\NormalTok{, }\AttributeTok{ubound =} \DecValTok{360}\NormalTok{),}
  \FunctionTok{mxMatrix}\NormalTok{(}\AttributeTok{type =} \StringTok{\textquotesingle{}Full\textquotesingle{}}\NormalTok{, }\AttributeTok{nrow =} \FunctionTok{length}\NormalTok{(manifest\_vars), }\AttributeTok{ncol =} \DecValTok{1}\NormalTok{, }\AttributeTok{free =} \ConstantTok{TRUE}\NormalTok{, }
           \AttributeTok{labels =} \StringTok{\textquotesingle{}gamma\_fixed\textquotesingle{}}\NormalTok{, }\AttributeTok{name =} \StringTok{\textquotesingle{}g\textquotesingle{}}\NormalTok{, }\AttributeTok{values =} \DecValTok{1}\NormalTok{, }\AttributeTok{lbound =} \DecValTok{1}\NormalTok{, }\AttributeTok{ubound =} \DecValTok{360}\NormalTok{), }

  \FunctionTok{mxMatrix}\NormalTok{(}\AttributeTok{type =} \StringTok{\textquotesingle{}Full\textquotesingle{}}\NormalTok{, }\AttributeTok{nrow =} \FunctionTok{length}\NormalTok{(manifest\_vars), }\AttributeTok{ncol =} \DecValTok{1}\NormalTok{, }\AttributeTok{free =} \ConstantTok{FALSE}\NormalTok{, }
           \AttributeTok{values =}\NormalTok{ measurement\_days, }\AttributeTok{name =} \StringTok{\textquotesingle{}time\textquotesingle{}}\NormalTok{),}
  
  \CommentTok{\#Algebra specifying first{-}order partial derivatives; }
  \FunctionTok{mxAlgebra}\NormalTok{(}\AttributeTok{expression =} \DecValTok{1} \SpecialCharTok{{-}} \DecValTok{1}\SpecialCharTok{/}\NormalTok{(}\DecValTok{1} \SpecialCharTok{+} \FunctionTok{exp}\NormalTok{((b }\SpecialCharTok{{-}}\NormalTok{ time)}\SpecialCharTok{/}\NormalTok{g)), }\AttributeTok{name=}\StringTok{"Tl"}\NormalTok{),}
  \FunctionTok{mxAlgebra}\NormalTok{(}\AttributeTok{expression =} \DecValTok{1}\SpecialCharTok{/}\NormalTok{(}\DecValTok{1} \SpecialCharTok{+} \FunctionTok{exp}\NormalTok{((b }\SpecialCharTok{{-}}\NormalTok{ time)}\SpecialCharTok{/}\NormalTok{g)), }\AttributeTok{name =} \StringTok{\textquotesingle{}Al\textquotesingle{}}\NormalTok{), }
  
  \FunctionTok{mxAlgebra}\NormalTok{(}\AttributeTok{expression =} \SpecialCharTok{{-}}\NormalTok{((a }\SpecialCharTok{{-}}\NormalTok{ t) }\SpecialCharTok{*}\NormalTok{ (}\FunctionTok{exp}\NormalTok{((b }\SpecialCharTok{{-}}\NormalTok{ time)}\SpecialCharTok{/}\NormalTok{g) }\SpecialCharTok{*}\NormalTok{ (}\DecValTok{1}\SpecialCharTok{/}\NormalTok{g))}\SpecialCharTok{/}\NormalTok{(}\DecValTok{1} \SpecialCharTok{+} \FunctionTok{exp}\NormalTok{((b }\SpecialCharTok{{-}}\NormalTok{ time)}\SpecialCharTok{/}\NormalTok{g))}\SpecialCharTok{\^{}}\DecValTok{2}\NormalTok{), }\AttributeTok{name =} \StringTok{\textquotesingle{}Bl\textquotesingle{}}\NormalTok{),}
  \FunctionTok{mxAlgebra}\NormalTok{(}\AttributeTok{expression =}\NormalTok{  (a }\SpecialCharTok{{-}}\NormalTok{ t) }\SpecialCharTok{*}\NormalTok{ (}\FunctionTok{exp}\NormalTok{((b }\SpecialCharTok{{-}}\NormalTok{ time)}\SpecialCharTok{/}\NormalTok{g) }\SpecialCharTok{*}\NormalTok{ ((b }\SpecialCharTok{{-}}\NormalTok{ time)}\SpecialCharTok{/}\NormalTok{g}\SpecialCharTok{\^{}}\DecValTok{2}\NormalTok{))}\SpecialCharTok{/}\NormalTok{(}\DecValTok{1} \SpecialCharTok{+} \FunctionTok{exp}\NormalTok{((b }\SpecialCharTok{{-}}\NormalTok{time)}\SpecialCharTok{/}\NormalTok{g))}\SpecialCharTok{\^{}}\DecValTok{2}\NormalTok{, }\AttributeTok{name =} \StringTok{\textquotesingle{}Gl\textquotesingle{}}\NormalTok{),}
  
  \CommentTok{\#Factor loadings; all fixed and, importantly, constrained to change according to their partial derivatives (i.e., nonlinear functions) }
  \FunctionTok{mxPath}\NormalTok{(}\AttributeTok{from =} \StringTok{\textquotesingle{}theta\textquotesingle{}}\NormalTok{, }\AttributeTok{to =}\NormalTok{ manifest\_vars, }\AttributeTok{arrows=}\DecValTok{1}\NormalTok{, }\AttributeTok{free=}\ConstantTok{FALSE}\NormalTok{,  }
         \AttributeTok{labels =} \FunctionTok{sprintf}\NormalTok{(}\AttributeTok{fmt =} \StringTok{\textquotesingle{}Tl[\%d,1]\textquotesingle{}}\NormalTok{, }\DecValTok{1}\SpecialCharTok{:}\FunctionTok{length}\NormalTok{(manifest\_vars))),}
  \FunctionTok{mxPath}\NormalTok{(}\AttributeTok{from =} \StringTok{\textquotesingle{}alpha\textquotesingle{}}\NormalTok{, }\AttributeTok{to =}\NormalTok{ manifest\_vars, }\AttributeTok{arrows=}\DecValTok{1}\NormalTok{, }\AttributeTok{free=}\ConstantTok{FALSE}\NormalTok{,  }
         \AttributeTok{labels =} \FunctionTok{sprintf}\NormalTok{(}\AttributeTok{fmt =} \StringTok{\textquotesingle{}Al[\%d,1]\textquotesingle{}}\NormalTok{, }\DecValTok{1}\SpecialCharTok{:}\FunctionTok{length}\NormalTok{(manifest\_vars))), }
  \FunctionTok{mxPath}\NormalTok{(}\AttributeTok{from=}\StringTok{\textquotesingle{}beta\textquotesingle{}}\NormalTok{, }\AttributeTok{to =}\NormalTok{ manifest\_vars, }\AttributeTok{arrows=}\DecValTok{1}\NormalTok{,  }\AttributeTok{free=}\ConstantTok{FALSE}\NormalTok{,}
         \AttributeTok{labels =}  \FunctionTok{sprintf}\NormalTok{(}\AttributeTok{fmt =} \StringTok{\textquotesingle{}Bl[\%d,1]\textquotesingle{}}\NormalTok{, }\DecValTok{1}\SpecialCharTok{:}\FunctionTok{length}\NormalTok{(manifest\_vars))), }
  \FunctionTok{mxPath}\NormalTok{(}\AttributeTok{from=}\StringTok{\textquotesingle{}gamma\textquotesingle{}}\NormalTok{, }\AttributeTok{to =}\NormalTok{ manifest\_vars, }\AttributeTok{arrows=}\DecValTok{1}\NormalTok{,  }\AttributeTok{free=}\ConstantTok{FALSE}\NormalTok{,}
         \AttributeTok{labels =}  \FunctionTok{sprintf}\NormalTok{(}\AttributeTok{fmt =} \StringTok{\textquotesingle{}Gl[\%d,1]\textquotesingle{}}\NormalTok{, }\DecValTok{1}\SpecialCharTok{:}\FunctionTok{length}\NormalTok{(manifest\_vars))), }
  
  \CommentTok{\#Fit function used to estimate free parameter values. }
  \FunctionTok{mxFitFunctionML}\NormalTok{(}\AttributeTok{vector =} \ConstantTok{FALSE}\NormalTok{)}
\NormalTok{)}

\CommentTok{\#Use starting value function from OpenMx to generate good starting values (uses weighted least squares)}
\NormalTok{latent\_growth\_model }\OtherTok{\textless{}{-}} \FunctionTok{mxAutoStart}\NormalTok{(}\AttributeTok{model =}\NormalTok{ latent\_growth\_model)}

\CommentTok{\#Fit model using mxTryHard(). Increases probability of convergence by attempting model convergence by randomly shifting starting values. }
\NormalTok{model\_results }\OtherTok{\textless{}{-}} \FunctionTok{mxTryHard}\NormalTok{(latent\_growth\_model)}
\end{Highlighting}
\end{Shaded}
\app{Complete Versions of Bias/Precision Plots (Day- and Likert-Unit Parameters)}

\label{complete-versions}
\secapp{Experiment 1}
\subapp{Equal Spacing}
\begin{apaFigure}
[portrait]
[samepage]
[0cm]
{Bias/Precision Plots for Day- and Likert-Unit Parameters With Equal Spacing in Experiment 1}
{exp1_plot_equal_app}
{0.16}
{Figures/exp1_plot_days_equal spacing}
{}
\end{apaFigure}
\addtocounter{figure}{-1}
\begin{apaFigure}
[portrait]
[samepage]
[0cm]
{Bias/Precision Plots for Day- and Likert-Unit Parameters With Equal Spacing in Experiment 1 (continued)}
{exp1_plot_equal_app}
{0.16}
{Figures/exp1_plot_likert_equal spacing}
{Panels A--B:  Bias/precision plots for the fixed- and random-effect days-to-halfway elevation parameters, respectively ($\upbeta_{fixed}$ and $\upbeta_{random}$). Panels C--D: Bias/precision plots for the fixed- and random-effect triquarter-halfway elevation parameters, respectively ($\upgamma_{fixed}$ and $\upgamma_{random}$). Panels E--F: Bias/precision plots for the fixed- and random-effect baseline parameters, respectively ($\uptheta_{fixed}$ and $\uptheta_{random}$). Panels G--H: Bias/precision plots for the fixed- and random-effect maximal elevation parameters, respectively ($\upalpha_{fixed}$ and $\upalpha_{random}$). Blue horizontal lines in each panel represent the population value for each parameter. Population values for each day-unit parameter are as follows: $\upbeta_{fixed} \in$ {80, 180, 280}, $\upbeta_{random}$ = 10.00, $\upgamma_{fixed}$ = 20.00, $\upgamma_{random}$ = 4.00, $\uptheta_{fixed}$ = 3.00, $\uptheta_{random}$ = 0.05, $\upalpha_{fixed}$ = 3.32, $\upalpha_{random}$ = 0.05, $\upepsilon$ = 0.05. Gray bands indicate the $\pm 10\%$ margin of error for each parameter and unfilled dots indicate cells with average parameter estimates outside of the margin or biased estimates. Error bars represent the middle 95\% of estimated values, with light blue error bars indicating imprecise estimation. I considered dots that fell outside the gray bands as biased and error bar lengths with at least one whisker length exceeding the 10\% cutoff (i.e., or longer than the portion of the gray band underlying the whisker) as imprecise. Note that random-effect parameter units are in standard deviation units. Importantly, across all nature-of-change values (i.e., population values used for $\upbeta_{fixed}$), the acceptable amount of bias and precision was based on a population value of 180. See Table \ref{tab:param-exp-1} for specific values estimated for each parameter.}
[notrack]
\end{apaFigure}
\subapp{Time-Interval Increasing Spacing}
\begin{apaFigure}
[portrait]
[samepage]
[0cm]
{Bias/Precision Plots for Day- and Likert-Unit Parameters With Time-Interval Increasing Spacing in Experiment 1}
{exp1_plot_equal_app}
{0.16}
{Figures/exp1_plot_days_time-interval increasing}
{}
\end{apaFigure}
\addtocounter{figure}{-1}
\begin{apaFigure}
[portrait]
[samepage]
[0cm]
{Bias/Precision Plots for Day- and Likert-Unit Parameters With Time-Interval Increasing Spacing in Experiment 1 (continued)}
{exp1_plot_equal_app}
{0.16}
{Figures/exp1_plot_likert_time-interval increasing}
{Panels A--B:  Bias/precision plots for the fixed- and random-effect days-to-halfway elevation parameters, respectively ($\upbeta_{fixed}$ and $\upbeta_{random}$). Panels C--D: Bias/precision plots for the fixed- and random-effect triquarter-halfway elevation parameters, respectively ($\upgamma_{fixed}$ and $\upgamma_{random}$). Panels E--F: Bias/precision plots for the fixed- and random-effect baseline parameters, respectively ($\uptheta_{fixed}$ and $\uptheta_{random}$). Panels G--H: Bias/precision plots for the fixed- and random-effect maximal elevation parameters, respectively ($\upalpha_{fixed}$ and $\upalpha_{random}$). Blue horizontal lines in each panel represent the population value for each parameter. Population values for each day-unit parameter are as follows: $\upbeta_{fixed} \in$ {80, 180, 280}, $\upbeta_{random}$ = 10.00, $\upgamma_{fixed}$ = 20.00, $\upgamma_{random}$ = 4.00, $\uptheta_{fixed}$ = 3.00, $\uptheta_{random}$ = 0.05, $\upalpha_{fixed}$ = 3.32, $\upalpha_{random}$ = 0.05, $\upepsilon$ = 0.05. Gray bands indicate the $\pm 10\%$ margin of error for each parameter and unfilled dots indicate cells with average parameter estimates outside of the margin or biased estimates. Error bars represent the middle 95\% of estimated values, with light blue error bars indicating imprecise estimation. I considered dots that fell outside the gray bands as biased and error bar lengths with at least one whisker length exceeding the 10\% cutoff (i.e., or longer than the portion of the gray band underlying the whisker) as imprecise. Note that random-effect parameter units are in standard deviation units. Importantly, across all nature-of-change values (i.e., population values used for $\upbeta_{fixed}$), the acceptable amount of bias and precision was based on a population value of 180. See Table \ref{tab:param-exp-1} for specific values estimated for each parameter.}
[notrack]
\end{apaFigure}
\subapp{Time-Interval Decreasing Spacing}
\begin{apaFigure}
[portrait]
[samepage]
[0cm]
{Bias/Precision Plots for Day- and Likert-Unit Parameters With Time-Interval Decreasing Spacing in Experiment 1}
{exp1_plot_equal_app}
{0.16}
{Figures/exp1_plot_days_time-interval decreasing}
{}
\end{apaFigure}
\addtocounter{figure}{-1}
\begin{apaFigure}
[portrait]
[samepage]
[0cm]
{Bias/Precision Plots for Day- and Likert-Unit Parameters With Time-Interval Decreasing Spacing in Experiment 1 (continued)}
{exp1_plot_equal_app}
{0.16}
{Figures/exp1_plot_likert_time-interval decreasing}
{Panels A--B:  Bias/precision plots for the fixed- and random-effect days-to-halfway elevation parameters, respectively ($\upbeta_{fixed}$ and $\upbeta_{random}$). Panels C--D: Bias/precision plots for the fixed- and random-effect triquarter-halfway elevation parameters, respectively ($\upgamma_{fixed}$ and $\upgamma_{random}$). Panels E--F: Bias/precision plots for the fixed- and random-effect baseline parameters, respectively ($\uptheta_{fixed}$ and $\uptheta_{random}$). Panels G--H: Bias/precision plots for the fixed- and random-effect maximal elevation parameters, respectively ($\upalpha_{fixed}$ and $\upalpha_{random}$). Blue horizontal lines in each panel represent the population value for each parameter. Population values for each day-unit parameter are as follows: $\upbeta_{fixed} \in$ {80, 180, 280}, $\upbeta_{random}$ = 10.00, $\upgamma_{fixed}$ = 20.00, $\upgamma_{random}$ = 4.00, $\uptheta_{fixed}$ = 3.00, $\uptheta_{random}$ = 0.05, $\upalpha_{fixed}$ = 3.32, $\upalpha_{random}$ = 0.05, $\upepsilon$ = 0.05. Gray bands indicate the $\pm 10\%$ margin of error for each parameter and unfilled dots indicate cells with average parameter estimates outside of the margin or biased estimates. Error bars represent the middle 95\% of estimated values, with light blue error bars indicating imprecise estimation. I considered dots that fell outside the gray bands as biased and error bar lengths with at least one whisker length exceeding the 10\% cutoff (i.e., or longer than the portion of the gray band underlying the whisker) as imprecise. Note that random-effect parameter units are in standard deviation units. Importantly, across all nature-of-change values (i.e., population values used for $\upbeta_{fixed}$), the acceptable amount of bias and precision was based on a population value of 180. See Table \ref{tab:param-exp-1} for specific values estimated for each parameter.}
[notrack]
\end{apaFigure}
\subapp{Middle-and-Extreme Spacing}
\begin{apaFigure}
[portrait]
[samepage]
[0cm]
{Bias/Precision Plots for Day- and Likert-Unit Parameters With Middle-and-Extreme Spacing in Experiment 1}
{exp1_plot_equal_app}
{0.16}
{Figures/exp1_plot_days_middle-and-extreme spacing}
{}
\end{apaFigure}
\addtocounter{figure}{-1}
\begin{apaFigure}
[portrait]
[samepage]
[0cm]
{Bias/Precision Plots for Day- and Likert-Unit Parameters With Middle-and-Extreme Spacing in Experiment 1 (continued)}
{exp1_plot_equal_app}
{0.16}
{Figures/exp1_plot_likert_middle-and-extreme spacing}
{Panels A--B:  Bias/precision plots for the fixed- and random-effect days-to-halfway elevation parameters, respectively ($\upbeta_{fixed}$ and $\upbeta_{random}$). Panels C--D: Bias/precision plots for the fixed- and random-effect triquarter-halfway elevation parameters, respectively ($\upgamma_{fixed}$ and $\upgamma_{random}$). Panels E--F: Bias/precision plots for the fixed- and random-effect baseline parameters, respectively ($\uptheta_{fixed}$ and $\uptheta_{random}$). Panels G--H: Bias/precision plots for the fixed- and random-effect maximal elevation parameters, respectively ($\upalpha_{fixed}$ and $\upalpha_{random}$). Blue horizontal lines in each panel represent the population value for each parameter. Population values for each day-unit parameter are as follows: $\upbeta_{fixed} \in$ {80, 180, 280}, $\upbeta_{random}$ = 10.00, $\upgamma_{fixed}$ = 20.00, $\upgamma_{random}$ = 4.00, $\uptheta_{fixed}$ = 3.00, $\uptheta_{random}$ = 0.05, $\upalpha_{fixed}$ = 3.32, $\upalpha_{random}$ = 0.05, $\upepsilon$ = 0.05. Gray bands indicate the $\pm 10\%$ margin of error for each parameter and unfilled dots indicate cells with average parameter estimates outside of the margin or biased estimates. Error bars represent the middle 95\% of estimated values, with light blue error bars indicating imprecise estimation. I considered dots that fell outside the gray bands as biased and error bar lengths with at least one whisker length exceeding the 10\% cutoff (i.e., or longer than the portion of the gray band underlying the whisker) as imprecise. Note that random-effect parameter units are in standard deviation units. Importantly, across all nature-of-change values (i.e., population values used for $\upbeta_{fixed}$), the acceptable amount of bias and precision was based on a population value of 180. See Table \ref{tab:param-exp-1} for specific values estimated for each parameter.}
[notrack]
\end{apaFigure}
\secapp{Experiment 2}
\subapp{Equal Spacing}
\begin{apaFigure}
[portrait]
[samepage]
[0cm]
{Bias/Precision Plots for Day- and Likert-Unit Parameters With Equal Spacing in Experiment 2}
{exp2_plot_equal_app}
{0.16}
{Figures/exp2_plot_days_equal spacing}
{}
\end{apaFigure}
\addtocounter{figure}{-1}
\begin{apaFigure}
[portrait]
[samepage]
[0cm]
{Bias/Precision Plots for Day- and Likert-Unit Parameters With Equal Spacing in Experiment 2 (continued)}
{exp2_plot_equal_app}
{0.16}
{Figures/exp2_plot_likert_equal spacing}
{Panels A--B: Bias/precision plots for the fixed- and random-effect days-to-halfway elevation parameters, respectively ($\upbeta_{fixed}$ and $\upbeta_{random}$). Panels C--D: Bias/precision plots for the fixed- and random-effect triquarter-halfway elevation parameters, respectively ($\upgamma_{fixed}$ and $\upgamma_{random}$). Panels E--F: Bias/precision plots for the fixed- and random-effect baseline parameters, respectively ($\uptheta_{fixed}$ and $\uptheta_{random}$). Panels G--H: Bias/precision plots for the fixed- and random-effect maximal elevation parameters, respectively ($\upalpha_{fixed}$ and $\upalpha_{random}$). Blue horizontal lines in each panel represent the population value for each parameter. Population values for each day-unit parameter are as follows: $\upbeta_{fixed} \in$ {80, 180, 280}, $\upbeta_{random}$ = 10.00, $\upgamma_{fixed}$ = 20.00, $\upgamma_{random}$ = 4.00, $\uptheta_{fixed}$ = 3.00, $\uptheta_{random}$ = 0.05, $\upalpha_{fixed}$ = 3.32, $\upalpha_{random}$ = 0.05, $\upepsilon$ = 0.05. Gray bands indicate the $\pm 10\%$ margin of error for each parameter and unfilled dots indicate cells with average parameter estimates outside of the margin or biased estimates. Error bars represent the middle 95\% of estimated values, with light blue error bars indicating imprecise estimation. I considered dots that fell outside the gray bands as biased and error bar lengths with at least one whisker length exceeding the 10\% cutoff (i.e., or longer than the portion of the gray band underlying the whisker) as imprecise. Note that random-effect parameter units are in standard deviation units. See Table \ref{tab:param-exp-2} for specific values estimated for each parameter.}
[notrack]
\end{apaFigure}
\subapp{Time-Interval Increasing Spacing}
\begin{apaFigure}
[portrait]
[samepage]
[0cm]
{Bias/Precision Plots for Day- and Likert-Unit Parameters With Time-Interval Increasing Spacing in Experiment 2}
{exp2_plot_time_inc_app}
{0.16}
{Figures/exp2_plot_days_time-interval increasing}
{}
\end{apaFigure}
\addtocounter{figure}{-1}
\begin{apaFigure}
[portrait]
[samepage]
[0cm]
{Bias/Precision Plots for Day- and Likert-Unit Parameters With Time-Interval Increasing Spacing in Experiment 2 (continued)}
{exp2_plot_time_inc_app}
{0.16}
{Figures/exp2_plot_likert_time-interval increasing}
{Panels A--B:  Bias/precision plots for the fixed- and random-effect days-to-halfway elevation parameters, respectively ($\upbeta_{fixed}$ and $\upbeta_{random}$). Panels C--D: Bias/precision plots for the fixed- and random-effect triquarter-halfway elevation parameters, respectively ($\upgamma_{fixed}$ and $\upgamma_{random}$). Panels E--F: Bias/precision plots for the fixed- and random-effect baseline parameters, respectively ($\uptheta_{fixed}$ and $\uptheta_{random}$). Panels G--H: Bias/precision plots for the fixed- and random-effect maximal elevation parameters, respectively ($\upalpha_{fixed}$ and $\upalpha_{random}$). Blue horizontal lines in each panel represent the population value for each parameter. Population values for each day-unit parameter are as follows: $\upbeta_{fixed} \in$ {80, 180, 280}, $\upbeta_{random}$ = 10.00, $\upgamma_{fixed}$ = 20.00, $\upgamma_{random}$ = 4.00, $\uptheta_{fixed}$ = 3.00, $\uptheta_{random}$ = 0.05, $\upalpha_{fixed}$ = 3.32, $\upalpha_{random}$ = 0.05, $\upepsilon$ = 0.05. Gray bands indicate the $\pm 10\%$ margin of error for each parameter and unfilled dots indicate cells with average parameter estimates outside of the margin or biased estimates. Error bars represent the middle 95\% of estimated values, with light blue error bars indicating imprecise estimation. I considered dots that fell outside the gray bands as biased and error bar lengths with at least one whisker length exceeding the 10\% cutoff (i.e., or longer than the portion of the gray band underlying the whisker) as imprecise. Note that random-effect parameter units are in standard deviation units. See Table \ref{tab:param-exp-2} for specific values estimated for each parameter.}
[notrack]
\end{apaFigure}
\subapp{Time-Interval Decreasing Spacing}
\begin{apaFigure}
[portrait]
[samepage]
[0cm]
{Bias/Precision Plots for Day- and Likert-Unit Parameters With Time-Interval Decreasing Spacing in Experiment 2}
{exp2_plot_time_dec_app}
{0.16}
{Figures/exp2_plot_days_time-interval decreasing}
{}
\end{apaFigure}
\addtocounter{figure}{-1}
\begin{apaFigure}
[portrait]
[samepage]
[0cm]
{Bias/Precision Plots for Day- and Likert-Unit Parameters With Time-Interval Decreasing Spacing in Experiment 2 (continued)}
{exp2_plot_time_dec_app}
{0.16}
{Figures/exp1_plot_likert_time-interval decreasing}
{Panels A--B: Bias/precision plots for the fixed- and random-effect days-to-halfway elevation parameters, respectively ($\upbeta_{fixed}$ and $\upbeta_{random}$). Panels C--D: Bias/precision plots for the fixed- and random-effect triquarter-halfway elevation parameters, respectively ($\upgamma_{fixed}$ and $\upgamma_{random}$). Panels E--F: Bias/precision plots for the fixed- and random-effect baseline parameters, respectively ($\uptheta_{fixed}$ and $\uptheta_{random}$). Panels G--H: Bias/precision plots for the fixed- and random-effect maximal elevation parameters, respectively ($\upalpha_{fixed}$ and $\upalpha_{random}$). Blue horizontal lines in each panel represent the population value for each parameter. Population values for each day-unit parameter are as follows: $\upbeta_{fixed} \in$ {80, 180, 280}, $\upbeta_{random}$ = 10.00, $\upgamma_{fixed}$ = 20.00, $\upgamma_{random}$ = 4.00, $\uptheta_{fixed}$ = 3.00, $\uptheta_{random}$ = 0.05, $\upalpha_{fixed}$ = 3.32, $\upalpha_{random}$ = 0.05, $\upepsilon$ = 0.05. Gray bands indicate the $\pm 10\%$ margin of error for each parameter and unfilled dots indicate cells with average parameter estimates outside of the margin or biased estimates. Error bars represent the middle 95\% of estimated values, with light blue error bars indicating imprecise estimation. I considered dots that fell outside the gray bands as biased and error bar lengths with at least one whisker length exceeding the 10\% cutoff (i.e., or longer than the portion of the gray band underlying the whisker) as imprecise. Note that random-effect parameter units are in standard deviation units. See Table \ref{tab:param-exp-2} for specific values estimated for each parameter.}
[notrack]
\end{apaFigure}
\subapp{Middle-and-Extreme Spacing}
\begin{apaFigure}
[portrait]
[samepage]
[0cm]
{Bias/Precision Plots for Day- and Likert-Unit Parameters With Middle-and-Extreme Spacing in Experiment 2}
{exp2_plot_mid_ext_app}
{0.16}
{Figures/exp2_plot_days_middle-and-extreme spacing}
{}
\end{apaFigure}
\addtocounter{figure}{-1}
\begin{apaFigure}
[portrait]
[samepage]
[0cm]
{Bias/Precision Plots for Day- and Likert-Unit Parameters With Middle-and-Extreme Spacing in Experiment 2 (continued)}
{exp2_plot_mid_ext_app}
{0.16}
{Figures/exp2_plot_likert_middle-and-extreme spacing}
{Panels A--B:  Bias/precision plots for the fixed- and random-effect days-to-halfway elevation parameters, respectively ($\upbeta_{fixed}$ and $\upbeta_{random}$). Panels C--D: Bias/precision plots for the fixed- and random-effect triquarter-halfway elevation parameters, respectively ($\upgamma_{fixed}$ and $\upgamma_{random}$). Panels E--F: Bias/precision plots for the fixed- and random-effect baseline parameters, respectively ($\uptheta_{fixed}$ and $\uptheta_{random}$). Panels G--H: Bias/precision plots for the fixed- and random-effect maximal elevation parameters, respectively ($\upalpha_{fixed}$ and $\upalpha_{random}$). Blue horizontal lines in each panel represent the population value for each parameter. Population values for each day-unit parameter are as follows: $\upbeta_{fixed} \in$ {80, 180, 280}, $\upbeta_{random}$ = 10.00, $\upgamma_{fixed}$ = 20.00, $\upgamma_{random}$ = 4.00, $\uptheta_{fixed}$ = 3.00, $\uptheta_{random}$ = 0.05, $\upalpha_{fixed}$ = 3.32, $\upalpha_{random}$ = 0.05, $\upepsilon$ = 0.05. Gray bands indicate the $\pm 10\%$ margin of error for each parameter and unfilled dots indicate cells with average parameter estimates outside of the margin or biased estimates. Error bars represent the middle 95\% of estimated values, with light blue error bars indicating imprecise estimation. I considered dots that fell outside the gray bands as biased and error bar lengths with at least one whisker length exceeding the 10\% cutoff (i.e., or longer than the portion of the gray band underlying the whisker) as imprecise. Note that random-effect parameter units are in standard deviation units. See Table \ref{tab:param-exp-2} for specific values estimated for each parameter.}
[notrack]
\end{apaFigure}
\secapp{Experiment 3}
\subapp{Time-Structured Data}
\begin{apaFigure}
[portrait]
[samepage]
[0cm]
{Bias/Precision Plots for Day- and Likert-Unit Parameters With Time-Structured Data in Experiment 3}
{exp3_plot_days_time_struc_app}
{0.16}
{Figures/exp3_plot_days_time structured}
{}
\end{apaFigure}
\addtocounter{figure}{-1}
\begin{apaFigure}
[portrait]
[samepage]
[0cm]
{Bias/Precision Plots for Day- and Likert-Unit Parameters With Time-Structured Data in Experiment 3 (continued)}
{exp3_plot_days_time_struc_app}
{0.16}
{Figures/exp3_plot_likert_time structured}
{Panels A--B: Bias/precision plots for the fixed- and random-effect days-to-halfway elevation parameters, respectively ($\upbeta_{fixed}$ and $\upbeta_{random}$). Panels C--D: Bias/precision plots for the fixed- and random-effect triquarter-halfway elevation parameters, respectively ($\upgamma_{fixed}$ and $\upgamma_{random}$). Panels E--F: Bias/precision plots for the fixed- and random-effect baseline parameters, respectively ($\uptheta_{fixed}$ and $\uptheta_{random}$). Panels G--H: Bias/precision plots for the fixed- and random-effect maximal elevation parameters, respectively ($\upalpha_{fixed}$ and $\upalpha_{random}$). Blue horizontal lines in each panel represent the population value for each parameter. Population values for each day-unit parameter are as follows: $\upbeta_{fixed} \in$ {80, 180, 280}, $\upbeta_{random}$ = 10.00, $\upgamma_{fixed}$ = 20.00, $\upgamma_{random}$ = 4.00, $\uptheta_{fixed}$ = 3.00, $\uptheta_{random}$ = 0.05, $\upalpha_{fixed}$ = 3.32, $\upalpha_{random}$ = 0.05, $\upepsilon$ = 0.05. Gray bands indicate the $\pm 10\%$ margin of error for each parameter and unfilled dots indicate cells with average parameter estimates outside of the margin or biased estimates. Error bars represent the middle 95\% of estimated values, with light blue error bars indicating imprecise estimation. I considered dots that fell outside the gray bands as biased and error bar lengths with at least one whisker length exceeding the 10\% cutoff (i.e., or longer than the portion of the gray band underlying the whisker) as imprecise. Note that random-effect parameter units are in standard deviation units. See Table \ref{tab:param-exp-3} for specific values estimated for each parameter.}
[notrack]
\end{apaFigure}
\subapp{Time-Unstructured Data Characterized by a Fast Response Rate}
\begin{apaFigure}
[portrait]
[samepage]
[0cm]
{Bias/Precision Plots for Day- and Likert-Unit Parameters With Time-Unstructured Data Characterized by a Fast Response Rate in Experiment 3}
{exp3_plot_days_fast_app}
{0.16}
{Figures/exp3_plot_days_time unstructured (fast response)}
{}
\end{apaFigure}
\addtocounter{figure}{-1}
\begin{apaFigure}
[portrait]
[samepage]
[0cm]
{Bias/Precision Plots for Day- and Likert-Unit Parameters With Time-Unstructured Data Characterized by a Fast Response Rate in Experiment 3 (continued)}
{exp3_plot_days_fast_app}
{0.16}
{Figures/exp3_plot_likert_time unstructured (fast response)}
{Panels A--B: Bias/precision plots for the fixed- and random-effect days-to-halfway elevation parameters, respectively ($\upbeta_{fixed}$ and $\upbeta_{random}$). Panels C--D: Bias/precision plots for the fixed- and random-effect triquarter-halfway elevation parameters, respectively ($\upgamma_{fixed}$ and $\upgamma_{random}$). Panels E--F: Bias/precision plots for the fixed- and random-effect baseline parameters, respectively ($\uptheta_{fixed}$ and $\uptheta_{random}$). Panels G--H: Bias/precision plots for the fixed- and random-effect maximal elevation parameters, respectively ($\upalpha_{fixed}$ and $\upalpha_{random}$). Blue horizontal lines in each panel represent the population value for each parameter. Population values for each day-unit parameter are as follows: $\upbeta_{fixed} \in$ {80, 180, 280}, $\upbeta_{random}$ = 10.00, $\upgamma_{fixed}$ = 20.00, $\upgamma_{random}$ = 4.00, $\uptheta_{fixed}$ = 3.00, $\uptheta_{random}$ = 0.05, $\upalpha_{fixed}$ = 3.32, $\upalpha_{random}$ = 0.05, $\upepsilon$ = 0.05. Gray bands indicate the $\pm 10\%$ margin of error for each parameter and unfilled dots indicate cells with average parameter estimates outside of the margin or biased estimates. Error bars represent the middle 95\% of estimated values, with light blue error bars indicating imprecise estimation. I considered dots that fell outside the gray bands as biased and error bar lengths with at least one whisker length exceeding the 10\% cutoff (i.e., or longer than the portion of the gray band underlying the whisker) as imprecise. Note that random-effect parameter units are in standard deviation units. See Table \ref{tab:param-exp-3} for specific values estimated for each parameter.}
[notrack]
\end{apaFigure}
\subapp{Time-Unstructured Data Characterized by a Slow Response Rate}
\begin{apaFigure}
[portrait]
[samepage]
[0cm]
{Bias/Precision Plots for Day- and Likert-Unit Parameters With Time-Unstructured Data Characterized by a Slow Response Rate in Experiment 3}
{exp3_plot_days_slow_app}
{0.16}
{Figures/exp3_plot_days_time unstructured (slow response)}
{}
\end{apaFigure}
\addtocounter{figure}{-1}
\begin{apaFigure}
[portrait]
[samepage]
[0cm]
{Bias/Precision Plots for Day- and Likert-Unit Parameters With Time-Unstructured Data Characterized by a Slow Response Rate in Experiment 3 (continued)}
{exp3_plot_days_slow_app}
{0.16}
{Figures/exp3_plot_likert_time unstructured (slow response)}
{Panels A--B: Bias/precision plots for the fixed- and random-effect days-to-halfway elevation parameters, respectively ($\upbeta_{fixed}$ and $\upbeta_{random}$). Panels C--D: Bias/precision plots for the fixed- and random-effect triquarter-halfway elevation parameters, respectively ($\upgamma_{fixed}$ and $\upgamma_{random}$). Panels E--F: Bias/precision plots for the fixed- and random-effect baseline parameters, respectively ($\uptheta_{fixed}$ and $\uptheta_{random}$). Panels G--H: Bias/precision plots for the fixed- and random-effect maximal elevation parameters, respectively ($\upalpha_{fixed}$ and $\upalpha_{random}$). Blue horizontal lines in each panel represent the population value for each parameter. Population values for each day-unit parameter are as follows: $\upbeta_{fixed} \in$ {80, 180, 280}, $\upbeta_{random}$ = 10.00, $\upgamma_{fixed}$ = 20.00, $\upgamma_{random}$ = 4.00, $\uptheta_{fixed}$ = 3.00, $\uptheta_{random}$ = 0.05, $\upalpha_{fixed}$ = 3.32, $\upalpha_{random}$ = 0.05, $\upepsilon$ = 0.05. Gray bands indicate the $\pm 10\%$ margin of error for each parameter and unfilled dots indicate cells with average parameter estimates outside of the margin or biased estimates. Error bars represent the middle 95\% of estimated values, with light blue error bars indicating imprecise estimation. I considered dots that fell outside the gray bands as biased and error bar lengths with at least one whisker length exceeding the 10\% cutoff (i.e., or longer than the portion of the gray band underlying the whisker) as imprecise. Note that random-effect parameter units are in standard deviation units. See Table \ref{tab:param-exp-3} for specific values estimated for each parameter.}
[notrack]
\end{apaFigure}
\subapp{Time-Unstructured Data Characterized by a Slow Response Rate and Modelled with Definition Variables}
\begin{apaFigure}
[portrait]
[samepage]
[-0.2cm]
{Bias/Precision Plots for Day- and Likert-Unit Parameters When Using Definition Variables To Model Time-Unstructured Data Characterized by a Slow Response Rate}
{exp3_plot_days_def_app}
{0.165}
{Figures/exp3_defplot_days_time unstructured (slow response)}
{}
\end{apaFigure}
\addtocounter{figure}{-1}
\begin{apaFigure}
[portrait]
[samepage]
[-0.2cm]
{Bias/Precision Plots for Day- and Likert-Unit Parameters When Using Definition Variables To Model Time-Unstructured Data Characterized by a Slow Response Rate (continued)}
{exp3_plot_days_def_app}
{0.165}
{Figures/exp3_defplot_likert_time unstructured (slow response)}
{Panels A--B: Bias/precision plots for the fixed- and random-effect days-to-halfway elevation parameters, respectively ($\upbeta_{fixed}$ and $\upbeta_{random}$). Panels C--D: Bias/precision plots for the fixed- and random-effect triquarter-halfway elevation parameters, respectively ($\upgamma_{fixed}$ and $\upgamma_{random}$). Panels E--F: Bias/precision plots for the fixed- and random-effect baseline parameters, respectively ($\uptheta_{fixed}$ and $\uptheta_{random}$). Panels G--H: Bias/precision plots for the fixed- and random-effect maximal elevation parameters, respectively ($\upalpha_{fixed}$ and $\upalpha_{random}$). Blue horizontal lines in each panel represent the population value for each parameter. Population values for each day-unit parameter are as follows: $\upbeta_{fixed} \in$ {80, 180, 280}, $\upbeta_{random}$ = 10.00, $\upgamma_{fixed}$ = 20.00, $\upgamma_{random}$ = 4.00, $\uptheta_{fixed}$ = 3.00, $\uptheta_{random}$ = 0.05, $\upalpha_{fixed}$ = 3.32, $\upalpha_{random}$ = 0.05, $\upepsilon$ = 0.05. Gray bands indicate the $\pm 10\%$ margin of error for each parameter and unfilled dots indicate cells with average parameter estimates outside of the margin or biased estimates. Error bars represent the middle 95\% of estimated values, with light blue error bars indicating imprecise estimation. I considered dots that fell outside the gray bands as biased and error bar lengths with at least one whisker length exceeding the 10\% cutoff (i.e., or longer than the portion of the gray band underlying the whisker) as imprecise. Note that random-effect parameter units are in standard deviation units. See Table \ref{tab:param-exp-3} for specific values estimated for each parameter.}
\end{apaFigure}
\app{Convergence Success Rates}

\label{convergence-tables}

\secapp{Experiment 1}
\begin{ThreePartTable}
\begin{TableNotes}
\item \textit{Note. }Cells shaded in gray indicate conditions where less than 90\% of models converged.
\end{TableNotes}
\begin{longtable}[l]{>{\raggedright\arraybackslash}p{3cm}>{\raggedright\arraybackslash}p{3cm}>{\centering\arraybackslash}p{1cm}>{\centering\arraybackslash}p{1cm}>{\centering\arraybackslash}p{1cm}>{}p{1cm}>{}p{1cm}>{}p{1cm}>{}p{1cm}>{}p{1cm}>{}p{1cm}>{}p{1cm}>{}p{1cm}>{}p{1cm}>{}p{1cm}>{}p{1cm}}
\caption{\label{tab:conv-exp-1}Convergence Success Rates in Experiment 1}\\
\toprule
\multicolumn{1}{c}{} & \multicolumn{1}{c}{} & \multicolumn{3}{c}{Days to Halfway Elevation} \\
\cmidrule(l{3pt}r{3pt}){3-5}
Measurement Spacing & Number of Measurements & 80 & 180 & 280\\
\midrule
 & 5 & \cellcolor[HTML]{ffffff}{1.00} & \cellcolor[HTML]{ffffff}{0.98} & \cellcolor[HTML]{ffffff}{0.95}\\
\nopagebreak
 & 7 & \cellcolor[HTML]{ffffff}{1.00} & \cellcolor[HTML]{ffffff}{1.00} & \cellcolor[HTML]{ffffff}{0.99}\\
\nopagebreak
 & 9 & \cellcolor[HTML]{ffffff}{1.00} & \cellcolor[HTML]{ffffff}{1.00} & \cellcolor[HTML]{ffffff}{1.00}\\
\nopagebreak
\multirow{-4}{3cm}{\raggedright\arraybackslash Equal} & 11 & \cellcolor[HTML]{ffffff}{1.00} & \cellcolor[HTML]{ffffff}{1.00} & \cellcolor[HTML]{ffffff}{1.00}\\
\cmidrule{1-5}\pagebreak[0]
 & 5 & \cellcolor[HTML]{ffffff}{1.00} & \cellcolor[HTML]{ffffff}{1.00} & \cellcolor[HTML]{ffffff}{1.00}\\
\nopagebreak
 & 7 & \cellcolor[HTML]{ffffff}{1.00} & \cellcolor[HTML]{ffffff}{1.00} & \cellcolor[HTML]{ffffff}{1.00}\\
\nopagebreak
 & 9 & \cellcolor[HTML]{ffffff}{1.00} & \cellcolor[HTML]{ffffff}{1.00} & \cellcolor[HTML]{ffffff}{1.00}\\
\nopagebreak
\multirow{-4}{3cm}{\raggedright\arraybackslash Time-interval increasing} & 11 & \cellcolor[HTML]{ffffff}{1.00} & \cellcolor[HTML]{ffffff}{1.00} & \cellcolor[HTML]{ffffff}{1.00}\\
\cmidrule{1-5}\pagebreak[0]
 & 5 & \cellcolor[HTML]{ffffff}{1.00} & \cellcolor[HTML]{ffffff}{0.96} & \cellcolor[HTML]{eeeeee}{0.82}\\
\nopagebreak
 & 7 & \cellcolor[HTML]{ffffff}{1.00} & \cellcolor[HTML]{ffffff}{0.99} & \cellcolor[HTML]{ffffff}{0.98}\\
\nopagebreak
 & 9 & \cellcolor[HTML]{ffffff}{1.00} & \cellcolor[HTML]{ffffff}{1.00} & \cellcolor[HTML]{ffffff}{1.00}\\
\nopagebreak
\multirow{-4}{3cm}{\raggedright\arraybackslash Time-interval decreasing} & 11 & \cellcolor[HTML]{ffffff}{1.00} & \cellcolor[HTML]{ffffff}{1.00} & \cellcolor[HTML]{ffffff}{1.00}\\
\cmidrule{1-5}\pagebreak[0]
 & 5 & \cellcolor[HTML]{ffffff}{1.00} & \cellcolor[HTML]{ffffff}{0.96} & \cellcolor[HTML]{eeeeee}{0.86}\\
\nopagebreak
 & 7 & \cellcolor[HTML]{ffffff}{1.00} & \cellcolor[HTML]{ffffff}{1.00} & \cellcolor[HTML]{ffffff}{1.00}\\
\nopagebreak
 & 9 & \cellcolor[HTML]{ffffff}{1.00} & \cellcolor[HTML]{ffffff}{1.00} & \cellcolor[HTML]{ffffff}{1.00}\\
\nopagebreak
\multirow{-4}{3cm}{\raggedright\arraybackslash Middle-and-extreme} & 11 & \cellcolor[HTML]{ffffff}{1.00} & \cellcolor[HTML]{ffffff}{1.00} & \cellcolor[HTML]{ffffff}{1.00}\\
\bottomrule
\insertTableNotes
\end{longtable}
\end{ThreePartTable}
\secapp{Experiment 2}
\begin{ThreePartTable}
\begin{TableNotes}
\item \textit{Note. }Cells shaded in gray indicate conditions where less than 90\% of models converged.
\end{TableNotes}
\begin{longtable}[l]{>{\raggedright\arraybackslash}p{3cm}>{\raggedright\arraybackslash}p{3cm}>{\centering\arraybackslash}p{1cm}>{\centering\arraybackslash}p{1cm}>{\centering\arraybackslash}p{1cm}>{\centering\arraybackslash}p{1cm}>{\centering\arraybackslash}p{1cm}>{\centering\arraybackslash}p{1cm}>{}p{1cm}>{}p{1cm}>{}p{1cm}>{}p{1cm}>{}p{1cm}>{}p{1cm}>{}p{1cm}>{}p{1cm}}
\caption{\label{tab:conv-exp-2}Convergence Success Rates in Experiment 2}\\
\toprule
\multicolumn{1}{c}{} & \multicolumn{1}{c}{} & \multicolumn{6}{c}{Sample Size (\textit{N})} \\
\cmidrule(l{3pt}r{3pt}){3-8}
Measurement Spacing & Number of Measurements & 30 & 50 & 100 & 200 & 500 & 1000\\
\midrule
 & 5 & \cellcolor[HTML]{ffffff}{1.00} & \cellcolor[HTML]{ffffff}{1.00} & \cellcolor[HTML]{ffffff}{0.99} & \cellcolor[HTML]{ffffff}{0.98} & \cellcolor[HTML]{ffffff}{0.95} & \cellcolor[HTML]{ffffff}{0.92}\\
\nopagebreak
 & 7 & \cellcolor[HTML]{ffffff}{1.00} & \cellcolor[HTML]{ffffff}{1.00} & \cellcolor[HTML]{ffffff}{1.00} & \cellcolor[HTML]{ffffff}{1.00} & \cellcolor[HTML]{ffffff}{0.99} & \cellcolor[HTML]{ffffff}{0.98}\\
\nopagebreak
 & 9 & \cellcolor[HTML]{ffffff}{1.00} & \cellcolor[HTML]{ffffff}{1.00} & \cellcolor[HTML]{ffffff}{1.00} & \cellcolor[HTML]{ffffff}{1.00} & \cellcolor[HTML]{ffffff}{1.00} & \cellcolor[HTML]{ffffff}{1.00}\\
\nopagebreak
\multirow{-4}{3cm}{\raggedright\arraybackslash Equal} & 11 & \cellcolor[HTML]{ffffff}{1.00} & \cellcolor[HTML]{ffffff}{1.00} & \cellcolor[HTML]{ffffff}{1.00} & \cellcolor[HTML]{ffffff}{1.00} & \cellcolor[HTML]{ffffff}{1.00} & \cellcolor[HTML]{ffffff}{1.00}\\
\cmidrule{1-8}\pagebreak[0]
 & 5 & \cellcolor[HTML]{ffffff}{1.00} & \cellcolor[HTML]{ffffff}{1.00} & \cellcolor[HTML]{ffffff}{1.00} & \cellcolor[HTML]{ffffff}{1.00} & \cellcolor[HTML]{ffffff}{1.00} & \cellcolor[HTML]{ffffff}{1.00}\\
\nopagebreak
 & 7 & \cellcolor[HTML]{ffffff}{1.00} & \cellcolor[HTML]{ffffff}{1.00} & \cellcolor[HTML]{ffffff}{1.00} & \cellcolor[HTML]{ffffff}{1.00} & \cellcolor[HTML]{ffffff}{1.00} & \cellcolor[HTML]{ffffff}{1.00}\\
\nopagebreak
 & 9 & \cellcolor[HTML]{ffffff}{1.00} & \cellcolor[HTML]{ffffff}{1.00} & \cellcolor[HTML]{ffffff}{1.00} & \cellcolor[HTML]{ffffff}{1.00} & \cellcolor[HTML]{ffffff}{1.00} & \cellcolor[HTML]{ffffff}{1.00}\\
\nopagebreak
\multirow{-4}{3cm}{\raggedright\arraybackslash Time-interval increasing} & 11 & \cellcolor[HTML]{ffffff}{1.00} & \cellcolor[HTML]{ffffff}{1.00} & \cellcolor[HTML]{ffffff}{1.00} & \cellcolor[HTML]{ffffff}{1.00} & \cellcolor[HTML]{ffffff}{1.00} & \cellcolor[HTML]{ffffff}{1.00}\\
\cmidrule{1-8}\pagebreak[0]
 & 5 & \cellcolor[HTML]{ffffff}{1.00} & \cellcolor[HTML]{ffffff}{0.99} & \cellcolor[HTML]{ffffff}{0.98} & \cellcolor[HTML]{ffffff}{0.95} & \cellcolor[HTML]{ffffff}{0.93} & \cellcolor[HTML]{eeeeee}{0.88}\\
\nopagebreak
 & 7 & \cellcolor[HTML]{ffffff}{1.00} & \cellcolor[HTML]{ffffff}{1.00} & \cellcolor[HTML]{ffffff}{0.99} & \cellcolor[HTML]{ffffff}{0.99} & \cellcolor[HTML]{ffffff}{0.98} & \cellcolor[HTML]{ffffff}{0.95}\\
\nopagebreak
 & 9 & \cellcolor[HTML]{ffffff}{1.00} & \cellcolor[HTML]{ffffff}{1.00} & \cellcolor[HTML]{ffffff}{1.00} & \cellcolor[HTML]{ffffff}{1.00} & \cellcolor[HTML]{ffffff}{1.00} & \cellcolor[HTML]{ffffff}{0.99}\\
\nopagebreak
\multirow{-4}{3cm}{\raggedright\arraybackslash Time-interval decreasing} & 11 & \cellcolor[HTML]{ffffff}{1.00} & \cellcolor[HTML]{ffffff}{1.00} & \cellcolor[HTML]{ffffff}{1.00} & \cellcolor[HTML]{ffffff}{1.00} & \cellcolor[HTML]{ffffff}{1.00} & \cellcolor[HTML]{ffffff}{1.00}\\
\cmidrule{1-8}\pagebreak[0]
 & 5 & \cellcolor[HTML]{ffffff}{1.00} & \cellcolor[HTML]{ffffff}{0.99} & \cellcolor[HTML]{ffffff}{0.98} & \cellcolor[HTML]{ffffff}{0.96} & \cellcolor[HTML]{eeeeee}{0.90} & \cellcolor[HTML]{eeeeee}{0.81}\\
\nopagebreak
 & 7 & \cellcolor[HTML]{ffffff}{1.00} & \cellcolor[HTML]{ffffff}{1.00} & \cellcolor[HTML]{ffffff}{1.00} & \cellcolor[HTML]{ffffff}{1.00} & \cellcolor[HTML]{ffffff}{1.00} & \cellcolor[HTML]{ffffff}{1.00}\\
\nopagebreak
 & 9 & \cellcolor[HTML]{ffffff}{1.00} & \cellcolor[HTML]{ffffff}{1.00} & \cellcolor[HTML]{ffffff}{1.00} & \cellcolor[HTML]{ffffff}{1.00} & \cellcolor[HTML]{ffffff}{1.00} & \cellcolor[HTML]{ffffff}{1.00}\\
\nopagebreak
\multirow{-4}{3cm}{\raggedright\arraybackslash Middle-and-extreme} & 11 & \cellcolor[HTML]{ffffff}{1.00} & \cellcolor[HTML]{ffffff}{1.00} & \cellcolor[HTML]{ffffff}{1.00} & \cellcolor[HTML]{ffffff}{1.00} & \cellcolor[HTML]{ffffff}{1.00} & \cellcolor[HTML]{ffffff}{1.00}\\
\bottomrule
\insertTableNotes
\end{longtable}
\end{ThreePartTable}
\secapp{Experiment 3}

\label{conv-exp-3}
\begin{ThreePartTable}
\begin{TableNotes}
\item \textit{Note. }Cells shaded in gray indicate conditions where less than 90\% of models converged.
\end{TableNotes}
\begin{longtable}[l]{>{\raggedright\arraybackslash}p{3cm}>{\raggedright\arraybackslash}p{3cm}>{\centering\arraybackslash}p{1cm}>{\centering\arraybackslash}p{1cm}>{\centering\arraybackslash}p{1cm}>{\centering\arraybackslash}p{1cm}>{\centering\arraybackslash}p{1cm}>{\centering\arraybackslash}p{1cm}>{}p{1cm}>{}p{1cm}>{}p{1cm}>{}p{1cm}>{}p{1cm}>{}p{1cm}>{}p{1cm}>{}p{1cm}}
\caption{\label{tab:conv-exp-3}Convergence Success Rates in Experiment 3}\\
\toprule
\multicolumn{1}{c}{} & \multicolumn{1}{c}{} & \multicolumn{6}{c}{Sample Size (\textit{N})} \\
\cmidrule(l{3pt}r{3pt}){3-8}
Time Structuredness & Number of Measurements & 30 & 50 & 100 & 200 & 500 & 1000\\
\midrule
 & 5 & \cellcolor[HTML]{ffffff}{1.00} & \cellcolor[HTML]{ffffff}{0.99} & \cellcolor[HTML]{ffffff}{0.99} & \cellcolor[HTML]{ffffff}{0.98} & \cellcolor[HTML]{ffffff}{0.96} & \cellcolor[HTML]{eeeeee}{0.90}\\
\nopagebreak
 & 7 & \cellcolor[HTML]{ffffff}{1.00} & \cellcolor[HTML]{ffffff}{1.00} & \cellcolor[HTML]{ffffff}{1.00} & \cellcolor[HTML]{ffffff}{1.00} & \cellcolor[HTML]{ffffff}{0.99} & \cellcolor[HTML]{ffffff}{0.98}\\
\nopagebreak
 & 9 & \cellcolor[HTML]{ffffff}{1.00} & \cellcolor[HTML]{ffffff}{1.00} & \cellcolor[HTML]{ffffff}{1.00} & \cellcolor[HTML]{ffffff}{1.00} & \cellcolor[HTML]{ffffff}{1.00} & \cellcolor[HTML]{ffffff}{1.00}\\
\nopagebreak
\multirow{-4}{3cm}{\raggedright\arraybackslash Time structured} & 11 & \cellcolor[HTML]{ffffff}{1.00} & \cellcolor[HTML]{ffffff}{1.00} & \cellcolor[HTML]{ffffff}{1.00} & \cellcolor[HTML]{ffffff}{1.00} & \cellcolor[HTML]{ffffff}{1.00} & \cellcolor[HTML]{ffffff}{1.00}\\
\cmidrule{1-8}\pagebreak[0]
 & 5 & \cellcolor[HTML]{ffffff}{1.00} & \cellcolor[HTML]{ffffff}{1.00} & \cellcolor[HTML]{ffffff}{0.98} & \cellcolor[HTML]{ffffff}{0.99} & \cellcolor[HTML]{ffffff}{0.96} & \cellcolor[HTML]{ffffff}{0.90}\\
\nopagebreak
 & 7 & \cellcolor[HTML]{ffffff}{1.00} & \cellcolor[HTML]{ffffff}{1.00} & \cellcolor[HTML]{ffffff}{1.00} & \cellcolor[HTML]{ffffff}{0.99} & \cellcolor[HTML]{ffffff}{0.98} & \cellcolor[HTML]{ffffff}{0.99}\\
\nopagebreak
 & 9 & \cellcolor[HTML]{ffffff}{1.00} & \cellcolor[HTML]{ffffff}{1.00} & \cellcolor[HTML]{ffffff}{1.00} & \cellcolor[HTML]{ffffff}{1.00} & \cellcolor[HTML]{ffffff}{1.00} & \cellcolor[HTML]{ffffff}{1.00}\\
\nopagebreak
\multirow{-4}{3cm}{\raggedright\arraybackslash Time unstructured (fast response)} & 11 & \cellcolor[HTML]{ffffff}{1.00} & \cellcolor[HTML]{ffffff}{1.00} & \cellcolor[HTML]{ffffff}{1.00} & \cellcolor[HTML]{ffffff}{1.00} & \cellcolor[HTML]{ffffff}{1.00} & \cellcolor[HTML]{ffffff}{1.00}\\
\cmidrule{1-8}\pagebreak[0]
 & 5 & \cellcolor[HTML]{ffffff}{1.00} & \cellcolor[HTML]{ffffff}{1.00} & \cellcolor[HTML]{ffffff}{0.99} & \cellcolor[HTML]{ffffff}{1.00} & \cellcolor[HTML]{ffffff}{0.95} & \cellcolor[HTML]{ffffff}{0.92}\\
\nopagebreak
 & 7 & \cellcolor[HTML]{ffffff}{1.00} & \cellcolor[HTML]{ffffff}{1.00} & \cellcolor[HTML]{ffffff}{1.00} & \cellcolor[HTML]{ffffff}{0.99} & \cellcolor[HTML]{ffffff}{0.99} & \cellcolor[HTML]{ffffff}{0.98}\\
\nopagebreak
 & 9 & \cellcolor[HTML]{ffffff}{1.00} & \cellcolor[HTML]{ffffff}{1.00} & \cellcolor[HTML]{ffffff}{1.00} & \cellcolor[HTML]{ffffff}{1.00} & \cellcolor[HTML]{ffffff}{1.00} & \cellcolor[HTML]{ffffff}{1.00}\\
\nopagebreak
\multirow{-4}{3cm}{\raggedright\arraybackslash Time unstructured (slow response)} & 11 & \cellcolor[HTML]{ffffff}{1.00} & \cellcolor[HTML]{ffffff}{1.00} & \cellcolor[HTML]{ffffff}{1.00} & \cellcolor[HTML]{ffffff}{1.00} & \cellcolor[HTML]{ffffff}{1.00} & \cellcolor[HTML]{ffffff}{1.00}\\
\cmidrule{1-8}\pagebreak[0]
 & 5 & \cellcolor[HTML]{ffffff}{1.00} & \cellcolor[HTML]{ffffff}{1.00} & \cellcolor[HTML]{ffffff}{1.00} & \cellcolor[HTML]{ffffff}{1.00} & \cellcolor[HTML]{ffffff}{0.99} & \cellcolor[HTML]{ffffff}{0.98}\\
\nopagebreak
 & 7 & \cellcolor[HTML]{ffffff}{1.00} & \cellcolor[HTML]{ffffff}{1.00} & \cellcolor[HTML]{ffffff}{1.00} & \cellcolor[HTML]{ffffff}{1.00} & \cellcolor[HTML]{ffffff}{1.00} & \cellcolor[HTML]{ffffff}{0.99}\\
\nopagebreak
 & 9 & \cellcolor[HTML]{ffffff}{1.00} & \cellcolor[HTML]{ffffff}{1.00} & \cellcolor[HTML]{ffffff}{1.00} & \cellcolor[HTML]{ffffff}{1.00} & \cellcolor[HTML]{ffffff}{1.00} & \cellcolor[HTML]{ffffff}{1.00}\\
\nopagebreak
\multirow{-4}{3cm}{\raggedright\arraybackslash Time unstructured (slow response) with definition variables} & 11 & \cellcolor[HTML]{ffffff}{1.00} & \cellcolor[HTML]{ffffff}{1.00} & \cellcolor[HTML]{ffffff}{1.00} & \cellcolor[HTML]{ffffff}{1.00} & \cellcolor[HTML]{ffffff}{1.00} & \cellcolor[HTML]{ffffff}{1.00}\\
\bottomrule
\insertTableNotes
\end{longtable}
\end{ThreePartTable}
\begin{ThreePartTable}
\begin{TableNotes}
\item \textit{Note. }Cells shaded in gray indicate conditions where less than 90\% of models converged.
\end{TableNotes}
\begin{longtable}[l]{>{\raggedright\arraybackslash}p{3cm}>{\raggedright\arraybackslash}p{3cm}>{\centering\arraybackslash}p{1cm}>{\centering\arraybackslash}p{1cm}cccc}
\caption{\label{tab:conv-exp-3-def}Convergence Success in Experiment 3 With Definition Variables  }\\
\toprule
\multicolumn{1}{c}{} & \multicolumn{1}{c}{} & \multicolumn{6}{c}{Sample size (\textit{N})} \\
\cmidrule(l{3pt}r{3pt}){3-8}
Time Structuredness & Number of Measurements & 30 & 50 & 100 & 200 & 500 & 1000\\
\midrule
 & 5 & \cellcolor[HTML]{ffffff}{1.00} & \cellcolor[HTML]{ffffff}{1.00} & \cellcolor[HTML]{ffffff}{1.00} & \cellcolor[HTML]{ffffff}{1.00} & \cellcolor[HTML]{ffffff}{0.99} & \cellcolor[HTML]{ffffff}{0.98}\\
\nopagebreak
 & 7 & \cellcolor[HTML]{ffffff}{1.00} & \cellcolor[HTML]{ffffff}{1.00} & \cellcolor[HTML]{ffffff}{1.00} & \cellcolor[HTML]{ffffff}{1.00} & \cellcolor[HTML]{ffffff}{1.00} & \cellcolor[HTML]{ffffff}{0.99}\\
\nopagebreak
 & 9 & \cellcolor[HTML]{ffffff}{1.00} & \cellcolor[HTML]{ffffff}{1.00} & \cellcolor[HTML]{ffffff}{1.00} & \cellcolor[HTML]{ffffff}{1.00} & \cellcolor[HTML]{ffffff}{1.00} & \cellcolor[HTML]{ffffff}{1.00}\\
\nopagebreak
\multirow{-4}{3cm}{\raggedright\arraybackslash Time unstructured (slow response) with definition variables} & 11 & \cellcolor[HTML]{ffffff}{1.00} & \cellcolor[HTML]{ffffff}{1.00} & \cellcolor[HTML]{ffffff}{1.00} & \cellcolor[HTML]{ffffff}{1.00} & \cellcolor[HTML]{ffffff}{1.00} & \cellcolor[HTML]{ffffff}{1.00}\\
\bottomrule
\insertTableNotes
\end{longtable}
\end{ThreePartTable}
\app{Parameter Estimate Tables}
\secapp{Experiment 1}

\newgeometry{margin=2.54cm}
\begin{landscape}
\begin{ThreePartTable}
\begin{longtable}[l]{>{\raggedright\arraybackslash}p{3cm}>{\raggedright\arraybackslash}p{3cm}cccccccccccc}
\caption{Parameter Values Estimated for Day- and Likert-Unit Parameters in Experiment 1}\label{tab:param-exp-1}\\
\toprule
\multicolumn{1}{c}{} & \multicolumn{1}{c}{} & \multicolumn{3}{c}{\thead{$\upbeta_{fixed}$ (Days to \\ halfway elevation)}} & \multicolumn{3}{c}{\thead{$\upbeta_{random}$ (Days to \\ halfway elevation) \\ Pop value = 10.00}} & \multicolumn{3}{c}{\thead{$\upgamma_{fixed}$ (Triquarter- \\ halfway delta) \\ Pop value = 20.00}} & \multicolumn{3}{c}{\thead{$\upgamma_{random}$ (Triquarter- \\ halfway delta) \\ Pop value = 4.00}} \\
\cmidrule(l{3pt}r{3pt}){3-5} \cmidrule(l{3pt}r{3pt}){6-8} \cmidrule(l{3pt}r{3pt}){9-11} \cmidrule(l{3pt}r{3pt}){12-14}
Measurement Spacing & Number of Measurements & 80 & 180 & 280 & 80 & 180 & 280 & 80 & 180 & 280 & 80 & 180 & 280\\
\midrule
 & 5 & \cellcolor[HTML]{ffffff}{79.73} & \cellcolor[HTML]{ffffff}{179.78} & \cellcolor[HTML]{ffffff}{279.81$^{\square}$} & \cellcolor[HTML]{8cb9e3}{10.14} & \cellcolor[HTML]{8cb9e3}{10.40} & \cellcolor[HTML]{8cb9e3}{10.08} & \cellcolor[HTML]{8cb9e3}{19.37} & \cellcolor[HTML]{8cb9e3}{19.49} & \cellcolor[HTML]{8cb9e3}{19.71} & \cellcolor[HTML]{8cb9e3}{7.41$^{\square}$} & \cellcolor[HTML]{8cb9e3}{14.53$^{\square}$} & \cellcolor[HTML]{8cb9e3}{8.11$^{\square}$}\\
\nopagebreak
 & 7 & \cellcolor[HTML]{ffffff}{80.21} & \cellcolor[HTML]{ffffff}{178.99} & \cellcolor[HTML]{ffffff}{279.55$^{\square}$} & \cellcolor[HTML]{8cb9e3}{10.16} & \cellcolor[HTML]{8cb9e3}{10.55} & \cellcolor[HTML]{8cb9e3}{10.13} & \cellcolor[HTML]{8cb9e3}{20.67} & \cellcolor[HTML]{8cb9e3}{20.83} & \cellcolor[HTML]{8cb9e3}{20.60} & \cellcolor[HTML]{8cb9e3}{4.37} & \cellcolor[HTML]{8cb9e3}{ 5.14$^{\square}$} & \cellcolor[HTML]{8cb9e3}{4.41$^{\square}$}\\
\nopagebreak
 & 9 & \cellcolor[HTML]{ffffff}{80.00} & \cellcolor[HTML]{ffffff}{179.94} & \cellcolor[HTML]{ffffff}{279.99$^{\square}$} & \cellcolor[HTML]{8cb9e3}{10.29} & \cellcolor[HTML]{8cb9e3}{10.37} & \cellcolor[HTML]{8cb9e3}{10.34} & \cellcolor[HTML]{8cb9e3}{20.77} & \cellcolor[HTML]{8cb9e3}{20.76} & \cellcolor[HTML]{8cb9e3}{20.67} & \cellcolor[HTML]{8cb9e3}{4.24} & \cellcolor[HTML]{8cb9e3}{ 4.14} & \cellcolor[HTML]{8cb9e3}{4.30}\\
\nopagebreak
\multirow{-4}{3cm}{\raggedright\arraybackslash Equal spacing} & 11 & \cellcolor[HTML]{ffffff}{80.03} & \cellcolor[HTML]{ffffff}{180.01} & \cellcolor[HTML]{ffffff}{279.88$^{\square}$} & \cellcolor[HTML]{8cb9e3}{10.27} & \cellcolor[HTML]{8cb9e3}{10.29} & \cellcolor[HTML]{8cb9e3}{10.32} & \cellcolor[HTML]{8cb9e3}{20.64} & \cellcolor[HTML]{8cb9e3}{20.70} & \cellcolor[HTML]{8cb9e3}{20.64} & \cellcolor[HTML]{8cb9e3}{4.13} & \cellcolor[HTML]{8cb9e3}{ 4.08} & \cellcolor[HTML]{8cb9e3}{4.18}\\
\cmidrule{1-14}\pagebreak[0]
 & 5 & \cellcolor[HTML]{ffffff}{79.88} & \cellcolor[HTML]{8cb9e3}{180.10} & \cellcolor[HTML]{8cb9e3}{274.37$^{\square}$} & \cellcolor[HTML]{8cb9e3}{10.32} & \cellcolor[HTML]{8cb9e3}{ 9.73} & \cellcolor[HTML]{8cb9e3}{13.04$^{\square}$} & \cellcolor[HTML]{8cb9e3}{20.71} & \cellcolor[HTML]{8cb9e3}{20.39} & \cellcolor[HTML]{8cb9e3}{18.32} & \cellcolor[HTML]{8cb9e3}{4.57$^{\square}$} & \cellcolor[HTML]{8cb9e3}{ 4.99$^{\square}$} & \cellcolor[HTML]{8cb9e3}{6.20$^{\square}$}\\
\nopagebreak
 & 7 & \cellcolor[HTML]{ffffff}{80.19} & \cellcolor[HTML]{ffffff}{179.82} & \cellcolor[HTML]{ffffff}{279.86$^{\square}$} & \cellcolor[HTML]{8cb9e3}{10.42} & \cellcolor[HTML]{8cb9e3}{10.47} & \cellcolor[HTML]{8cb9e3}{10.14} & \cellcolor[HTML]{8cb9e3}{20.66} & \cellcolor[HTML]{8cb9e3}{20.79} & \cellcolor[HTML]{8cb9e3}{19.78} & \cellcolor[HTML]{8cb9e3}{4.29} & \cellcolor[HTML]{8cb9e3}{ 4.87$^{\square}$} & \cellcolor[HTML]{8cb9e3}{7.03$^{\square}$}\\
\nopagebreak
 & 9 & \cellcolor[HTML]{ffffff}{79.59} & \cellcolor[HTML]{ffffff}{179.06} & \cellcolor[HTML]{ffffff}{279.70$^{\square}$} & \cellcolor[HTML]{8cb9e3}{10.07} & \cellcolor[HTML]{8cb9e3}{10.22} & \cellcolor[HTML]{8cb9e3}{10.20} & \cellcolor[HTML]{8cb9e3}{20.33} & \cellcolor[HTML]{8cb9e3}{20.66} & \cellcolor[HTML]{8cb9e3}{20.72} & \cellcolor[HTML]{8cb9e3}{4.17} & \cellcolor[HTML]{8cb9e3}{ 4.25} & \cellcolor[HTML]{8cb9e3}{4.32}\\
\nopagebreak
\multirow{-4}{3cm}{\raggedright\arraybackslash Time-interval increasing} & 11 & \cellcolor[HTML]{ffffff}{79.89} & \cellcolor[HTML]{ffffff}{179.84} & \cellcolor[HTML]{ffffff}{279.62$^{\square}$} & \cellcolor[HTML]{8cb9e3}{10.38} & \cellcolor[HTML]{8cb9e3}{10.30} & \cellcolor[HTML]{8cb9e3}{10.47} & \cellcolor[HTML]{8cb9e3}{20.78} & \cellcolor[HTML]{8cb9e3}{20.75} & \cellcolor[HTML]{8cb9e3}{20.68} & \cellcolor[HTML]{8cb9e3}{4.23} & \cellcolor[HTML]{8cb9e3}{ 4.18} & \cellcolor[HTML]{8cb9e3}{4.13}\\
\cmidrule{1-14}\pagebreak[0]
 & 5 & \cellcolor[HTML]{8cb9e3}{70.67} & \cellcolor[HTML]{8cb9e3}{179.92} & \cellcolor[HTML]{ffffff}{279.63$^{\square}$} & \cellcolor[HTML]{8cb9e3}{15.28$^{\square}$} & \cellcolor[HTML]{8cb9e3}{ 9.80} & \cellcolor[HTML]{8cb9e3}{10.22} & \cellcolor[HTML]{8cb9e3}{16.63} & \cellcolor[HTML]{8cb9e3}{20.07} & \cellcolor[HTML]{8cb9e3}{20.55} & \cellcolor[HTML]{8cb9e3}{5.48$^{\square}$} & \cellcolor[HTML]{8cb9e3}{ 5.17$^{\square}$} & \cellcolor[HTML]{8cb9e3}{4.59$^{\square}$}\\
\nopagebreak
 & 7 & \cellcolor[HTML]{8cb9e3}{78.23} & \cellcolor[HTML]{ffffff}{178.22} & \cellcolor[HTML]{ffffff}{279.84$^{\square}$} & \cellcolor[HTML]{8cb9e3}{10.08} & \cellcolor[HTML]{8cb9e3}{10.46} & \cellcolor[HTML]{8cb9e3}{10.39} & \cellcolor[HTML]{8cb9e3}{19.38} & \cellcolor[HTML]{8cb9e3}{20.59} & \cellcolor[HTML]{8cb9e3}{20.69} & \cellcolor[HTML]{8cb9e3}{6.80$^{\square}$} & \cellcolor[HTML]{8cb9e3}{ 5.09$^{\square}$} & \cellcolor[HTML]{8cb9e3}{4.24}\\
\nopagebreak
 & 9 & \cellcolor[HTML]{ffffff}{79.95} & \cellcolor[HTML]{ffffff}{179.34} & \cellcolor[HTML]{ffffff}{278.98$^{\square}$} & \cellcolor[HTML]{8cb9e3}{10.03} & \cellcolor[HTML]{8cb9e3}{10.20} & \cellcolor[HTML]{8cb9e3}{10.05} & \cellcolor[HTML]{8cb9e3}{20.42} & \cellcolor[HTML]{8cb9e3}{20.54} & \cellcolor[HTML]{8cb9e3}{20.28} & \cellcolor[HTML]{8cb9e3}{4.37} & \cellcolor[HTML]{8cb9e3}{ 4.32} & \cellcolor[HTML]{8cb9e3}{4.19}\\
\nopagebreak
\multirow{-4}{3cm}{\raggedright\arraybackslash Time-interval decreasing} & 11 & \cellcolor[HTML]{ffffff}{79.42} & \cellcolor[HTML]{ffffff}{179.70} & \cellcolor[HTML]{ffffff}{279.52$^{\square}$} & \cellcolor[HTML]{8cb9e3}{10.38} & \cellcolor[HTML]{8cb9e3}{10.13} & \cellcolor[HTML]{8cb9e3}{10.06} & \cellcolor[HTML]{8cb9e3}{20.75} & \cellcolor[HTML]{8cb9e3}{20.45} & \cellcolor[HTML]{8cb9e3}{20.31} & \cellcolor[HTML]{8cb9e3}{4.17} & \cellcolor[HTML]{8cb9e3}{ 4.16} & \cellcolor[HTML]{8cb9e3}{4.17}\\
\cmidrule{1-14}\pagebreak[0]
 & 5 & \cellcolor[HTML]{8cb9e3}{71.95} & \cellcolor[HTML]{ffffff}{179.61} & \cellcolor[HTML]{8cb9e3}{287.73$^{\square}$} & \cellcolor[HTML]{8cb9e3}{16.78$^{\square}$} & \cellcolor[HTML]{8cb9e3}{10.26} & \cellcolor[HTML]{8cb9e3}{16.74$^{\square}$} & \cellcolor[HTML]{8cb9e3}{15.59} & \cellcolor[HTML]{8cb9e3}{20.61} & \cellcolor[HTML]{8cb9e3}{17.09} & \cellcolor[HTML]{8cb9e3}{6.54$^{\square}$} & \cellcolor[HTML]{8cb9e3}{ 4.24} & \cellcolor[HTML]{8cb9e3}{8.61$^{\square}$}\\
\nopagebreak
 & 7 & \cellcolor[HTML]{ffffff}{80.45} & \cellcolor[HTML]{ffffff}{180.00} & \cellcolor[HTML]{ffffff}{279.15$^{\square}$} & \cellcolor[HTML]{8cb9e3}{13.93$^{\square}$} & \cellcolor[HTML]{8cb9e3}{10.25} & \cellcolor[HTML]{8cb9e3}{13.69$^{\square}$} & \cellcolor[HTML]{8cb9e3}{20.71} & \cellcolor[HTML]{8cb9e3}{20.58} & \cellcolor[HTML]{8cb9e3}{20.61} & \cellcolor[HTML]{8cb9e3}{5.21$^{\square}$} & \cellcolor[HTML]{8cb9e3}{ 4.16} & \cellcolor[HTML]{8cb9e3}{4.98$^{\square}$}\\
\nopagebreak
 & 9 & \cellcolor[HTML]{ffffff}{80.28} & \cellcolor[HTML]{ffffff}{180.05} & \cellcolor[HTML]{ffffff}{279.63$^{\square}$} & \cellcolor[HTML]{8cb9e3}{10.42} & \cellcolor[HTML]{8cb9e3}{10.24} & \cellcolor[HTML]{8cb9e3}{10.24} & \cellcolor[HTML]{8cb9e3}{20.91} & \cellcolor[HTML]{8cb9e3}{20.65} & \cellcolor[HTML]{8cb9e3}{20.85} & \cellcolor[HTML]{8cb9e3}{4.74$^{\square}$} & \cellcolor[HTML]{8cb9e3}{ 4.26} & \cellcolor[HTML]{8cb9e3}{4.72$^{\square}$}\\
\nopagebreak
\multirow{-4}{3cm}{\raggedright\arraybackslash Middle-and-extreme spacing} & 11 & \cellcolor[HTML]{ffffff}{80.19} & \cellcolor[HTML]{ffffff}{179.96} & \cellcolor[HTML]{ffffff}{279.86$^{\square}$} & \cellcolor[HTML]{8cb9e3}{10.27} & \cellcolor[HTML]{8cb9e3}{10.28} & \cellcolor[HTML]{8cb9e3}{10.15} & \cellcolor[HTML]{8cb9e3}{20.71} & \cellcolor[HTML]{8cb9e3}{20.70} & \cellcolor[HTML]{8cb9e3}{20.71} & \cellcolor[HTML]{8cb9e3}{4.14} & \cellcolor[HTML]{8cb9e3}{ 4.08} & \cellcolor[HTML]{8cb9e3}{4.16}\\
\bottomrule
\end{longtable}
\end{ThreePartTable}
\addtocounter{table}{-1}
\begin{ThreePartTable}
\begin{TableNotes}
\item \textit{Note. }Cells shaded in light blue indicate cells where estimation is imprecise (i.e., lower and/or upper whisker lengths exceeding 10\% of the parameter's population value. Empty superscript squares ($^{\square}$) indicate biased estimates (i.e., bias exceeding 10\% of parameter's population value). Importantly, bias and precision cutoff values for the days-to-halfway elevation parameter ($\upbeta_{fixed}$) are based on a value of 180.00.
\end{TableNotes}
\begin{longtable}[l]{>{\raggedright\arraybackslash}p{3cm}>{\raggedright\arraybackslash}p{3cm}ccccccccccccccc}
\caption[]{Parameter Values Estimated for Day- and Likert-Unit Parameters in Experiment 1 (continued)}\\
\toprule
\multicolumn{1}{c}{} & \multicolumn{1}{c}{} & \multicolumn{3}{c}{\thead{$\uptheta_{fixed}$ (Baseline)  \\ Pop value = 3.00}} & \multicolumn{3}{c}{\thead{$\uptheta_{random}$ (Baseline) \\ Pop value = 0.05}} & \multicolumn{3}{c}{\thead{$\upalpha_{fixed}$ (Maximal \\ elevation)  \\ Pop value = 3.32}} & \multicolumn{3}{c}{\thead{$\upalpha_{random}$ (Maximal \\ elevation) \\ Pop value = 0.05}} & \multicolumn{3}{c}{\thead{$\upepsilon$(error) \\ Pop value = 0.03}} \\
\cmidrule(l{3pt}r{3pt}){3-5} \cmidrule(l{3pt}r{3pt}){6-8} \cmidrule(l{3pt}r{3pt}){9-11} \cmidrule(l{3pt}r{3pt}){12-14} \cmidrule(l{3pt}r{3pt}){15-17}
Measurement Spacing & Number of Measurements & 80 & 180 & 280 & 80 & 180 & 280 & 80 & 180 & 280 & 80 & 180 & 280 & 80 & 180 & 280\\
\midrule
 & 5 & \cellcolor[HTML]{ffffff}{3.00} & \cellcolor[HTML]{ffffff}{3.00} & \cellcolor[HTML]{ffffff}{3.00} & \cellcolor[HTML]{8cb9e3}{0.05} & \cellcolor[HTML]{8cb9e3}{0.05} & \cellcolor[HTML]{8cb9e3}{0.05} & \cellcolor[HTML]{ffffff}{3.32} & \cellcolor[HTML]{ffffff}{3.32} & \cellcolor[HTML]{ffffff}{3.32} & \cellcolor[HTML]{8cb9e3}{0.05} & \cellcolor[HTML]{8cb9e3}{0.05} & \cellcolor[HTML]{8cb9e3}{0.05} & \cellcolor[HTML]{ffffff}{0.05} & \cellcolor[HTML]{ffffff}{0.05} & \cellcolor[HTML]{ffffff}{0.05}\\
\nopagebreak
 & 7 & \cellcolor[HTML]{ffffff}{3.00} & \cellcolor[HTML]{ffffff}{3.00} & \cellcolor[HTML]{ffffff}{3.00} & \cellcolor[HTML]{8cb9e3}{0.05} & \cellcolor[HTML]{8cb9e3}{0.05} & \cellcolor[HTML]{8cb9e3}{0.05} & \cellcolor[HTML]{ffffff}{3.32} & \cellcolor[HTML]{ffffff}{3.32} & \cellcolor[HTML]{ffffff}{3.32} & \cellcolor[HTML]{8cb9e3}{0.05} & \cellcolor[HTML]{8cb9e3}{0.05} & \cellcolor[HTML]{8cb9e3}{0.05} & \cellcolor[HTML]{ffffff}{0.05} & \cellcolor[HTML]{ffffff}{0.05} & \cellcolor[HTML]{ffffff}{0.05}\\
\nopagebreak
 & 9 & \cellcolor[HTML]{ffffff}{3.00} & \cellcolor[HTML]{ffffff}{3.00} & \cellcolor[HTML]{ffffff}{3.00} & \cellcolor[HTML]{8cb9e3}{0.05} & \cellcolor[HTML]{8cb9e3}{0.05} & \cellcolor[HTML]{8cb9e3}{0.05} & \cellcolor[HTML]{ffffff}{3.32} & \cellcolor[HTML]{ffffff}{3.32} & \cellcolor[HTML]{ffffff}{3.32} & \cellcolor[HTML]{8cb9e3}{0.05} & \cellcolor[HTML]{8cb9e3}{0.05} & \cellcolor[HTML]{8cb9e3}{0.05} & \cellcolor[HTML]{ffffff}{0.05} & \cellcolor[HTML]{ffffff}{0.05} & \cellcolor[HTML]{ffffff}{0.05}\\
\nopagebreak
\multirow{-4}{3cm}{\raggedright\arraybackslash Equal spacing} & 11 & \cellcolor[HTML]{ffffff}{3.00} & \cellcolor[HTML]{ffffff}{3.00} & \cellcolor[HTML]{ffffff}{3.00} & \cellcolor[HTML]{8cb9e3}{0.05} & \cellcolor[HTML]{8cb9e3}{0.05} & \cellcolor[HTML]{8cb9e3}{0.05} & \cellcolor[HTML]{ffffff}{3.32} & \cellcolor[HTML]{ffffff}{3.32} & \cellcolor[HTML]{ffffff}{3.32} & \cellcolor[HTML]{8cb9e3}{0.05} & \cellcolor[HTML]{8cb9e3}{0.05} & \cellcolor[HTML]{8cb9e3}{0.05} & \cellcolor[HTML]{ffffff}{0.05} & \cellcolor[HTML]{ffffff}{0.05} & \cellcolor[HTML]{ffffff}{0.05}\\
\cmidrule{1-17}\pagebreak[0]
 & 5 & \cellcolor[HTML]{ffffff}{3.00} & \cellcolor[HTML]{ffffff}{3.00} & \cellcolor[HTML]{ffffff}{3.00} & \cellcolor[HTML]{8cb9e3}{0.05} & \cellcolor[HTML]{8cb9e3}{0.05} & \cellcolor[HTML]{8cb9e3}{0.05} & \cellcolor[HTML]{ffffff}{3.32} & \cellcolor[HTML]{ffffff}{3.32} & \cellcolor[HTML]{ffffff}{3.33} & \cellcolor[HTML]{8cb9e3}{0.05} & \cellcolor[HTML]{8cb9e3}{0.05} & \cellcolor[HTML]{8cb9e3}{0.05} & \cellcolor[HTML]{ffffff}{0.05} & \cellcolor[HTML]{ffffff}{0.05} & \cellcolor[HTML]{ffffff}{0.05}\\
\nopagebreak
 & 7 & \cellcolor[HTML]{ffffff}{3.00} & \cellcolor[HTML]{ffffff}{3.00} & \cellcolor[HTML]{ffffff}{3.00} & \cellcolor[HTML]{8cb9e3}{0.05} & \cellcolor[HTML]{8cb9e3}{0.05} & \cellcolor[HTML]{8cb9e3}{0.05} & \cellcolor[HTML]{ffffff}{3.32} & \cellcolor[HTML]{ffffff}{3.32} & \cellcolor[HTML]{ffffff}{3.32} & \cellcolor[HTML]{8cb9e3}{0.05} & \cellcolor[HTML]{8cb9e3}{0.05} & \cellcolor[HTML]{8cb9e3}{0.05} & \cellcolor[HTML]{ffffff}{0.05} & \cellcolor[HTML]{ffffff}{0.05} & \cellcolor[HTML]{ffffff}{0.05}\\
\nopagebreak
 & 9 & \cellcolor[HTML]{ffffff}{3.00} & \cellcolor[HTML]{ffffff}{3.00} & \cellcolor[HTML]{ffffff}{3.00} & \cellcolor[HTML]{8cb9e3}{0.05} & \cellcolor[HTML]{8cb9e3}{0.05} & \cellcolor[HTML]{8cb9e3}{0.05} & \cellcolor[HTML]{ffffff}{3.32} & \cellcolor[HTML]{ffffff}{3.32} & \cellcolor[HTML]{ffffff}{3.32} & \cellcolor[HTML]{8cb9e3}{0.05} & \cellcolor[HTML]{8cb9e3}{0.05} & \cellcolor[HTML]{8cb9e3}{0.05} & \cellcolor[HTML]{ffffff}{0.05} & \cellcolor[HTML]{ffffff}{0.05} & \cellcolor[HTML]{ffffff}{0.05}\\
\nopagebreak
\multirow{-4}{3cm}{\raggedright\arraybackslash Time-interval increasing} & 11 & \cellcolor[HTML]{ffffff}{3.00} & \cellcolor[HTML]{ffffff}{3.00} & \cellcolor[HTML]{ffffff}{3.00} & \cellcolor[HTML]{8cb9e3}{0.05} & \cellcolor[HTML]{8cb9e3}{0.05} & \cellcolor[HTML]{8cb9e3}{0.05} & \cellcolor[HTML]{ffffff}{3.32} & \cellcolor[HTML]{ffffff}{3.32} & \cellcolor[HTML]{ffffff}{3.32} & \cellcolor[HTML]{8cb9e3}{0.05} & \cellcolor[HTML]{8cb9e3}{0.05} & \cellcolor[HTML]{8cb9e3}{0.05} & \cellcolor[HTML]{ffffff}{0.05} & \cellcolor[HTML]{ffffff}{0.05} & \cellcolor[HTML]{ffffff}{0.05}\\
\cmidrule{1-17}\pagebreak[0]
 & 5 & \cellcolor[HTML]{ffffff}{2.99} & \cellcolor[HTML]{ffffff}{3.00} & \cellcolor[HTML]{ffffff}{3.00} & \cellcolor[HTML]{8cb9e3}{0.05} & \cellcolor[HTML]{8cb9e3}{0.05} & \cellcolor[HTML]{8cb9e3}{0.05} & \cellcolor[HTML]{ffffff}{3.32} & \cellcolor[HTML]{ffffff}{3.32} & \cellcolor[HTML]{ffffff}{3.32} & \cellcolor[HTML]{8cb9e3}{0.05} & \cellcolor[HTML]{8cb9e3}{0.05} & \cellcolor[HTML]{8cb9e3}{0.05} & \cellcolor[HTML]{ffffff}{0.05} & \cellcolor[HTML]{ffffff}{0.05} & \cellcolor[HTML]{ffffff}{0.05}\\
\nopagebreak
 & 7 & \cellcolor[HTML]{ffffff}{3.00} & \cellcolor[HTML]{ffffff}{3.00} & \cellcolor[HTML]{ffffff}{3.00} & \cellcolor[HTML]{8cb9e3}{0.05} & \cellcolor[HTML]{8cb9e3}{0.05} & \cellcolor[HTML]{8cb9e3}{0.05} & \cellcolor[HTML]{ffffff}{3.32} & \cellcolor[HTML]{ffffff}{3.32} & \cellcolor[HTML]{ffffff}{3.32} & \cellcolor[HTML]{8cb9e3}{0.05} & \cellcolor[HTML]{8cb9e3}{0.05} & \cellcolor[HTML]{8cb9e3}{0.05} & \cellcolor[HTML]{8cb9e3}{0.05} & \cellcolor[HTML]{ffffff}{0.05} & \cellcolor[HTML]{ffffff}{0.05}\\
\nopagebreak
 & 9 & \cellcolor[HTML]{ffffff}{3.00} & \cellcolor[HTML]{ffffff}{3.00} & \cellcolor[HTML]{ffffff}{3.00} & \cellcolor[HTML]{8cb9e3}{0.05} & \cellcolor[HTML]{8cb9e3}{0.05} & \cellcolor[HTML]{8cb9e3}{0.05} & \cellcolor[HTML]{ffffff}{3.32} & \cellcolor[HTML]{ffffff}{3.32} & \cellcolor[HTML]{ffffff}{3.32} & \cellcolor[HTML]{8cb9e3}{0.05} & \cellcolor[HTML]{8cb9e3}{0.05} & \cellcolor[HTML]{8cb9e3}{0.05} & \cellcolor[HTML]{ffffff}{0.05} & \cellcolor[HTML]{ffffff}{0.05} & \cellcolor[HTML]{ffffff}{0.05}\\
\nopagebreak
\multirow{-4}{3cm}{\raggedright\arraybackslash Time-interval decreasing} & 11 & \cellcolor[HTML]{ffffff}{3.00} & \cellcolor[HTML]{ffffff}{3.00} & \cellcolor[HTML]{ffffff}{3.00} & \cellcolor[HTML]{8cb9e3}{0.05} & \cellcolor[HTML]{8cb9e3}{0.05} & \cellcolor[HTML]{8cb9e3}{0.05} & \cellcolor[HTML]{ffffff}{3.32} & \cellcolor[HTML]{ffffff}{3.32} & \cellcolor[HTML]{ffffff}{3.32} & \cellcolor[HTML]{8cb9e3}{0.05} & \cellcolor[HTML]{8cb9e3}{0.05} & \cellcolor[HTML]{8cb9e3}{0.05} & \cellcolor[HTML]{ffffff}{0.05} & \cellcolor[HTML]{ffffff}{0.05} & \cellcolor[HTML]{ffffff}{0.05}\\
\cmidrule{1-17}\pagebreak[0]
 & 5 & \cellcolor[HTML]{ffffff}{2.99} & \cellcolor[HTML]{ffffff}{3.00} & \cellcolor[HTML]{ffffff}{3.00} & \cellcolor[HTML]{8cb9e3}{0.05} & \cellcolor[HTML]{8cb9e3}{0.05} & \cellcolor[HTML]{8cb9e3}{0.05} & \cellcolor[HTML]{ffffff}{3.32} & \cellcolor[HTML]{ffffff}{3.32} & \cellcolor[HTML]{ffffff}{3.33} & \cellcolor[HTML]{8cb9e3}{0.05} & \cellcolor[HTML]{8cb9e3}{0.05} & \cellcolor[HTML]{8cb9e3}{0.05} & \cellcolor[HTML]{ffffff}{0.05} & \cellcolor[HTML]{ffffff}{0.05} & \cellcolor[HTML]{ffffff}{0.05}\\
\nopagebreak
 & 7 & \cellcolor[HTML]{ffffff}{3.00} & \cellcolor[HTML]{ffffff}{3.00} & \cellcolor[HTML]{ffffff}{3.00} & \cellcolor[HTML]{8cb9e3}{0.05} & \cellcolor[HTML]{8cb9e3}{0.05} & \cellcolor[HTML]{8cb9e3}{0.05} & \cellcolor[HTML]{ffffff}{3.32} & \cellcolor[HTML]{ffffff}{3.32} & \cellcolor[HTML]{ffffff}{3.32} & \cellcolor[HTML]{8cb9e3}{0.05} & \cellcolor[HTML]{8cb9e3}{0.05} & \cellcolor[HTML]{8cb9e3}{0.05} & \cellcolor[HTML]{ffffff}{0.05} & \cellcolor[HTML]{ffffff}{0.05} & \cellcolor[HTML]{ffffff}{0.05}\\
\nopagebreak
 & 9 & \cellcolor[HTML]{ffffff}{3.00} & \cellcolor[HTML]{ffffff}{3.00} & \cellcolor[HTML]{ffffff}{3.00} & \cellcolor[HTML]{8cb9e3}{0.05} & \cellcolor[HTML]{8cb9e3}{0.05} & \cellcolor[HTML]{8cb9e3}{0.05} & \cellcolor[HTML]{ffffff}{3.32} & \cellcolor[HTML]{ffffff}{3.32} & \cellcolor[HTML]{ffffff}{3.32} & \cellcolor[HTML]{8cb9e3}{0.05} & \cellcolor[HTML]{8cb9e3}{0.05} & \cellcolor[HTML]{8cb9e3}{0.05} & \cellcolor[HTML]{ffffff}{0.05} & \cellcolor[HTML]{ffffff}{0.05} & \cellcolor[HTML]{ffffff}{0.05}\\
\nopagebreak
\multirow{-4}{3cm}{\raggedright\arraybackslash Middle-and-extreme spacing} & 11 & \cellcolor[HTML]{ffffff}{3.00} & \cellcolor[HTML]{ffffff}{3.00} & \cellcolor[HTML]{ffffff}{3.00} & \cellcolor[HTML]{8cb9e3}{0.05} & \cellcolor[HTML]{8cb9e3}{0.05} & \cellcolor[HTML]{8cb9e3}{0.05} & \cellcolor[HTML]{ffffff}{3.32} & \cellcolor[HTML]{ffffff}{3.32} & \cellcolor[HTML]{ffffff}{3.32} & \cellcolor[HTML]{8cb9e3}{0.05} & \cellcolor[HTML]{8cb9e3}{0.05} & \cellcolor[HTML]{8cb9e3}{0.05} & \cellcolor[HTML]{ffffff}{0.05} & \cellcolor[HTML]{ffffff}{0.05} & \cellcolor[HTML]{ffffff}{0.05}\\
\bottomrule
\insertTableNotes
\end{longtable}
\end{ThreePartTable}
\secapp{Experiment 2}

\end{landscape}
\restoregeometry

\newgeometry{margin=2.54cm}
\begin{landscape}
\begin{ThreePartTable}
\begin{TableNotes}
\item 
\end{TableNotes}
\begin{longtable}[l]{>{\raggedright\arraybackslash}p{3cm}>{\raggedright\arraybackslash}p{3cm}cccccccccccc}
\caption{Parameter Values Estimated in Experiment 2}\label{tab:param-exp-2}\\
\toprule
\multicolumn{1}{c}{} & \multicolumn{1}{c}{} & \multicolumn{6}{c}{\thead{$\upbeta_{fixed}$ (Days to halfway elevation) \\ Pop value = 180.00}} & \multicolumn{6}{c}{\thead{$\upbeta_{random}$ (Days to halfway elevation) \\ Pop value = 10.00}} \\
\cmidrule(l{3pt}r{3pt}){3-8} \cmidrule(l{3pt}r{3pt}){9-14}
Measurement Spacing & Number of Measurements & 30 & 50 & 100 & 200 & 500 & 1000 & 30 & 50 & 100 & 200 & 500 & 1000\\
\midrule
 & 5 & \cellcolor[HTML]{ffffff}{179.71} & \cellcolor[HTML]{ffffff}{179.82} & \cellcolor[HTML]{ffffff}{179.53} & \cellcolor[HTML]{ffffff}{180.00} & \cellcolor[HTML]{ffffff}{179.99} & \cellcolor[HTML]{ffffff}{179.64} & \cellcolor[HTML]{8cb9e3}{10.40} & \cellcolor[HTML]{8cb9e3}{10.36} & \cellcolor[HTML]{8cb9e3}{10.04} & \cellcolor[HTML]{8cb9e3}{10.51} & \cellcolor[HTML]{8cb9e3}{10.65} & \cellcolor[HTML]{8cb9e3}{10.74}\\
\nopagebreak
 & 7 & \cellcolor[HTML]{ffffff}{180.05} & \cellcolor[HTML]{ffffff}{179.65} & \cellcolor[HTML]{ffffff}{179.53} & \cellcolor[HTML]{ffffff}{179.75} & \cellcolor[HTML]{ffffff}{179.76} & \cellcolor[HTML]{ffffff}{179.99} & \cellcolor[HTML]{8cb9e3}{10.18} & \cellcolor[HTML]{8cb9e3}{10.59} & \cellcolor[HTML]{8cb9e3}{10.49} & \cellcolor[HTML]{8cb9e3}{10.54} & \cellcolor[HTML]{8cb9e3}{10.60} & \cellcolor[HTML]{8cb9e3}{10.58}\\
\nopagebreak
 & 9 & \cellcolor[HTML]{ffffff}{179.84} & \cellcolor[HTML]{ffffff}{180.07} & \cellcolor[HTML]{ffffff}{179.94} & \cellcolor[HTML]{ffffff}{180.00} & \cellcolor[HTML]{ffffff}{180.02} & \cellcolor[HTML]{ffffff}{180.03} & \cellcolor[HTML]{8cb9e3}{10.28} & \cellcolor[HTML]{8cb9e3}{10.20} & \cellcolor[HTML]{8cb9e3}{10.30} & \cellcolor[HTML]{8cb9e3}{10.40} & \cellcolor[HTML]{8cb9e3}{10.39} & \cellcolor[HTML]{8cb9e3}{10.36}\\
\nopagebreak
\multirow{-4}{3cm}{\raggedright\arraybackslash Equal spacing} & 11 & \cellcolor[HTML]{ffffff}{180.11} & \cellcolor[HTML]{ffffff}{180.11} & \cellcolor[HTML]{ffffff}{180.01} & \cellcolor[HTML]{ffffff}{180.03} & \cellcolor[HTML]{ffffff}{179.98} & \cellcolor[HTML]{ffffff}{179.98} & \cellcolor[HTML]{8cb9e3}{10.08} & \cellcolor[HTML]{8cb9e3}{10.04} & \cellcolor[HTML]{8cb9e3}{10.28} & \cellcolor[HTML]{8cb9e3}{10.29} & \cellcolor[HTML]{8cb9e3}{10.38} & \cellcolor[HTML]{8cb9e3}{10.29}\\
\cmidrule{1-14}\pagebreak[0]
 & 5 & \cellcolor[HTML]{8cb9e3}{181.81} & \cellcolor[HTML]{8cb9e3}{181.16} & \cellcolor[HTML]{8cb9e3}{181.14} & \cellcolor[HTML]{8cb9e3}{180.27} & \cellcolor[HTML]{ffffff}{179.78} & \cellcolor[HTML]{ffffff}{179.57} & \cellcolor[HTML]{8cb9e3}{11.24$^{\square}$} & \cellcolor[HTML]{8cb9e3}{10.24} & \cellcolor[HTML]{8cb9e3}{ 9.93} & \cellcolor[HTML]{8cb9e3}{ 9.59} & \cellcolor[HTML]{8cb9e3}{ 9.91} & \cellcolor[HTML]{8cb9e3}{10.22}\\
\nopagebreak
 & 7 & \cellcolor[HTML]{ffffff}{179.99} & \cellcolor[HTML]{ffffff}{179.96} & \cellcolor[HTML]{ffffff}{179.73} & \cellcolor[HTML]{ffffff}{179.77} & \cellcolor[HTML]{ffffff}{179.79} & \cellcolor[HTML]{ffffff}{179.83} & \cellcolor[HTML]{8cb9e3}{10.26} & \cellcolor[HTML]{8cb9e3}{10.43} & \cellcolor[HTML]{8cb9e3}{10.50} & \cellcolor[HTML]{8cb9e3}{10.43} & \cellcolor[HTML]{8cb9e3}{10.47} & \cellcolor[HTML]{8cb9e3}{10.47}\\
\nopagebreak
 & 9 & \cellcolor[HTML]{ffffff}{179.33} & \cellcolor[HTML]{ffffff}{179.18} & \cellcolor[HTML]{ffffff}{178.99} & \cellcolor[HTML]{ffffff}{179.07} & \cellcolor[HTML]{ffffff}{179.11} & \cellcolor[HTML]{ffffff}{179.13} & \cellcolor[HTML]{8cb9e3}{10.15} & \cellcolor[HTML]{8cb9e3}{10.10} & \cellcolor[HTML]{8cb9e3}{10.17} & \cellcolor[HTML]{8cb9e3}{10.18} & \cellcolor[HTML]{8cb9e3}{10.21} & \cellcolor[HTML]{8cb9e3}{10.29}\\
\nopagebreak
\multirow{-4}{3cm}{\raggedright\arraybackslash Time-interval increasing} & 11 & \cellcolor[HTML]{ffffff}{179.81} & \cellcolor[HTML]{ffffff}{179.79} & \cellcolor[HTML]{ffffff}{179.86} & \cellcolor[HTML]{ffffff}{179.88} & \cellcolor[HTML]{ffffff}{179.81} & \cellcolor[HTML]{ffffff}{179.82} & \cellcolor[HTML]{8cb9e3}{ 9.99} & \cellcolor[HTML]{8cb9e3}{10.19} & \cellcolor[HTML]{8cb9e3}{10.32} & \cellcolor[HTML]{8cb9e3}{10.27} & \cellcolor[HTML]{8cb9e3}{10.30} & \cellcolor[HTML]{8cb9e3}{10.30}\\
\cmidrule{1-14}\pagebreak[0]
 & 5 & \cellcolor[HTML]{8cb9e3}{177.01} & \cellcolor[HTML]{8cb9e3}{178.48} & \cellcolor[HTML]{8cb9e3}{179.13} & \cellcolor[HTML]{8cb9e3}{179.23} & \cellcolor[HTML]{8cb9e3}{179.86} & \cellcolor[HTML]{ffffff}{180.37} & \cellcolor[HTML]{8cb9e3}{10.95} & \cellcolor[HTML]{8cb9e3}{11.38$^{\square}$} & \cellcolor[HTML]{8cb9e3}{ 9.97} & \cellcolor[HTML]{8cb9e3}{ 9.55} & \cellcolor[HTML]{8cb9e3}{10.36} & \cellcolor[HTML]{8cb9e3}{10.11}\\
\nopagebreak
 & 7 & \cellcolor[HTML]{ffffff}{178.98} & \cellcolor[HTML]{ffffff}{179.68} & \cellcolor[HTML]{ffffff}{179.12} & \cellcolor[HTML]{ffffff}{179.53} & \cellcolor[HTML]{ffffff}{180.07} & \cellcolor[HTML]{ffffff}{179.75} & \cellcolor[HTML]{8cb9e3}{10.07} & \cellcolor[HTML]{8cb9e3}{10.31} & \cellcolor[HTML]{8cb9e3}{10.48} & \cellcolor[HTML]{8cb9e3}{10.37} & \cellcolor[HTML]{8cb9e3}{10.46} & \cellcolor[HTML]{8cb9e3}{10.51}\\
\nopagebreak
 & 9 & \cellcolor[HTML]{ffffff}{179.65} & \cellcolor[HTML]{ffffff}{179.01} & \cellcolor[HTML]{ffffff}{178.46} & \cellcolor[HTML]{ffffff}{179.47} & \cellcolor[HTML]{ffffff}{179.64} & \cellcolor[HTML]{ffffff}{179.75} & \cellcolor[HTML]{8cb9e3}{10.11} & \cellcolor[HTML]{8cb9e3}{10.16} & \cellcolor[HTML]{8cb9e3}{10.20} & \cellcolor[HTML]{8cb9e3}{10.17} & \cellcolor[HTML]{8cb9e3}{10.28} & \cellcolor[HTML]{8cb9e3}{10.26}\\
\nopagebreak
\multirow{-4}{3cm}{\raggedright\arraybackslash Time-interval decreasing} & 11 & \cellcolor[HTML]{ffffff}{179.48} & \cellcolor[HTML]{ffffff}{179.68} & \cellcolor[HTML]{ffffff}{179.70} & \cellcolor[HTML]{ffffff}{179.65} & \cellcolor[HTML]{ffffff}{179.64} & \cellcolor[HTML]{ffffff}{179.68} & \cellcolor[HTML]{8cb9e3}{ 9.85} & \cellcolor[HTML]{8cb9e3}{ 9.98} & \cellcolor[HTML]{8cb9e3}{10.03} & \cellcolor[HTML]{8cb9e3}{10.12} & \cellcolor[HTML]{8cb9e3}{10.13} & \cellcolor[HTML]{8cb9e3}{10.11}\\
\cmidrule{1-14}\pagebreak[0]
 & 5 & \cellcolor[HTML]{ffffff}{177.99} & \cellcolor[HTML]{ffffff}{179.65} & \cellcolor[HTML]{ffffff}{179.15} & \cellcolor[HTML]{ffffff}{179.83} & \cellcolor[HTML]{ffffff}{179.61} & \cellcolor[HTML]{ffffff}{178.74} & \cellcolor[HTML]{8cb9e3}{10.30} & \cellcolor[HTML]{8cb9e3}{10.24} & \cellcolor[HTML]{8cb9e3}{10.40} & \cellcolor[HTML]{8cb9e3}{10.24} & \cellcolor[HTML]{8cb9e3}{10.28} & \cellcolor[HTML]{8cb9e3}{10.26}\\
\nopagebreak
 & 7 & \cellcolor[HTML]{ffffff}{179.96} & \cellcolor[HTML]{ffffff}{179.82} & \cellcolor[HTML]{ffffff}{179.97} & \cellcolor[HTML]{ffffff}{179.98} & \cellcolor[HTML]{ffffff}{180.02} & \cellcolor[HTML]{ffffff}{179.98} & \cellcolor[HTML]{8cb9e3}{10.25} & \cellcolor[HTML]{8cb9e3}{10.20} & \cellcolor[HTML]{8cb9e3}{10.32} & \cellcolor[HTML]{8cb9e3}{10.26} & \cellcolor[HTML]{8cb9e3}{10.29} & \cellcolor[HTML]{8cb9e3}{10.27}\\
\nopagebreak
 & 9 & \cellcolor[HTML]{ffffff}{179.88} & \cellcolor[HTML]{ffffff}{180.07} & \cellcolor[HTML]{ffffff}{179.89} & \cellcolor[HTML]{ffffff}{179.98} & \cellcolor[HTML]{ffffff}{179.98} & \cellcolor[HTML]{ffffff}{179.99} & \cellcolor[HTML]{8cb9e3}{10.12} & \cellcolor[HTML]{8cb9e3}{10.16} & \cellcolor[HTML]{8cb9e3}{10.24} & \cellcolor[HTML]{8cb9e3}{10.30} & \cellcolor[HTML]{8cb9e3}{10.24} & \cellcolor[HTML]{8cb9e3}{10.29}\\
\nopagebreak
\multirow{-4}{3cm}{\raggedright\arraybackslash Middle-and-extreme spacing} & 11 & \cellcolor[HTML]{ffffff}{180.02} & \cellcolor[HTML]{ffffff}{179.96} & \cellcolor[HTML]{ffffff}{180.01} & \cellcolor[HTML]{ffffff}{179.98} & \cellcolor[HTML]{ffffff}{180.01} & \cellcolor[HTML]{ffffff}{179.99} & \cellcolor[HTML]{8cb9e3}{10.08} & \cellcolor[HTML]{8cb9e3}{10.35} & \cellcolor[HTML]{8cb9e3}{10.15} & \cellcolor[HTML]{8cb9e3}{10.35} & \cellcolor[HTML]{8cb9e3}{10.30} & \cellcolor[HTML]{8cb9e3}{10.28}\\
\bottomrule
\end{longtable}
\end{ThreePartTable}
\addtocounter{table}{-1}
\begin{ThreePartTable}
\begin{TableNotes}
\item 
\end{TableNotes}
\begin{longtable}[l]{>{\raggedright\arraybackslash}p{3cm}>{\raggedright\arraybackslash}p{3cm}cccccccccccc}
\caption[]{Parameter Values Estimated for Day- and Likert-Unit Parameters in Experiment 2 (continued)}\\
\toprule
\multicolumn{1}{c}{} & \multicolumn{1}{c}{} & \multicolumn{6}{c}{\thead{$\upgamma_{fixed}$ (Triquarter-halfway delta) \\ Pop value = 20.00}} & \multicolumn{6}{c}{\thead{$\upgamma_{random}$ (Triquarter-halfway delta) \\ Pop value = 4.00}} \\
\cmidrule(l{3pt}r{3pt}){3-8} \cmidrule(l{3pt}r{3pt}){9-14}
Measurement Spacing & Number of Measurements & 30 & 50 & 100 & 200 & 500 & 1000 & 30 & 50 & 100 & 200 & 500 & 1000\\
\midrule
 & 5 & \cellcolor[HTML]{8cb9e3}{18.25} & \cellcolor[HTML]{8cb9e3}{18.11} & \cellcolor[HTML]{8cb9e3}{18.27} & \cellcolor[HTML]{8cb9e3}{19.59} & \cellcolor[HTML]{8cb9e3}{20.27} & \cellcolor[HTML]{8cb9e3}{20.60} & \cellcolor[HTML]{8cb9e3}{17.69$^{\square}$} & \cellcolor[HTML]{8cb9e3}{16.95$^{\square}$} & \cellcolor[HTML]{8cb9e3}{16.41$^{\square}$} & \cellcolor[HTML]{8cb9e3}{15.19$^{\square}$} & \cellcolor[HTML]{8cb9e3}{12.19$^{\square}$} & \cellcolor[HTML]{8cb9e3}{8.51$^{\square}$}\\
\nopagebreak
 & 7 & \cellcolor[HTML]{8cb9e3}{20.25} & \cellcolor[HTML]{8cb9e3}{20.53} & \cellcolor[HTML]{8cb9e3}{20.66} & \cellcolor[HTML]{8cb9e3}{20.75} & \cellcolor[HTML]{8cb9e3}{20.81} & \cellcolor[HTML]{8cb9e3}{20.74} & \cellcolor[HTML]{8cb9e3}{ 9.22$^{\square}$} & \cellcolor[HTML]{8cb9e3}{ 7.70$^{\square}$} & \cellcolor[HTML]{8cb9e3}{ 5.77$^{\square}$} & \cellcolor[HTML]{8cb9e3}{ 4.89$^{\square}$} & \cellcolor[HTML]{8cb9e3}{ 4.98$^{\square}$} & \cellcolor[HTML]{8cb9e3}{4.34}\\
\nopagebreak
 & 9 & \cellcolor[HTML]{8cb9e3}{20.88} & \cellcolor[HTML]{8cb9e3}{20.72} & \cellcolor[HTML]{8cb9e3}{20.73} & \cellcolor[HTML]{8cb9e3}{20.76} & \cellcolor[HTML]{ffffff}{20.75} & \cellcolor[HTML]{ffffff}{20.73} & \cellcolor[HTML]{8cb9e3}{ 5.30$^{\square}$} & \cellcolor[HTML]{8cb9e3}{ 4.99$^{\square}$} & \cellcolor[HTML]{8cb9e3}{ 4.44$^{\square}$} & \cellcolor[HTML]{8cb9e3}{ 4.27} & \cellcolor[HTML]{8cb9e3}{ 4.03} & \cellcolor[HTML]{8cb9e3}{4.00}\\
\nopagebreak
\multirow{-4}{3cm}{\raggedright\arraybackslash Equal spacing} & 11 & \cellcolor[HTML]{8cb9e3}{20.65} & \cellcolor[HTML]{8cb9e3}{20.66} & \cellcolor[HTML]{8cb9e3}{20.73} & \cellcolor[HTML]{8cb9e3}{20.70} & \cellcolor[HTML]{ffffff}{20.69} & \cellcolor[HTML]{ffffff}{20.71} & \cellcolor[HTML]{8cb9e3}{ 4.86$^{\square}$} & \cellcolor[HTML]{8cb9e3}{ 4.49$^{\square}$} & \cellcolor[HTML]{8cb9e3}{ 4.20} & \cellcolor[HTML]{8cb9e3}{ 4.10} & \cellcolor[HTML]{8cb9e3}{ 4.02} & \cellcolor[HTML]{8cb9e3}{4.07}\\
\cmidrule{1-14}\pagebreak[0]
 & 5 & \cellcolor[HTML]{8cb9e3}{18.81} & \cellcolor[HTML]{8cb9e3}{19.11} & \cellcolor[HTML]{8cb9e3}{19.56} & \cellcolor[HTML]{8cb9e3}{20.25} & \cellcolor[HTML]{8cb9e3}{20.80} & \cellcolor[HTML]{8cb9e3}{20.92} & \cellcolor[HTML]{8cb9e3}{ 6.18$^{\square}$} & \cellcolor[HTML]{8cb9e3}{ 5.88$^{\square}$} & \cellcolor[HTML]{8cb9e3}{ 5.25$^{\square}$} & \cellcolor[HTML]{8cb9e3}{ 4.94$^{\square}$} & \cellcolor[HTML]{8cb9e3}{ 4.68$^{\square}$} & \cellcolor[HTML]{8cb9e3}{4.42$^{\square}$}\\
\nopagebreak
 & 7 & \cellcolor[HTML]{8cb9e3}{20.74} & \cellcolor[HTML]{8cb9e3}{20.74} & \cellcolor[HTML]{8cb9e3}{20.94} & \cellcolor[HTML]{8cb9e3}{20.83} & \cellcolor[HTML]{8cb9e3}{20.83} & \cellcolor[HTML]{ffffff}{20.82} & \cellcolor[HTML]{8cb9e3}{ 7.38$^{\square}$} & \cellcolor[HTML]{8cb9e3}{ 6.31$^{\square}$} & \cellcolor[HTML]{8cb9e3}{ 5.45$^{\square}$} & \cellcolor[HTML]{8cb9e3}{ 5.06$^{\square}$} & \cellcolor[HTML]{8cb9e3}{ 4.66$^{\square}$} & \cellcolor[HTML]{8cb9e3}{4.45$^{\square}$}\\
\nopagebreak
 & 9 & \cellcolor[HTML]{8cb9e3}{20.72} & \cellcolor[HTML]{8cb9e3}{20.65} & \cellcolor[HTML]{8cb9e3}{20.69} & \cellcolor[HTML]{8cb9e3}{20.65} & \cellcolor[HTML]{ffffff}{20.63} & \cellcolor[HTML]{ffffff}{20.65} & \cellcolor[HTML]{8cb9e3}{ 5.15$^{\square}$} & \cellcolor[HTML]{8cb9e3}{ 4.83$^{\square}$} & \cellcolor[HTML]{8cb9e3}{ 4.44$^{\square}$} & \cellcolor[HTML]{8cb9e3}{ 4.26} & \cellcolor[HTML]{8cb9e3}{ 4.16} & \cellcolor[HTML]{8cb9e3}{4.23}\\
\nopagebreak
\multirow{-4}{3cm}{\raggedright\arraybackslash Time-interval increasing} & 11 & \cellcolor[HTML]{8cb9e3}{20.80} & \cellcolor[HTML]{8cb9e3}{20.69} & \cellcolor[HTML]{8cb9e3}{20.84} & \cellcolor[HTML]{8cb9e3}{20.76} & \cellcolor[HTML]{ffffff}{20.78} & \cellcolor[HTML]{ffffff}{20.76} & \cellcolor[HTML]{8cb9e3}{ 4.84$^{\square}$} & \cellcolor[HTML]{8cb9e3}{ 4.43$^{\square}$} & \cellcolor[HTML]{8cb9e3}{ 4.25} & \cellcolor[HTML]{8cb9e3}{ 4.26} & \cellcolor[HTML]{8cb9e3}{ 4.17} & \cellcolor[HTML]{8cb9e3}{4.14}\\
\cmidrule{1-14}\pagebreak[0]
 & 5 & \cellcolor[HTML]{8cb9e3}{19.21} & \cellcolor[HTML]{8cb9e3}{18.50} & \cellcolor[HTML]{8cb9e3}{19.21} & \cellcolor[HTML]{8cb9e3}{19.90} & \cellcolor[HTML]{8cb9e3}{20.50} & \cellcolor[HTML]{8cb9e3}{20.79} & \cellcolor[HTML]{8cb9e3}{ 7.17$^{\square}$} & \cellcolor[HTML]{8cb9e3}{ 6.01$^{\square}$} & \cellcolor[HTML]{8cb9e3}{ 5.18$^{\square}$} & \cellcolor[HTML]{8cb9e3}{ 5.12$^{\square}$} & \cellcolor[HTML]{8cb9e3}{ 4.91$^{\square}$} & \cellcolor[HTML]{8cb9e3}{4.66$^{\square}$}\\
\nopagebreak
 & 7 & \cellcolor[HTML]{8cb9e3}{20.36} & \cellcolor[HTML]{8cb9e3}{20.49} & \cellcolor[HTML]{8cb9e3}{20.57} & \cellcolor[HTML]{8cb9e3}{20.69} & \cellcolor[HTML]{8cb9e3}{21.03} & \cellcolor[HTML]{ffffff}{20.76} & \cellcolor[HTML]{8cb9e3}{ 6.98$^{\square}$} & \cellcolor[HTML]{8cb9e3}{ 6.18$^{\square}$} & \cellcolor[HTML]{8cb9e3}{ 5.43$^{\square}$} & \cellcolor[HTML]{8cb9e3}{ 5.20$^{\square}$} & \cellcolor[HTML]{8cb9e3}{ 4.67$^{\square}$} & \cellcolor[HTML]{8cb9e3}{4.68$^{\square}$}\\
\nopagebreak
 & 9 & \cellcolor[HTML]{8cb9e3}{20.69} & \cellcolor[HTML]{8cb9e3}{20.60} & \cellcolor[HTML]{8cb9e3}{20.55} & \cellcolor[HTML]{8cb9e3}{20.62} & \cellcolor[HTML]{ffffff}{20.70} & \cellcolor[HTML]{ffffff}{20.63} & \cellcolor[HTML]{8cb9e3}{ 5.48$^{\square}$} & \cellcolor[HTML]{8cb9e3}{ 5.12$^{\square}$} & \cellcolor[HTML]{8cb9e3}{ 4.72$^{\square}$} & \cellcolor[HTML]{8cb9e3}{ 4.52$^{\square}$} & \cellcolor[HTML]{8cb9e3}{ 4.72$^{\square}$} & \cellcolor[HTML]{8cb9e3}{4.83$^{\square}$}\\
\nopagebreak
\multirow{-4}{3cm}{\raggedright\arraybackslash Time-interval decreasing} & 11 & \cellcolor[HTML]{8cb9e3}{20.49} & \cellcolor[HTML]{8cb9e3}{20.53} & \cellcolor[HTML]{8cb9e3}{20.38} & \cellcolor[HTML]{8cb9e3}{20.41} & \cellcolor[HTML]{ffffff}{20.47} & \cellcolor[HTML]{ffffff}{20.41} & \cellcolor[HTML]{8cb9e3}{ 4.66$^{\square}$} & \cellcolor[HTML]{8cb9e3}{ 4.57$^{\square}$} & \cellcolor[HTML]{8cb9e3}{ 4.34} & \cellcolor[HTML]{8cb9e3}{ 4.20} & \cellcolor[HTML]{8cb9e3}{ 4.18} & \cellcolor[HTML]{8cb9e3}{4.17}\\
\cmidrule{1-14}\pagebreak[0]
 & 5 & \cellcolor[HTML]{8cb9e3}{20.80} & \cellcolor[HTML]{8cb9e3}{20.69} & \cellcolor[HTML]{8cb9e3}{20.65} & \cellcolor[HTML]{8cb9e3}{20.67} & \cellcolor[HTML]{ffffff}{20.64} & \cellcolor[HTML]{ffffff}{20.59} & \cellcolor[HTML]{8cb9e3}{ 5.21$^{\square}$} & \cellcolor[HTML]{8cb9e3}{ 4.68$^{\square}$} & \cellcolor[HTML]{8cb9e3}{ 4.43$^{\square}$} & \cellcolor[HTML]{8cb9e3}{ 4.18} & \cellcolor[HTML]{8cb9e3}{ 4.15} & \cellcolor[HTML]{8cb9e3}{4.11}\\
\nopagebreak
 & 7 & \cellcolor[HTML]{8cb9e3}{20.76} & \cellcolor[HTML]{8cb9e3}{20.55} & \cellcolor[HTML]{8cb9e3}{20.70} & \cellcolor[HTML]{8cb9e3}{20.63} & \cellcolor[HTML]{ffffff}{20.60} & \cellcolor[HTML]{ffffff}{20.63} & \cellcolor[HTML]{8cb9e3}{ 5.07$^{\square}$} & \cellcolor[HTML]{8cb9e3}{ 4.60$^{\square}$} & \cellcolor[HTML]{8cb9e3}{ 4.39} & \cellcolor[HTML]{8cb9e3}{ 4.23} & \cellcolor[HTML]{8cb9e3}{ 4.19} & \cellcolor[HTML]{8cb9e3}{4.15}\\
\nopagebreak
 & 9 & \cellcolor[HTML]{8cb9e3}{20.68} & \cellcolor[HTML]{8cb9e3}{20.71} & \cellcolor[HTML]{8cb9e3}{20.67} & \cellcolor[HTML]{8cb9e3}{20.63} & \cellcolor[HTML]{ffffff}{20.58} & \cellcolor[HTML]{ffffff}{20.63} & \cellcolor[HTML]{8cb9e3}{ 4.99$^{\square}$} & \cellcolor[HTML]{8cb9e3}{ 4.67$^{\square}$} & \cellcolor[HTML]{8cb9e3}{ 4.49$^{\square}$} & \cellcolor[HTML]{8cb9e3}{ 4.17} & \cellcolor[HTML]{8cb9e3}{ 4.13} & \cellcolor[HTML]{8cb9e3}{4.15}\\
\nopagebreak
\multirow{-4}{3cm}{\raggedright\arraybackslash Middle-and-extreme spacing} & 11 & \cellcolor[HTML]{8cb9e3}{20.64} & \cellcolor[HTML]{8cb9e3}{20.74} & \cellcolor[HTML]{8cb9e3}{20.67} & \cellcolor[HTML]{8cb9e3}{20.70} & \cellcolor[HTML]{ffffff}{20.66} & \cellcolor[HTML]{ffffff}{20.68} & \cellcolor[HTML]{8cb9e3}{ 4.57$^{\square}$} & \cellcolor[HTML]{8cb9e3}{ 4.47$^{\square}$} & \cellcolor[HTML]{8cb9e3}{ 4.22} & \cellcolor[HTML]{8cb9e3}{ 4.19} & \cellcolor[HTML]{8cb9e3}{ 4.09} & \cellcolor[HTML]{8cb9e3}{4.07}\\
\bottomrule
\end{longtable}
\end{ThreePartTable}
\addtocounter{table}{-1}
\begin{ThreePartTable}
\begin{TableNotes}
\item 
\end{TableNotes}
\begin{longtable}[l]{>{\raggedright\arraybackslash}p{3cm}>{\raggedright\arraybackslash}p{3cm}cccccccccccc}
\caption[]{Parameter Values Estimated for Day- and Likert-Unit Parameters in Experiment 2 (continued)}\\
\toprule
\multicolumn{1}{c}{} & \multicolumn{1}{c}{} & \multicolumn{6}{c}{\thead{$\uptheta_{fixed}$ (Baseline) \\ Pop value = 3.00}} & \multicolumn{6}{c}{\thead{$\uptheta_{random}$ (Baseline) \\ Pop value = 0.05}} \\
\cmidrule(l{3pt}r{3pt}){3-8} \cmidrule(l{3pt}r{3pt}){9-14}
Measurement Spacing & Number of Measurements & 30 & 50 & 100 & 200 & 500 & 1000 & 30 & 50 & 100 & 200 & 500 & 1000\\
\midrule
 & 5 & \cellcolor[HTML]{ffffff}{3.00} & \cellcolor[HTML]{ffffff}{3.00} & \cellcolor[HTML]{ffffff}{3.00} & \cellcolor[HTML]{ffffff}{3.00} & \cellcolor[HTML]{ffffff}{3.00} & \cellcolor[HTML]{ffffff}{3.00} & \cellcolor[HTML]{8cb9e3}{0.05} & \cellcolor[HTML]{8cb9e3}{0.05} & \cellcolor[HTML]{8cb9e3}{0.05} & \cellcolor[HTML]{8cb9e3}{0.05} & \cellcolor[HTML]{ffffff}{0.05} & \cellcolor[HTML]{ffffff}{0.05}\\
\nopagebreak
 & 7 & \cellcolor[HTML]{ffffff}{3.00} & \cellcolor[HTML]{ffffff}{3.00} & \cellcolor[HTML]{ffffff}{3.00} & \cellcolor[HTML]{ffffff}{3.00} & \cellcolor[HTML]{ffffff}{3.00} & \cellcolor[HTML]{ffffff}{3.00} & \cellcolor[HTML]{8cb9e3}{0.05} & \cellcolor[HTML]{8cb9e3}{0.05} & \cellcolor[HTML]{8cb9e3}{0.05} & \cellcolor[HTML]{8cb9e3}{0.05} & \cellcolor[HTML]{ffffff}{0.05} & \cellcolor[HTML]{ffffff}{0.05}\\
\nopagebreak
 & 9 & \cellcolor[HTML]{ffffff}{3.00} & \cellcolor[HTML]{ffffff}{3.00} & \cellcolor[HTML]{ffffff}{3.00} & \cellcolor[HTML]{ffffff}{3.00} & \cellcolor[HTML]{ffffff}{3.00} & \cellcolor[HTML]{ffffff}{3.00} & \cellcolor[HTML]{8cb9e3}{0.05} & \cellcolor[HTML]{8cb9e3}{0.05} & \cellcolor[HTML]{8cb9e3}{0.05} & \cellcolor[HTML]{8cb9e3}{0.05} & \cellcolor[HTML]{ffffff}{0.05} & \cellcolor[HTML]{ffffff}{0.05}\\
\nopagebreak
\multirow{-4}{3cm}{\raggedright\arraybackslash Equal spacing} & 11 & \cellcolor[HTML]{ffffff}{3.00} & \cellcolor[HTML]{ffffff}{3.00} & \cellcolor[HTML]{ffffff}{3.00} & \cellcolor[HTML]{ffffff}{3.00} & \cellcolor[HTML]{ffffff}{3.00} & \cellcolor[HTML]{ffffff}{3.00} & \cellcolor[HTML]{8cb9e3}{0.05} & \cellcolor[HTML]{8cb9e3}{0.05} & \cellcolor[HTML]{8cb9e3}{0.05} & \cellcolor[HTML]{8cb9e3}{0.05} & \cellcolor[HTML]{ffffff}{0.05} & \cellcolor[HTML]{ffffff}{0.05}\\
\cmidrule{1-14}\pagebreak[0]
 & 5 & \cellcolor[HTML]{ffffff}{3.00} & \cellcolor[HTML]{ffffff}{3.00} & \cellcolor[HTML]{ffffff}{3.00} & \cellcolor[HTML]{ffffff}{3.00} & \cellcolor[HTML]{ffffff}{3.00} & \cellcolor[HTML]{ffffff}{3.00} & \cellcolor[HTML]{8cb9e3}{0.05} & \cellcolor[HTML]{8cb9e3}{0.05} & \cellcolor[HTML]{8cb9e3}{0.05} & \cellcolor[HTML]{8cb9e3}{0.05} & \cellcolor[HTML]{ffffff}{0.05} & \cellcolor[HTML]{ffffff}{0.05}\\
\nopagebreak
 & 7 & \cellcolor[HTML]{ffffff}{3.00} & \cellcolor[HTML]{ffffff}{3.00} & \cellcolor[HTML]{ffffff}{3.00} & \cellcolor[HTML]{ffffff}{3.00} & \cellcolor[HTML]{ffffff}{3.00} & \cellcolor[HTML]{ffffff}{3.00} & \cellcolor[HTML]{8cb9e3}{0.05} & \cellcolor[HTML]{8cb9e3}{0.05} & \cellcolor[HTML]{8cb9e3}{0.05} & \cellcolor[HTML]{8cb9e3}{0.05} & \cellcolor[HTML]{ffffff}{0.05} & \cellcolor[HTML]{ffffff}{0.05}\\
\nopagebreak
 & 9 & \cellcolor[HTML]{ffffff}{3.00} & \cellcolor[HTML]{ffffff}{3.00} & \cellcolor[HTML]{ffffff}{3.00} & \cellcolor[HTML]{ffffff}{3.00} & \cellcolor[HTML]{ffffff}{3.00} & \cellcolor[HTML]{ffffff}{3.00} & \cellcolor[HTML]{8cb9e3}{0.05} & \cellcolor[HTML]{8cb9e3}{0.05} & \cellcolor[HTML]{8cb9e3}{0.05} & \cellcolor[HTML]{8cb9e3}{0.05} & \cellcolor[HTML]{ffffff}{0.05} & \cellcolor[HTML]{ffffff}{0.05}\\
\nopagebreak
\multirow{-4}{3cm}{\raggedright\arraybackslash Time-interval increasing} & 11 & \cellcolor[HTML]{ffffff}{3.00} & \cellcolor[HTML]{ffffff}{3.00} & \cellcolor[HTML]{ffffff}{3.00} & \cellcolor[HTML]{ffffff}{3.00} & \cellcolor[HTML]{ffffff}{3.00} & \cellcolor[HTML]{ffffff}{3.00} & \cellcolor[HTML]{8cb9e3}{0.05} & \cellcolor[HTML]{8cb9e3}{0.05} & \cellcolor[HTML]{8cb9e3}{0.05} & \cellcolor[HTML]{8cb9e3}{0.05} & \cellcolor[HTML]{ffffff}{0.05} & \cellcolor[HTML]{ffffff}{0.05}\\
\cmidrule{1-14}\pagebreak[0]
 & 5 & \cellcolor[HTML]{ffffff}{3.00} & \cellcolor[HTML]{ffffff}{3.00} & \cellcolor[HTML]{ffffff}{3.00} & \cellcolor[HTML]{ffffff}{3.00} & \cellcolor[HTML]{ffffff}{3.00} & \cellcolor[HTML]{ffffff}{3.00} & \cellcolor[HTML]{8cb9e3}{0.05} & \cellcolor[HTML]{8cb9e3}{0.05} & \cellcolor[HTML]{8cb9e3}{0.05} & \cellcolor[HTML]{8cb9e3}{0.05} & \cellcolor[HTML]{8cb9e3}{0.05} & \cellcolor[HTML]{ffffff}{0.05}\\
\nopagebreak
 & 7 & \cellcolor[HTML]{ffffff}{3.00} & \cellcolor[HTML]{ffffff}{3.00} & \cellcolor[HTML]{ffffff}{3.00} & \cellcolor[HTML]{ffffff}{3.00} & \cellcolor[HTML]{ffffff}{3.00} & \cellcolor[HTML]{ffffff}{3.00} & \cellcolor[HTML]{8cb9e3}{0.05} & \cellcolor[HTML]{8cb9e3}{0.05} & \cellcolor[HTML]{8cb9e3}{0.05} & \cellcolor[HTML]{8cb9e3}{0.05} & \cellcolor[HTML]{ffffff}{0.05} & \cellcolor[HTML]{ffffff}{0.05}\\
\nopagebreak
 & 9 & \cellcolor[HTML]{ffffff}{3.00} & \cellcolor[HTML]{ffffff}{3.00} & \cellcolor[HTML]{ffffff}{3.00} & \cellcolor[HTML]{ffffff}{3.00} & \cellcolor[HTML]{ffffff}{3.00} & \cellcolor[HTML]{ffffff}{3.00} & \cellcolor[HTML]{8cb9e3}{0.05} & \cellcolor[HTML]{8cb9e3}{0.05} & \cellcolor[HTML]{8cb9e3}{0.05} & \cellcolor[HTML]{8cb9e3}{0.05} & \cellcolor[HTML]{ffffff}{0.05} & \cellcolor[HTML]{ffffff}{0.05}\\
\nopagebreak
\multirow{-4}{3cm}{\raggedright\arraybackslash Time-interval decreasing} & 11 & \cellcolor[HTML]{ffffff}{3.00} & \cellcolor[HTML]{ffffff}{3.00} & \cellcolor[HTML]{ffffff}{3.00} & \cellcolor[HTML]{ffffff}{3.00} & \cellcolor[HTML]{ffffff}{3.00} & \cellcolor[HTML]{ffffff}{3.00} & \cellcolor[HTML]{8cb9e3}{0.05} & \cellcolor[HTML]{8cb9e3}{0.05} & \cellcolor[HTML]{8cb9e3}{0.05} & \cellcolor[HTML]{8cb9e3}{0.05} & \cellcolor[HTML]{ffffff}{0.05} & \cellcolor[HTML]{ffffff}{0.05}\\
\cmidrule{1-14}\pagebreak[0]
 & 5 & \cellcolor[HTML]{ffffff}{3.00} & \cellcolor[HTML]{ffffff}{3.00} & \cellcolor[HTML]{ffffff}{3.00} & \cellcolor[HTML]{ffffff}{3.00} & \cellcolor[HTML]{ffffff}{3.00} & \cellcolor[HTML]{ffffff}{3.00} & \cellcolor[HTML]{8cb9e3}{0.05} & \cellcolor[HTML]{8cb9e3}{0.05} & \cellcolor[HTML]{8cb9e3}{0.05} & \cellcolor[HTML]{8cb9e3}{0.05} & \cellcolor[HTML]{8cb9e3}{0.05} & \cellcolor[HTML]{ffffff}{0.05}\\
\nopagebreak
 & 7 & \cellcolor[HTML]{ffffff}{3.00} & \cellcolor[HTML]{ffffff}{3.00} & \cellcolor[HTML]{ffffff}{3.00} & \cellcolor[HTML]{ffffff}{3.00} & \cellcolor[HTML]{ffffff}{3.00} & \cellcolor[HTML]{ffffff}{3.00} & \cellcolor[HTML]{8cb9e3}{0.05} & \cellcolor[HTML]{8cb9e3}{0.05} & \cellcolor[HTML]{8cb9e3}{0.05} & \cellcolor[HTML]{8cb9e3}{0.05} & \cellcolor[HTML]{ffffff}{0.05} & \cellcolor[HTML]{ffffff}{0.05}\\
\nopagebreak
 & 9 & \cellcolor[HTML]{ffffff}{3.00} & \cellcolor[HTML]{ffffff}{3.00} & \cellcolor[HTML]{ffffff}{3.00} & \cellcolor[HTML]{ffffff}{3.00} & \cellcolor[HTML]{ffffff}{3.00} & \cellcolor[HTML]{ffffff}{3.00} & \cellcolor[HTML]{8cb9e3}{0.05} & \cellcolor[HTML]{8cb9e3}{0.05} & \cellcolor[HTML]{8cb9e3}{0.05} & \cellcolor[HTML]{8cb9e3}{0.05} & \cellcolor[HTML]{ffffff}{0.05} & \cellcolor[HTML]{ffffff}{0.05}\\
\nopagebreak
\multirow{-4}{3cm}{\raggedright\arraybackslash Middle-and-extreme spacing} & 11 & \cellcolor[HTML]{ffffff}{3.00} & \cellcolor[HTML]{ffffff}{3.00} & \cellcolor[HTML]{ffffff}{3.00} & \cellcolor[HTML]{ffffff}{3.00} & \cellcolor[HTML]{ffffff}{3.00} & \cellcolor[HTML]{ffffff}{3.00} & \cellcolor[HTML]{8cb9e3}{0.05} & \cellcolor[HTML]{8cb9e3}{0.05} & \cellcolor[HTML]{8cb9e3}{0.05} & \cellcolor[HTML]{8cb9e3}{0.05} & \cellcolor[HTML]{ffffff}{0.05} & \cellcolor[HTML]{ffffff}{0.05}\\
\bottomrule
\end{longtable}
\end{ThreePartTable}
\addtocounter{table}{-1}
\begin{ThreePartTable}
\begin{TableNotes}
\item 
\end{TableNotes}
\begin{longtable}[l]{>{\raggedright\arraybackslash}p{3cm}>{\raggedright\arraybackslash}p{3cm}cccccccccccc}
\caption[]{Parameter Values Estimated for Day- and Likert-Unit Parameters in Experiment 2 (continued)}\\
\toprule
\multicolumn{1}{c}{} & \multicolumn{1}{c}{} & \multicolumn{6}{c}{\thead{$\upalpha_{fixed}$ (Maximal elevation) \\ Pop value = 3.32}} & \multicolumn{6}{c}{\thead{$\upalpha_{random}$ (Maximal elevation) \\ Pop value = 0.05}} \\
\cmidrule(l{3pt}r{3pt}){3-8} \cmidrule(l{3pt}r{3pt}){9-14}
Measurement Spacing & Number of Measurements & 30 & 50 & 100 & 200 & 500 & 1000 & 30 & 50 & 100 & 200 & 500 & 1000\\
\midrule
 & 5 & \cellcolor[HTML]{ffffff}{3.32} & \cellcolor[HTML]{ffffff}{3.32} & \cellcolor[HTML]{ffffff}{3.32} & \cellcolor[HTML]{ffffff}{3.32} & \cellcolor[HTML]{ffffff}{3.32} & \cellcolor[HTML]{ffffff}{3.32} & \cellcolor[HTML]{8cb9e3}{0.05} & \cellcolor[HTML]{8cb9e3}{0.05} & \cellcolor[HTML]{8cb9e3}{0.05} & \cellcolor[HTML]{8cb9e3}{0.05} & \cellcolor[HTML]{ffffff}{0.05} & \cellcolor[HTML]{ffffff}{0.05}\\
\nopagebreak
 & 7 & \cellcolor[HTML]{ffffff}{3.32} & \cellcolor[HTML]{ffffff}{3.32} & \cellcolor[HTML]{ffffff}{3.32} & \cellcolor[HTML]{ffffff}{3.32} & \cellcolor[HTML]{ffffff}{3.32} & \cellcolor[HTML]{ffffff}{3.32} & \cellcolor[HTML]{8cb9e3}{0.05} & \cellcolor[HTML]{8cb9e3}{0.05} & \cellcolor[HTML]{8cb9e3}{0.05} & \cellcolor[HTML]{8cb9e3}{0.05} & \cellcolor[HTML]{ffffff}{0.05} & \cellcolor[HTML]{ffffff}{0.05}\\
\nopagebreak
 & 9 & \cellcolor[HTML]{ffffff}{3.32} & \cellcolor[HTML]{ffffff}{3.32} & \cellcolor[HTML]{ffffff}{3.32} & \cellcolor[HTML]{ffffff}{3.32} & \cellcolor[HTML]{ffffff}{3.32} & \cellcolor[HTML]{ffffff}{3.32} & \cellcolor[HTML]{8cb9e3}{0.05} & \cellcolor[HTML]{8cb9e3}{0.05} & \cellcolor[HTML]{8cb9e3}{0.05} & \cellcolor[HTML]{8cb9e3}{0.05} & \cellcolor[HTML]{ffffff}{0.05} & \cellcolor[HTML]{ffffff}{0.05}\\
\nopagebreak
\multirow{-4}{3cm}{\raggedright\arraybackslash Equal spacing} & 11 & \cellcolor[HTML]{ffffff}{3.32} & \cellcolor[HTML]{ffffff}{3.32} & \cellcolor[HTML]{ffffff}{3.32} & \cellcolor[HTML]{ffffff}{3.32} & \cellcolor[HTML]{ffffff}{3.32} & \cellcolor[HTML]{ffffff}{3.32} & \cellcolor[HTML]{8cb9e3}{0.05} & \cellcolor[HTML]{8cb9e3}{0.05} & \cellcolor[HTML]{8cb9e3}{0.05} & \cellcolor[HTML]{8cb9e3}{0.05} & \cellcolor[HTML]{ffffff}{0.05} & \cellcolor[HTML]{ffffff}{0.05}\\
\cmidrule{1-14}\pagebreak[0]
 & 5 & \cellcolor[HTML]{ffffff}{3.32} & \cellcolor[HTML]{ffffff}{3.32} & \cellcolor[HTML]{ffffff}{3.32} & \cellcolor[HTML]{ffffff}{3.32} & \cellcolor[HTML]{ffffff}{3.32} & \cellcolor[HTML]{ffffff}{3.32} & \cellcolor[HTML]{8cb9e3}{0.05} & \cellcolor[HTML]{8cb9e3}{0.05} & \cellcolor[HTML]{8cb9e3}{0.05} & \cellcolor[HTML]{8cb9e3}{0.05} & \cellcolor[HTML]{8cb9e3}{0.05} & \cellcolor[HTML]{ffffff}{0.05}\\
\nopagebreak
 & 7 & \cellcolor[HTML]{ffffff}{3.32} & \cellcolor[HTML]{ffffff}{3.32} & \cellcolor[HTML]{ffffff}{3.32} & \cellcolor[HTML]{ffffff}{3.32} & \cellcolor[HTML]{ffffff}{3.32} & \cellcolor[HTML]{ffffff}{3.32} & \cellcolor[HTML]{8cb9e3}{0.05} & \cellcolor[HTML]{8cb9e3}{0.05} & \cellcolor[HTML]{8cb9e3}{0.05} & \cellcolor[HTML]{8cb9e3}{0.05} & \cellcolor[HTML]{ffffff}{0.05} & \cellcolor[HTML]{ffffff}{0.05}\\
\nopagebreak
 & 9 & \cellcolor[HTML]{ffffff}{3.32} & \cellcolor[HTML]{ffffff}{3.32} & \cellcolor[HTML]{ffffff}{3.32} & \cellcolor[HTML]{ffffff}{3.32} & \cellcolor[HTML]{ffffff}{3.32} & \cellcolor[HTML]{ffffff}{3.32} & \cellcolor[HTML]{8cb9e3}{0.05} & \cellcolor[HTML]{8cb9e3}{0.05} & \cellcolor[HTML]{8cb9e3}{0.05} & \cellcolor[HTML]{8cb9e3}{0.05} & \cellcolor[HTML]{ffffff}{0.05} & \cellcolor[HTML]{ffffff}{0.05}\\
\nopagebreak
\multirow{-4}{3cm}{\raggedright\arraybackslash Time-interval increasing} & 11 & \cellcolor[HTML]{ffffff}{3.32} & \cellcolor[HTML]{ffffff}{3.32} & \cellcolor[HTML]{ffffff}{3.32} & \cellcolor[HTML]{ffffff}{3.32} & \cellcolor[HTML]{ffffff}{3.32} & \cellcolor[HTML]{ffffff}{3.32} & \cellcolor[HTML]{8cb9e3}{0.05} & \cellcolor[HTML]{8cb9e3}{0.05} & \cellcolor[HTML]{8cb9e3}{0.05} & \cellcolor[HTML]{8cb9e3}{0.05} & \cellcolor[HTML]{ffffff}{0.05} & \cellcolor[HTML]{ffffff}{0.05}\\
\cmidrule{1-14}\pagebreak[0]
 & 5 & \cellcolor[HTML]{ffffff}{3.32} & \cellcolor[HTML]{ffffff}{3.32} & \cellcolor[HTML]{ffffff}{3.32} & \cellcolor[HTML]{ffffff}{3.32} & \cellcolor[HTML]{ffffff}{3.32} & \cellcolor[HTML]{ffffff}{3.32} & \cellcolor[HTML]{8cb9e3}{0.05} & \cellcolor[HTML]{8cb9e3}{0.05} & \cellcolor[HTML]{8cb9e3}{0.05} & \cellcolor[HTML]{8cb9e3}{0.05} & \cellcolor[HTML]{ffffff}{0.05} & \cellcolor[HTML]{ffffff}{0.05}\\
\nopagebreak
 & 7 & \cellcolor[HTML]{ffffff}{3.32} & \cellcolor[HTML]{ffffff}{3.32} & \cellcolor[HTML]{ffffff}{3.32} & \cellcolor[HTML]{ffffff}{3.32} & \cellcolor[HTML]{ffffff}{3.32} & \cellcolor[HTML]{ffffff}{3.32} & \cellcolor[HTML]{8cb9e3}{0.05} & \cellcolor[HTML]{8cb9e3}{0.05} & \cellcolor[HTML]{8cb9e3}{0.05} & \cellcolor[HTML]{8cb9e3}{0.05} & \cellcolor[HTML]{ffffff}{0.05} & \cellcolor[HTML]{ffffff}{0.05}\\
\nopagebreak
 & 9 & \cellcolor[HTML]{ffffff}{3.32} & \cellcolor[HTML]{ffffff}{3.32} & \cellcolor[HTML]{ffffff}{3.32} & \cellcolor[HTML]{ffffff}{3.32} & \cellcolor[HTML]{ffffff}{3.32} & \cellcolor[HTML]{ffffff}{3.32} & \cellcolor[HTML]{8cb9e3}{0.05} & \cellcolor[HTML]{8cb9e3}{0.05} & \cellcolor[HTML]{8cb9e3}{0.05} & \cellcolor[HTML]{8cb9e3}{0.05} & \cellcolor[HTML]{ffffff}{0.05} & \cellcolor[HTML]{ffffff}{0.05}\\
\nopagebreak
\multirow{-4}{3cm}{\raggedright\arraybackslash Time-interval decreasing} & 11 & \cellcolor[HTML]{ffffff}{3.32} & \cellcolor[HTML]{ffffff}{3.32} & \cellcolor[HTML]{ffffff}{3.32} & \cellcolor[HTML]{ffffff}{3.32} & \cellcolor[HTML]{ffffff}{3.32} & \cellcolor[HTML]{ffffff}{3.32} & \cellcolor[HTML]{8cb9e3}{0.05} & \cellcolor[HTML]{8cb9e3}{0.05} & \cellcolor[HTML]{8cb9e3}{0.05} & \cellcolor[HTML]{8cb9e3}{0.05} & \cellcolor[HTML]{ffffff}{0.05} & \cellcolor[HTML]{ffffff}{0.05}\\
\cmidrule{1-14}\pagebreak[0]
 & 5 & \cellcolor[HTML]{ffffff}{3.32} & \cellcolor[HTML]{ffffff}{3.32} & \cellcolor[HTML]{ffffff}{3.32} & \cellcolor[HTML]{ffffff}{3.32} & \cellcolor[HTML]{ffffff}{3.32} & \cellcolor[HTML]{ffffff}{3.32} & \cellcolor[HTML]{8cb9e3}{0.05} & \cellcolor[HTML]{8cb9e3}{0.05} & \cellcolor[HTML]{8cb9e3}{0.05} & \cellcolor[HTML]{8cb9e3}{0.05} & \cellcolor[HTML]{8cb9e3}{0.05} & \cellcolor[HTML]{ffffff}{0.05}\\
\nopagebreak
 & 7 & \cellcolor[HTML]{ffffff}{3.32} & \cellcolor[HTML]{ffffff}{3.32} & \cellcolor[HTML]{ffffff}{3.32} & \cellcolor[HTML]{ffffff}{3.32} & \cellcolor[HTML]{ffffff}{3.32} & \cellcolor[HTML]{ffffff}{3.32} & \cellcolor[HTML]{8cb9e3}{0.05} & \cellcolor[HTML]{8cb9e3}{0.05} & \cellcolor[HTML]{8cb9e3}{0.05} & \cellcolor[HTML]{8cb9e3}{0.05} & \cellcolor[HTML]{ffffff}{0.05} & \cellcolor[HTML]{ffffff}{0.05}\\
\nopagebreak
 & 9 & \cellcolor[HTML]{ffffff}{3.32} & \cellcolor[HTML]{ffffff}{3.32} & \cellcolor[HTML]{ffffff}{3.32} & \cellcolor[HTML]{ffffff}{3.32} & \cellcolor[HTML]{ffffff}{3.32} & \cellcolor[HTML]{ffffff}{3.32} & \cellcolor[HTML]{8cb9e3}{0.05} & \cellcolor[HTML]{8cb9e3}{0.05} & \cellcolor[HTML]{8cb9e3}{0.05} & \cellcolor[HTML]{8cb9e3}{0.05} & \cellcolor[HTML]{ffffff}{0.05} & \cellcolor[HTML]{ffffff}{0.05}\\
\nopagebreak
\multirow{-4}{3cm}{\raggedright\arraybackslash Middle-and-extreme spacing} & 11 & \cellcolor[HTML]{ffffff}{3.32} & \cellcolor[HTML]{ffffff}{3.32} & \cellcolor[HTML]{ffffff}{3.32} & \cellcolor[HTML]{ffffff}{3.32} & \cellcolor[HTML]{ffffff}{3.32} & \cellcolor[HTML]{ffffff}{3.32} & \cellcolor[HTML]{8cb9e3}{0.05} & \cellcolor[HTML]{8cb9e3}{0.05} & \cellcolor[HTML]{8cb9e3}{0.05} & \cellcolor[HTML]{8cb9e3}{0.05} & \cellcolor[HTML]{ffffff}{0.05} & \cellcolor[HTML]{ffffff}{0.05}\\
\bottomrule
\end{longtable}
\end{ThreePartTable}
\addtocounter{table}{-1}

\begin{ThreePartTable}
\begin{TableNotes}
\item \textit{Note. }Cells shaded in light blue indicate cells where estimation is imprecise (i.e., lower and/or upper whisker lengths exceeding 10\% of the parameter's population value. Empty superscript squares ($^{\square}$) indicate biased estimates (i.e., bias exceeding 10\% of parameter's population value).
\end{TableNotes}
\begin{longtable}[l]{>{\raggedright\arraybackslash}p{3cm}>{\raggedright\arraybackslash}p{3cm}cccccc}
\caption[]{Parameter Values Estimated for Day- and Likert-Unit Parameters in Experiment 2 (continued)}\\
\toprule
\multicolumn{1}{c}{} & \multicolumn{1}{c}{} & \multicolumn{6}{c}{\thead{$\upepsilon$(error) \\ Pop value = 0.03}} \\
\cmidrule(l{3pt}r{3pt}){3-8}
Measurement Spacing & Number of Measurements & 30 & 50 & 100 & 200 & 500 & 1000\\
\midrule
 & 5 & \cellcolor[HTML]{8cb9e3}{0.05} & \cellcolor[HTML]{8cb9e3}{0.05} & \cellcolor[HTML]{8cb9e3}{0.05} & \cellcolor[HTML]{ffffff}{0.05} & \cellcolor[HTML]{ffffff}{0.05} & \cellcolor[HTML]{ffffff}{0.05}\\
\nopagebreak
 & 7 & \cellcolor[HTML]{8cb9e3}{0.05} & \cellcolor[HTML]{8cb9e3}{0.05} & \cellcolor[HTML]{ffffff}{0.05} & \cellcolor[HTML]{ffffff}{0.05} & \cellcolor[HTML]{ffffff}{0.05} & \cellcolor[HTML]{ffffff}{0.05}\\
\nopagebreak
 & 9 & \cellcolor[HTML]{8cb9e3}{0.05} & \cellcolor[HTML]{ffffff}{0.05} & \cellcolor[HTML]{ffffff}{0.05} & \cellcolor[HTML]{ffffff}{0.05} & \cellcolor[HTML]{ffffff}{0.05} & \cellcolor[HTML]{ffffff}{0.05}\\
\nopagebreak
\multirow{-4}{3cm}{\raggedright\arraybackslash Equal spacing} & 11 & \cellcolor[HTML]{ffffff}{0.05} & \cellcolor[HTML]{ffffff}{0.05} & \cellcolor[HTML]{ffffff}{0.05} & \cellcolor[HTML]{ffffff}{0.05} & \cellcolor[HTML]{ffffff}{0.05} & \cellcolor[HTML]{ffffff}{0.05}\\
\cmidrule{1-8}\pagebreak[0]
 & 5 & \cellcolor[HTML]{8cb9e3}{0.05} & \cellcolor[HTML]{8cb9e3}{0.05} & \cellcolor[HTML]{8cb9e3}{0.05} & \cellcolor[HTML]{ffffff}{0.05} & \cellcolor[HTML]{ffffff}{0.05} & \cellcolor[HTML]{ffffff}{0.05}\\
\nopagebreak
 & 7 & \cellcolor[HTML]{8cb9e3}{0.05} & \cellcolor[HTML]{8cb9e3}{0.05} & \cellcolor[HTML]{ffffff}{0.05} & \cellcolor[HTML]{ffffff}{0.05} & \cellcolor[HTML]{ffffff}{0.05} & \cellcolor[HTML]{ffffff}{0.05}\\
\nopagebreak
 & 9 & \cellcolor[HTML]{8cb9e3}{0.05} & \cellcolor[HTML]{ffffff}{0.05} & \cellcolor[HTML]{ffffff}{0.05} & \cellcolor[HTML]{ffffff}{0.05} & \cellcolor[HTML]{ffffff}{0.05} & \cellcolor[HTML]{ffffff}{0.05}\\
\nopagebreak
\multirow{-4}{3cm}{\raggedright\arraybackslash Time-interval increasing} & 11 & \cellcolor[HTML]{ffffff}{0.05} & \cellcolor[HTML]{ffffff}{0.05} & \cellcolor[HTML]{ffffff}{0.05} & \cellcolor[HTML]{ffffff}{0.05} & \cellcolor[HTML]{ffffff}{0.05} & \cellcolor[HTML]{ffffff}{0.05}\\
\cmidrule{1-8}\pagebreak[0]
 & 5 & \cellcolor[HTML]{8cb9e3}{0.05} & \cellcolor[HTML]{8cb9e3}{0.05} & \cellcolor[HTML]{8cb9e3}{0.05} & \cellcolor[HTML]{ffffff}{0.05} & \cellcolor[HTML]{ffffff}{0.05} & \cellcolor[HTML]{ffffff}{0.05}\\
\nopagebreak
 & 7 & \cellcolor[HTML]{8cb9e3}{0.05} & \cellcolor[HTML]{8cb9e3}{0.05} & \cellcolor[HTML]{ffffff}{0.05} & \cellcolor[HTML]{ffffff}{0.05} & \cellcolor[HTML]{ffffff}{0.05} & \cellcolor[HTML]{ffffff}{0.05}\\
\nopagebreak
 & 9 & \cellcolor[HTML]{8cb9e3}{0.05} & \cellcolor[HTML]{ffffff}{0.05} & \cellcolor[HTML]{ffffff}{0.05} & \cellcolor[HTML]{ffffff}{0.05} & \cellcolor[HTML]{ffffff}{0.05} & \cellcolor[HTML]{ffffff}{0.05}\\
\nopagebreak
\multirow{-4}{3cm}{\raggedright\arraybackslash Time-interval decreasing} & 11 & \cellcolor[HTML]{ffffff}{0.05} & \cellcolor[HTML]{ffffff}{0.05} & \cellcolor[HTML]{ffffff}{0.05} & \cellcolor[HTML]{ffffff}{0.05} & \cellcolor[HTML]{ffffff}{0.05} & \cellcolor[HTML]{ffffff}{0.05}\\
\cmidrule{1-8}\pagebreak[0]
 & 5 & \cellcolor[HTML]{8cb9e3}{0.05} & \cellcolor[HTML]{8cb9e3}{0.05} & \cellcolor[HTML]{8cb9e3}{0.05} & \cellcolor[HTML]{ffffff}{0.05} & \cellcolor[HTML]{ffffff}{0.05} & \cellcolor[HTML]{ffffff}{0.05}\\
\nopagebreak
 & 7 & \cellcolor[HTML]{8cb9e3}{0.05} & \cellcolor[HTML]{8cb9e3}{0.05} & \cellcolor[HTML]{ffffff}{0.05} & \cellcolor[HTML]{ffffff}{0.05} & \cellcolor[HTML]{ffffff}{0.05} & \cellcolor[HTML]{ffffff}{0.05}\\
\nopagebreak
 & 9 & \cellcolor[HTML]{8cb9e3}{0.05} & \cellcolor[HTML]{ffffff}{0.05} & \cellcolor[HTML]{ffffff}{0.05} & \cellcolor[HTML]{ffffff}{0.05} & \cellcolor[HTML]{ffffff}{0.05} & \cellcolor[HTML]{ffffff}{0.05}\\
\nopagebreak
\multirow{-4}{3cm}{\raggedright\arraybackslash Middle-and-extreme spacing} & 11 & \cellcolor[HTML]{8cb9e3}{0.05} & \cellcolor[HTML]{ffffff}{0.05} & \cellcolor[HTML]{ffffff}{0.05} & \cellcolor[HTML]{ffffff}{0.05} & \cellcolor[HTML]{ffffff}{0.05} & \cellcolor[HTML]{ffffff}{0.05}\\
\bottomrule
\insertTableNotes
\end{longtable}
\end{ThreePartTable}
\secapp{Experiment 3}
\end{landscape}
\restoregeometry

\newgeometry{margin=2.54cm}
\begin{landscape}
\begin{ThreePartTable}
\begin{TableNotes}
\item 
\end{TableNotes}
\begin{longtable}[l]{>{\raggedright\arraybackslash}p{3cm}>{\raggedright\arraybackslash}p{3cm}cccccccccccc}
\caption{Parameter Values Estimated in Experiment 3}\label{tab:param-exp-3}\\
\toprule
\multicolumn{1}{c}{} & \multicolumn{1}{c}{} & \multicolumn{6}{c}{\thead{$\upbeta_{fixed}$ (Days to halfway elevation) \\ Pop value = 180.00}} & \multicolumn{6}{c}{\thead{$\upbeta_{random}$ (Days to halfway elevation) \\ Pop value = 10.00}} \\
\cmidrule(l{3pt}r{3pt}){3-8} \cmidrule(l{3pt}r{3pt}){9-14}
Time Structuredness & Number of Measurements & 30 & 50 & 100 & 200 & 500 & 1000 & 30 & 50 & 100 & 200 & 500 & 1000\\
\midrule
 & 5 & \cellcolor[HTML]{ffffff}{179.71} & \cellcolor[HTML]{ffffff}{179.67} & \cellcolor[HTML]{ffffff}{179.75} & \cellcolor[HTML]{ffffff}{179.98} & \cellcolor[HTML]{ffffff}{180.00} & \cellcolor[HTML]{ffffff}{179.66} & \cellcolor[HTML]{8cb9e3}{10.40} & \cellcolor[HTML]{8cb9e3}{10.27} & \cellcolor[HTML]{8cb9e3}{10.37} & \cellcolor[HTML]{8cb9e3}{10.56} & \cellcolor[HTML]{8cb9e3}{10.73} & \cellcolor[HTML]{8cb9e3}{10.69}\\
\nopagebreak
 & 7 & \cellcolor[HTML]{ffffff}{180.05} & \cellcolor[HTML]{ffffff}{179.59} & \cellcolor[HTML]{ffffff}{179.02} & \cellcolor[HTML]{ffffff}{179.66} & \cellcolor[HTML]{ffffff}{180.03} & \cellcolor[HTML]{ffffff}{179.63} & \cellcolor[HTML]{8cb9e3}{10.18} & \cellcolor[HTML]{8cb9e3}{10.42} & \cellcolor[HTML]{8cb9e3}{10.65} & \cellcolor[HTML]{8cb9e3}{10.52} & \cellcolor[HTML]{8cb9e3}{10.76} & \cellcolor[HTML]{8cb9e3}{10.60}\\
\nopagebreak
 & 9 & \cellcolor[HTML]{ffffff}{179.84} & \cellcolor[HTML]{ffffff}{180.01} & \cellcolor[HTML]{ffffff}{180.01} & \cellcolor[HTML]{ffffff}{179.97} & \cellcolor[HTML]{ffffff}{180.01} & \cellcolor[HTML]{ffffff}{180.00} & \cellcolor[HTML]{8cb9e3}{10.28} & \cellcolor[HTML]{8cb9e3}{10.28} & \cellcolor[HTML]{8cb9e3}{10.37} & \cellcolor[HTML]{8cb9e3}{10.46} & \cellcolor[HTML]{8cb9e3}{10.42} & \cellcolor[HTML]{8cb9e3}{10.41}\\
\nopagebreak
\multirow{-4}{3cm}{\raggedright\arraybackslash Time structured} & 11 & \cellcolor[HTML]{ffffff}{180.11} & \cellcolor[HTML]{ffffff}{179.91} & \cellcolor[HTML]{ffffff}{179.94} & \cellcolor[HTML]{ffffff}{180.00} & \cellcolor[HTML]{ffffff}{180.00} & \cellcolor[HTML]{ffffff}{180.00} & \cellcolor[HTML]{8cb9e3}{10.08} & \cellcolor[HTML]{8cb9e3}{10.32} & \cellcolor[HTML]{8cb9e3}{10.21} & \cellcolor[HTML]{8cb9e3}{10.29} & \cellcolor[HTML]{8cb9e3}{10.36} & \cellcolor[HTML]{8cb9e3}{10.31}\\
\cmidrule{1-14}\pagebreak[0]
 & 5 & \cellcolor[HTML]{ffffff}{177.48} & \cellcolor[HTML]{ffffff}{177.24} & \cellcolor[HTML]{ffffff}{176.74} & \cellcolor[HTML]{ffffff}{177.50} & \cellcolor[HTML]{ffffff}{177.42} & \cellcolor[HTML]{ffffff}{177.06} & \cellcolor[HTML]{8cb9e3}{10.65} & \cellcolor[HTML]{8cb9e3}{10.36} & \cellcolor[HTML]{8cb9e3}{10.38} & \cellcolor[HTML]{8cb9e3}{10.65} & \cellcolor[HTML]{8cb9e3}{10.85} & \cellcolor[HTML]{8cb9e3}{10.96}\\
\nopagebreak
 & 7 & \cellcolor[HTML]{ffffff}{176.89} & \cellcolor[HTML]{ffffff}{177.03} & \cellcolor[HTML]{ffffff}{176.37} & \cellcolor[HTML]{ffffff}{175.92} & \cellcolor[HTML]{ffffff}{177.20} & \cellcolor[HTML]{ffffff}{176.95} & \cellcolor[HTML]{8cb9e3}{10.53} & \cellcolor[HTML]{8cb9e3}{10.60} & \cellcolor[HTML]{8cb9e3}{10.88} & \cellcolor[HTML]{8cb9e3}{10.83} & \cellcolor[HTML]{8cb9e3}{10.84} & \cellcolor[HTML]{8cb9e3}{10.84}\\
\nopagebreak
 & 9 & \cellcolor[HTML]{ffffff}{177.54} & \cellcolor[HTML]{ffffff}{177.28} & \cellcolor[HTML]{ffffff}{177.27} & \cellcolor[HTML]{ffffff}{177.31} & \cellcolor[HTML]{ffffff}{177.34} & \cellcolor[HTML]{ffffff}{177.33} & \cellcolor[HTML]{8cb9e3}{10.66} & \cellcolor[HTML]{8cb9e3}{10.43} & \cellcolor[HTML]{8cb9e3}{10.44} & \cellcolor[HTML]{8cb9e3}{10.61} & \cellcolor[HTML]{8cb9e3}{10.65} & \cellcolor[HTML]{8cb9e3}{10.59}\\
\nopagebreak
\multirow{-4}{3cm}{\raggedright\arraybackslash Time unstructured (fast response)} & 11 & \cellcolor[HTML]{ffffff}{177.25} & \cellcolor[HTML]{ffffff}{177.35} & \cellcolor[HTML]{ffffff}{177.27} & \cellcolor[HTML]{ffffff}{177.37} & \cellcolor[HTML]{ffffff}{177.35} & \cellcolor[HTML]{ffffff}{177.30} & \cellcolor[HTML]{8cb9e3}{10.41} & \cellcolor[HTML]{8cb9e3}{10.37} & \cellcolor[HTML]{8cb9e3}{10.37} & \cellcolor[HTML]{8cb9e3}{10.45} & \cellcolor[HTML]{8cb9e3}{10.52} & \cellcolor[HTML]{8cb9e3}{10.51}\\
\cmidrule{1-14}\pagebreak[0]
 & 5 & \cellcolor[HTML]{ffffff}{174.13} & \cellcolor[HTML]{ffffff}{174.02} & \cellcolor[HTML]{ffffff}{173.65} & \cellcolor[HTML]{ffffff}{173.85} & \cellcolor[HTML]{ffffff}{173.41} & \cellcolor[HTML]{ffffff}{173.63} & \cellcolor[HTML]{8cb9e3}{11.23$^{\square}$} & \cellcolor[HTML]{8cb9e3}{10.93} & \cellcolor[HTML]{8cb9e3}{11.22$^{\square}$} & \cellcolor[HTML]{8cb9e3}{11.80$^{\square}$} & \cellcolor[HTML]{8cb9e3}{12.10$^{\square}$} & \cellcolor[HTML]{8cb9e3}{12.07$^{\square}$}\\
\nopagebreak
 & 7 & \cellcolor[HTML]{ffffff}{173.31} & \cellcolor[HTML]{ffffff}{173.63} & \cellcolor[HTML]{ffffff}{173.01} & \cellcolor[HTML]{ffffff}{173.06} & \cellcolor[HTML]{ffffff}{173.55} & \cellcolor[HTML]{ffffff}{173.55} & \cellcolor[HTML]{8cb9e3}{11.71$^{\square}$} & \cellcolor[HTML]{8cb9e3}{11.67$^{\square}$} & \cellcolor[HTML]{8cb9e3}{11.88$^{\square}$} & \cellcolor[HTML]{8cb9e3}{11.97$^{\square}$} & \cellcolor[HTML]{8cb9e3}{11.91$^{\square}$} & \cellcolor[HTML]{8cb9e3}{11.94$^{\square}$}\\
\nopagebreak
 & 9 & \cellcolor[HTML]{ffffff}{173.37} & \cellcolor[HTML]{ffffff}{173.37} & \cellcolor[HTML]{ffffff}{173.54} & \cellcolor[HTML]{ffffff}{173.52} & \cellcolor[HTML]{ffffff}{173.50} & \cellcolor[HTML]{ffffff}{173.49} & \cellcolor[HTML]{8cb9e3}{11.26$^{\square}$} & \cellcolor[HTML]{8cb9e3}{11.38$^{\square}$} & \cellcolor[HTML]{8cb9e3}{11.42$^{\square}$} & \cellcolor[HTML]{8cb9e3}{11.40$^{\square}$} & \cellcolor[HTML]{8cb9e3}{11.47$^{\square}$} & \cellcolor[HTML]{8cb9e3}{11.46$^{\square}$}\\
\nopagebreak
\multirow{-4}{3cm}{\raggedright\arraybackslash Time unstructured (slow response)} & 11 & \cellcolor[HTML]{ffffff}{173.58} & \cellcolor[HTML]{ffffff}{173.56} & \cellcolor[HTML]{ffffff}{173.50} & \cellcolor[HTML]{ffffff}{173.51} & \cellcolor[HTML]{ffffff}{173.49} & \cellcolor[HTML]{ffffff}{173.47} & \cellcolor[HTML]{8cb9e3}{10.87} & \cellcolor[HTML]{8cb9e3}{10.98} & \cellcolor[HTML]{8cb9e3}{11.12$^{\square}$} & \cellcolor[HTML]{8cb9e3}{11.18$^{\square}$} & \cellcolor[HTML]{8cb9e3}{11.14$^{\square}$} & \cellcolor[HTML]{8cb9e3}{11.16$^{\square}$}\\
\cmidrule{1-14}\pagebreak[0]
 & 5 & \cellcolor[HTML]{ffffff}{179.92} & \cellcolor[HTML]{ffffff}{179.87} & \cellcolor[HTML]{ffffff}{179.97} & \cellcolor[HTML]{ffffff}{179.92} & \cellcolor[HTML]{ffffff}{179.87} & \cellcolor[HTML]{ffffff}{179.88} & \cellcolor[HTML]{8cb9e3}{10.70} & \cellcolor[HTML]{8cb9e3}{10.40} & \cellcolor[HTML]{8cb9e3}{10.35} & \cellcolor[HTML]{8cb9e3}{10.50} & \cellcolor[HTML]{8cb9e3}{10.66} & \cellcolor[HTML]{8cb9e3}{10.61}\\
\nopagebreak
 & 7 & \cellcolor[HTML]{ffffff}{180.07} & \cellcolor[HTML]{ffffff}{179.96} & \cellcolor[HTML]{ffffff}{179.96} & \cellcolor[HTML]{ffffff}{179.92} & \cellcolor[HTML]{ffffff}{179.91} & \cellcolor[HTML]{ffffff}{179.94} & \cellcolor[HTML]{8cb9e3}{10.32} & \cellcolor[HTML]{8cb9e3}{10.32} & \cellcolor[HTML]{8cb9e3}{10.33} & \cellcolor[HTML]{8cb9e3}{10.52} & \cellcolor[HTML]{8cb9e3}{10.53} & \cellcolor[HTML]{8cb9e3}{10.50}\\
\nopagebreak
 & 9 & \cellcolor[HTML]{ffffff}{180.17} & \cellcolor[HTML]{ffffff}{179.86} & \cellcolor[HTML]{ffffff}{179.88} & \cellcolor[HTML]{ffffff}{179.97} & \cellcolor[HTML]{ffffff}{179.95} & \cellcolor[HTML]{ffffff}{179.98} & \cellcolor[HTML]{8cb9e3}{10.12} & \cellcolor[HTML]{8cb9e3}{10.26} & \cellcolor[HTML]{8cb9e3}{10.43} & \cellcolor[HTML]{8cb9e3}{10.32} & \cellcolor[HTML]{8cb9e3}{10.40} & \cellcolor[HTML]{8cb9e3}{10.38}\\
\nopagebreak
\multirow{-4}{3cm}{\raggedright\arraybackslash Time unstructured (slow response) with definition variables} & 11 & \cellcolor[HTML]{ffffff}{179.93} & \cellcolor[HTML]{ffffff}{180.20} & \cellcolor[HTML]{ffffff}{179.94} & \cellcolor[HTML]{ffffff}{179.97} & \cellcolor[HTML]{ffffff}{179.99} & \cellcolor[HTML]{ffffff}{179.99} & \cellcolor[HTML]{8cb9e3}{10.11} & \cellcolor[HTML]{8cb9e3}{10.20} & \cellcolor[HTML]{8cb9e3}{10.34} & \cellcolor[HTML]{8cb9e3}{10.31} & \cellcolor[HTML]{8cb9e3}{10.27} & \cellcolor[HTML]{8cb9e3}{10.32}\\
\bottomrule
\insertTableNotes
\end{longtable}
\end{ThreePartTable}
\addtocounter{table}{-1}
\begin{ThreePartTable}
\begin{TableNotes}
\item 
\end{TableNotes}
\begin{longtable}[l]{>{\raggedright\arraybackslash}p{3cm}>{\raggedright\arraybackslash}p{3cm}cccccccccccc}
\caption[]{Parameter Values Estimated for Day- and Likert-Unit Parameters in Experiment 3 (continued)}\\
\toprule
\multicolumn{1}{c}{} & \multicolumn{1}{c}{} & \multicolumn{6}{c}{\thead{$\upgamma_{fixed}$ (Triquarter-halfway delta) \\ Pop value = 20.00}} & \multicolumn{6}{c}{\thead{$\upgamma_{random}$ (Triquarter-halfway delta) \\ Pop value = 4.00}} \\
\cmidrule(l{3pt}r{3pt}){3-8} \cmidrule(l{3pt}r{3pt}){9-14}
Time Structuredness & Number of Measurements & 30 & 50 & 100 & 200 & 500 & 1000 & 30 & 50 & 100 & 200 & 500 & 1000\\
\midrule
 & 5 & \cellcolor[HTML]{8cb9e3}{18.25} & \cellcolor[HTML]{8cb9e3}{18.11} & \cellcolor[HTML]{8cb9e3}{18.46} & \cellcolor[HTML]{8cb9e3}{19.67} & \cellcolor[HTML]{8cb9e3}{20.55} & \cellcolor[HTML]{8cb9e3}{20.65} & \cellcolor[HTML]{8cb9e3}{17.69$^{\square}$} & \cellcolor[HTML]{8cb9e3}{17.05$^{\square}$} & \cellcolor[HTML]{8cb9e3}{16.38$^{\square}$} & \cellcolor[HTML]{8cb9e3}{15.03$^{\square}$} & \cellcolor[HTML]{8cb9e3}{11.63$^{\square}$} & \cellcolor[HTML]{8cb9e3}{9.02$^{\square}$}\\
\nopagebreak
 & 7 & \cellcolor[HTML]{8cb9e3}{20.25} & \cellcolor[HTML]{8cb9e3}{20.79} & \cellcolor[HTML]{8cb9e3}{20.67} & \cellcolor[HTML]{8cb9e3}{20.77} & \cellcolor[HTML]{8cb9e3}{20.98} & \cellcolor[HTML]{8cb9e3}{20.93} & \cellcolor[HTML]{8cb9e3}{ 9.22$^{\square}$} & \cellcolor[HTML]{8cb9e3}{ 7.32$^{\square}$} & \cellcolor[HTML]{8cb9e3}{ 6.12$^{\square}$} & \cellcolor[HTML]{8cb9e3}{ 4.99$^{\square}$} & \cellcolor[HTML]{8cb9e3}{ 4.45$^{\square}$} & \cellcolor[HTML]{8cb9e3}{4.69$^{\square}$}\\
\nopagebreak
 & 9 & \cellcolor[HTML]{8cb9e3}{20.88} & \cellcolor[HTML]{8cb9e3}{20.79} & \cellcolor[HTML]{8cb9e3}{20.84} & \cellcolor[HTML]{8cb9e3}{20.69} & \cellcolor[HTML]{ffffff}{20.74} & \cellcolor[HTML]{ffffff}{20.71} & \cellcolor[HTML]{8cb9e3}{ 5.30$^{\square}$} & \cellcolor[HTML]{8cb9e3}{ 4.95$^{\square}$} & \cellcolor[HTML]{8cb9e3}{ 4.34} & \cellcolor[HTML]{8cb9e3}{ 4.13} & \cellcolor[HTML]{8cb9e3}{ 4.05} & \cellcolor[HTML]{8cb9e3}{3.96}\\
\nopagebreak
\multirow{-4}{3cm}{\raggedright\arraybackslash Time structured} & 11 & \cellcolor[HTML]{8cb9e3}{20.65} & \cellcolor[HTML]{8cb9e3}{20.74} & \cellcolor[HTML]{8cb9e3}{20.73} & \cellcolor[HTML]{8cb9e3}{20.69} & \cellcolor[HTML]{ffffff}{20.71} & \cellcolor[HTML]{ffffff}{20.67} & \cellcolor[HTML]{8cb9e3}{ 4.86$^{\square}$} & \cellcolor[HTML]{8cb9e3}{ 4.41$^{\square}$} & \cellcolor[HTML]{8cb9e3}{ 4.17} & \cellcolor[HTML]{8cb9e3}{ 4.13} & \cellcolor[HTML]{8cb9e3}{ 4.09} & \cellcolor[HTML]{8cb9e3}{4.03}\\
\cmidrule{1-14}\pagebreak[0]
 & 5 & \cellcolor[HTML]{8cb9e3}{18.57} & \cellcolor[HTML]{8cb9e3}{18.16} & \cellcolor[HTML]{8cb9e3}{18.59} & \cellcolor[HTML]{8cb9e3}{19.45} & \cellcolor[HTML]{8cb9e3}{20.15} & \cellcolor[HTML]{8cb9e3}{20.58} & \cellcolor[HTML]{8cb9e3}{16.85$^{\square}$} & \cellcolor[HTML]{8cb9e3}{16.21$^{\square}$} & \cellcolor[HTML]{8cb9e3}{14.96$^{\square}$} & \cellcolor[HTML]{8cb9e3}{13.48$^{\square}$} & \cellcolor[HTML]{8cb9e3}{ 9.94$^{\square}$} & \cellcolor[HTML]{8cb9e3}{7.72$^{\square}$}\\
\nopagebreak
 & 7 & \cellcolor[HTML]{8cb9e3}{20.39} & \cellcolor[HTML]{8cb9e3}{20.44} & \cellcolor[HTML]{8cb9e3}{20.67} & \cellcolor[HTML]{8cb9e3}{20.73} & \cellcolor[HTML]{8cb9e3}{20.77} & \cellcolor[HTML]{8cb9e3}{20.77} & \cellcolor[HTML]{8cb9e3}{ 9.65$^{\square}$} & \cellcolor[HTML]{8cb9e3}{ 7.07$^{\square}$} & \cellcolor[HTML]{8cb9e3}{ 6.25$^{\square}$} & \cellcolor[HTML]{8cb9e3}{ 5.47$^{\square}$} & \cellcolor[HTML]{8cb9e3}{ 4.61$^{\square}$} & \cellcolor[HTML]{8cb9e3}{4.34}\\
\nopagebreak
 & 9 & \cellcolor[HTML]{8cb9e3}{20.54} & \cellcolor[HTML]{8cb9e3}{20.66} & \cellcolor[HTML]{8cb9e3}{20.75} & \cellcolor[HTML]{8cb9e3}{20.71} & \cellcolor[HTML]{ffffff}{20.72} & \cellcolor[HTML]{ffffff}{20.74} & \cellcolor[HTML]{8cb9e3}{ 5.27$^{\square}$} & \cellcolor[HTML]{8cb9e3}{ 4.68$^{\square}$} & \cellcolor[HTML]{8cb9e3}{ 4.59$^{\square}$} & \cellcolor[HTML]{8cb9e3}{ 4.08} & \cellcolor[HTML]{8cb9e3}{ 4.06} & \cellcolor[HTML]{8cb9e3}{4.05}\\
\nopagebreak
\multirow{-4}{3cm}{\raggedright\arraybackslash Time unstructured (fast response)} & 11 & \cellcolor[HTML]{8cb9e3}{20.77} & \cellcolor[HTML]{8cb9e3}{20.70} & \cellcolor[HTML]{8cb9e3}{20.72} & \cellcolor[HTML]{8cb9e3}{20.70} & \cellcolor[HTML]{ffffff}{20.71} & \cellcolor[HTML]{ffffff}{20.73} & \cellcolor[HTML]{8cb9e3}{ 4.85$^{\square}$} & \cellcolor[HTML]{8cb9e3}{ 4.68$^{\square}$} & \cellcolor[HTML]{8cb9e3}{ 4.29} & \cellcolor[HTML]{8cb9e3}{ 4.14} & \cellcolor[HTML]{8cb9e3}{ 4.16} & \cellcolor[HTML]{8cb9e3}{4.14}\\
\cmidrule{1-14}\pagebreak[0]
 & 5 & \cellcolor[HTML]{8cb9e3}{18.66} & \cellcolor[HTML]{8cb9e3}{17.88} & \cellcolor[HTML]{8cb9e3}{18.34} & \cellcolor[HTML]{8cb9e3}{19.83} & \cellcolor[HTML]{8cb9e3}{20.57} & \cellcolor[HTML]{8cb9e3}{20.67} & \cellcolor[HTML]{8cb9e3}{14.54$^{\square}$} & \cellcolor[HTML]{8cb9e3}{13.26$^{\square}$} & \cellcolor[HTML]{8cb9e3}{11.51$^{\square}$} & \cellcolor[HTML]{8cb9e3}{10.05$^{\square}$} & \cellcolor[HTML]{8cb9e3}{ 7.89$^{\square}$} & \cellcolor[HTML]{8cb9e3}{6.65$^{\square}$}\\
\nopagebreak
 & 7 & \cellcolor[HTML]{8cb9e3}{20.51} & \cellcolor[HTML]{8cb9e3}{20.73} & \cellcolor[HTML]{8cb9e3}{20.75} & \cellcolor[HTML]{8cb9e3}{20.89} & \cellcolor[HTML]{8cb9e3}{20.89} & \cellcolor[HTML]{8cb9e3}{20.86} & \cellcolor[HTML]{8cb9e3}{ 7.62$^{\square}$} & \cellcolor[HTML]{8cb9e3}{ 6.65$^{\square}$} & \cellcolor[HTML]{8cb9e3}{ 5.61$^{\square}$} & \cellcolor[HTML]{8cb9e3}{ 5.21$^{\square}$} & \cellcolor[HTML]{8cb9e3}{ 4.83$^{\square}$} & \cellcolor[HTML]{8cb9e3}{4.67$^{\square}$}\\
\nopagebreak
 & 9 & \cellcolor[HTML]{8cb9e3}{20.91} & \cellcolor[HTML]{8cb9e3}{20.82} & \cellcolor[HTML]{8cb9e3}{20.82} & \cellcolor[HTML]{8cb9e3}{20.89} & \cellcolor[HTML]{8cb9e3}{20.94} & \cellcolor[HTML]{ffffff}{20.89} & \cellcolor[HTML]{8cb9e3}{ 6.00$^{\square}$} & \cellcolor[HTML]{8cb9e3}{ 5.32$^{\square}$} & \cellcolor[HTML]{8cb9e3}{ 4.97$^{\square}$} & \cellcolor[HTML]{8cb9e3}{ 4.67$^{\square}$} & \cellcolor[HTML]{8cb9e3}{ 4.74$^{\square}$} & \cellcolor[HTML]{8cb9e3}{4.70$^{\square}$}\\
\nopagebreak
\multirow{-4}{3cm}{\raggedright\arraybackslash Time unstructured (slow response)} & 11 & \cellcolor[HTML]{8cb9e3}{20.98} & \cellcolor[HTML]{8cb9e3}{20.85} & \cellcolor[HTML]{8cb9e3}{20.90} & \cellcolor[HTML]{8cb9e3}{20.92} & \cellcolor[HTML]{ffffff}{20.90} & \cellcolor[HTML]{ffffff}{20.90} & \cellcolor[HTML]{8cb9e3}{ 5.26$^{\square}$} & \cellcolor[HTML]{8cb9e3}{ 4.92$^{\square}$} & \cellcolor[HTML]{8cb9e3}{ 4.83$^{\square}$} & \cellcolor[HTML]{8cb9e3}{ 4.69$^{\square}$} & \cellcolor[HTML]{8cb9e3}{ 4.75$^{\square}$} & \cellcolor[HTML]{8cb9e3}{4.71$^{\square}$}\\
\cmidrule{1-14}\pagebreak[0]
 & 5 & \cellcolor[HTML]{8cb9e3}{20.58} & \cellcolor[HTML]{8cb9e3}{20.64} & \cellcolor[HTML]{8cb9e3}{20.76} & \cellcolor[HTML]{8cb9e3}{20.86} & \cellcolor[HTML]{8cb9e3}{20.90} & \cellcolor[HTML]{8cb9e3}{20.94} & \cellcolor[HTML]{8cb9e3}{11.12$^{\square}$} & \cellcolor[HTML]{8cb9e3}{ 9.82$^{\square}$} & \cellcolor[HTML]{8cb9e3}{ 8.51$^{\square}$} & \cellcolor[HTML]{8cb9e3}{ 6.86$^{\square}$} & \cellcolor[HTML]{8cb9e3}{ 5.78$^{\square}$} & \cellcolor[HTML]{8cb9e3}{5.17$^{\square}$}\\
\nopagebreak
 & 7 & \cellcolor[HTML]{8cb9e3}{20.55} & \cellcolor[HTML]{8cb9e3}{20.68} & \cellcolor[HTML]{8cb9e3}{20.73} & \cellcolor[HTML]{8cb9e3}{20.87} & \cellcolor[HTML]{8cb9e3}{20.81} & \cellcolor[HTML]{ffffff}{20.78} & \cellcolor[HTML]{8cb9e3}{ 6.68$^{\square}$} & \cellcolor[HTML]{8cb9e3}{ 5.93$^{\square}$} & \cellcolor[HTML]{8cb9e3}{ 5.14$^{\square}$} & \cellcolor[HTML]{8cb9e3}{ 4.74$^{\square}$} & \cellcolor[HTML]{8cb9e3}{ 4.11} & \cellcolor[HTML]{8cb9e3}{4.12}\\
\nopagebreak
 & 9 & \cellcolor[HTML]{8cb9e3}{20.69} & \cellcolor[HTML]{8cb9e3}{20.68} & \cellcolor[HTML]{8cb9e3}{20.69} & \cellcolor[HTML]{8cb9e3}{20.74} & \cellcolor[HTML]{ffffff}{20.70} & \cellcolor[HTML]{ffffff}{20.73} & \cellcolor[HTML]{8cb9e3}{ 5.22$^{\square}$} & \cellcolor[HTML]{8cb9e3}{ 4.77$^{\square}$} & \cellcolor[HTML]{8cb9e3}{ 4.53$^{\square}$} & \cellcolor[HTML]{8cb9e3}{ 4.24} & \cellcolor[HTML]{8cb9e3}{ 4.05} & \cellcolor[HTML]{8cb9e3}{4.05}\\
\nopagebreak
\multirow{-4}{3cm}{\raggedright\arraybackslash Time unstructured (slow response) with definition variables} & 11 & \cellcolor[HTML]{8cb9e3}{20.66} & \cellcolor[HTML]{8cb9e3}{20.77} & \cellcolor[HTML]{8cb9e3}{20.69} & \cellcolor[HTML]{8cb9e3}{20.69} & \cellcolor[HTML]{ffffff}{20.67} & \cellcolor[HTML]{ffffff}{20.69} & \cellcolor[HTML]{8cb9e3}{ 4.79$^{\square}$} & \cellcolor[HTML]{8cb9e3}{ 4.72$^{\square}$} & \cellcolor[HTML]{8cb9e3}{ 4.32} & \cellcolor[HTML]{8cb9e3}{ 4.01} & \cellcolor[HTML]{8cb9e3}{ 4.14} & \cellcolor[HTML]{8cb9e3}{4.11}\\
\bottomrule
\insertTableNotes
\end{longtable}
\end{ThreePartTable}
\addtocounter{table}{-1}
\begin{ThreePartTable}
\begin{TableNotes}
\item 
\end{TableNotes}
\begin{longtable}[l]{>{\raggedright\arraybackslash}p{3cm}>{\raggedright\arraybackslash}p{3cm}cccccccccccc}
\caption[]{Parameter Values Estimated for Day- and Likert-Unit Parameters in Experiment 3 (continued)}\\
\toprule
\multicolumn{1}{c}{} & \multicolumn{1}{c}{} & \multicolumn{6}{c}{\thead{$\uptheta_{fixed}$ (Baseline) \\ Pop value = 3.00}} & \multicolumn{6}{c}{\thead{$\uptheta_{random}$ (Baseline) \\ Pop value = 0.05}} \\
\cmidrule(l{3pt}r{3pt}){3-8} \cmidrule(l{3pt}r{3pt}){9-14}
Time Structuredness & Number of Measurements & 30 & 50 & 100 & 200 & 500 & 1000 & 30 & 50 & 100 & 200 & 500 & 1000\\
\midrule
 & 5 & \cellcolor[HTML]{ffffff}{3.00} & \cellcolor[HTML]{ffffff}{3.00} & \cellcolor[HTML]{ffffff}{3.00} & \cellcolor[HTML]{ffffff}{3.00} & \cellcolor[HTML]{ffffff}{3.00} & \cellcolor[HTML]{ffffff}{3.00} & \cellcolor[HTML]{8cb9e3}{0.05} & \cellcolor[HTML]{8cb9e3}{0.05} & \cellcolor[HTML]{8cb9e3}{0.05} & \cellcolor[HTML]{8cb9e3}{0.05} & \cellcolor[HTML]{ffffff}{0.05} & \cellcolor[HTML]{ffffff}{0.05}\\
\nopagebreak
 & 7 & \cellcolor[HTML]{ffffff}{3.00} & \cellcolor[HTML]{ffffff}{3.00} & \cellcolor[HTML]{ffffff}{3.00} & \cellcolor[HTML]{ffffff}{3.00} & \cellcolor[HTML]{ffffff}{3.00} & \cellcolor[HTML]{ffffff}{3.00} & \cellcolor[HTML]{8cb9e3}{0.05} & \cellcolor[HTML]{8cb9e3}{0.05} & \cellcolor[HTML]{8cb9e3}{0.05} & \cellcolor[HTML]{8cb9e3}{0.05} & \cellcolor[HTML]{ffffff}{0.05} & \cellcolor[HTML]{ffffff}{0.05}\\
\nopagebreak
 & 9 & \cellcolor[HTML]{ffffff}{3.00} & \cellcolor[HTML]{ffffff}{3.00} & \cellcolor[HTML]{ffffff}{3.00} & \cellcolor[HTML]{ffffff}{3.00} & \cellcolor[HTML]{ffffff}{3.00} & \cellcolor[HTML]{ffffff}{3.00} & \cellcolor[HTML]{8cb9e3}{0.05} & \cellcolor[HTML]{8cb9e3}{0.05} & \cellcolor[HTML]{8cb9e3}{0.05} & \cellcolor[HTML]{8cb9e3}{0.05} & \cellcolor[HTML]{ffffff}{0.05} & \cellcolor[HTML]{ffffff}{0.05}\\
\nopagebreak
\multirow{-4}{3cm}{\raggedright\arraybackslash Time structured} & 11 & \cellcolor[HTML]{ffffff}{3.00} & \cellcolor[HTML]{ffffff}{3.00} & \cellcolor[HTML]{ffffff}{3.00} & \cellcolor[HTML]{ffffff}{3.00} & \cellcolor[HTML]{ffffff}{3.00} & \cellcolor[HTML]{ffffff}{3.00} & \cellcolor[HTML]{8cb9e3}{0.05} & \cellcolor[HTML]{8cb9e3}{0.05} & \cellcolor[HTML]{8cb9e3}{0.05} & \cellcolor[HTML]{8cb9e3}{0.05} & \cellcolor[HTML]{ffffff}{0.05} & \cellcolor[HTML]{ffffff}{0.05}\\
\cmidrule{1-14}\pagebreak[0]
 & 5 & \cellcolor[HTML]{ffffff}{3.00} & \cellcolor[HTML]{ffffff}{3.00} & \cellcolor[HTML]{ffffff}{3.00} & \cellcolor[HTML]{ffffff}{3.00} & \cellcolor[HTML]{ffffff}{3.00} & \cellcolor[HTML]{ffffff}{3.00} & \cellcolor[HTML]{8cb9e3}{0.05} & \cellcolor[HTML]{8cb9e3}{0.05} & \cellcolor[HTML]{8cb9e3}{0.05} & \cellcolor[HTML]{8cb9e3}{0.05} & \cellcolor[HTML]{ffffff}{0.05} & \cellcolor[HTML]{ffffff}{0.05}\\
\nopagebreak
 & 7 & \cellcolor[HTML]{ffffff}{3.00} & \cellcolor[HTML]{ffffff}{3.00} & \cellcolor[HTML]{ffffff}{3.00} & \cellcolor[HTML]{ffffff}{3.00} & \cellcolor[HTML]{ffffff}{3.00} & \cellcolor[HTML]{ffffff}{3.00} & \cellcolor[HTML]{8cb9e3}{0.05} & \cellcolor[HTML]{8cb9e3}{0.05} & \cellcolor[HTML]{8cb9e3}{0.05} & \cellcolor[HTML]{8cb9e3}{0.05} & \cellcolor[HTML]{ffffff}{0.05} & \cellcolor[HTML]{ffffff}{0.05}\\
\nopagebreak
 & 9 & \cellcolor[HTML]{ffffff}{3.00} & \cellcolor[HTML]{ffffff}{3.00} & \cellcolor[HTML]{ffffff}{3.00} & \cellcolor[HTML]{ffffff}{3.00} & \cellcolor[HTML]{ffffff}{3.00} & \cellcolor[HTML]{ffffff}{3.00} & \cellcolor[HTML]{8cb9e3}{0.05} & \cellcolor[HTML]{8cb9e3}{0.05} & \cellcolor[HTML]{8cb9e3}{0.05} & \cellcolor[HTML]{8cb9e3}{0.05} & \cellcolor[HTML]{ffffff}{0.05} & \cellcolor[HTML]{ffffff}{0.05}\\
\nopagebreak
\multirow{-4}{3cm}{\raggedright\arraybackslash Time unstructured (fast response)} & 11 & \cellcolor[HTML]{ffffff}{3.00} & \cellcolor[HTML]{ffffff}{3.00} & \cellcolor[HTML]{ffffff}{3.00} & \cellcolor[HTML]{ffffff}{3.00} & \cellcolor[HTML]{ffffff}{3.00} & \cellcolor[HTML]{ffffff}{3.00} & \cellcolor[HTML]{8cb9e3}{0.05} & \cellcolor[HTML]{8cb9e3}{0.05} & \cellcolor[HTML]{8cb9e3}{0.05} & \cellcolor[HTML]{8cb9e3}{0.05} & \cellcolor[HTML]{ffffff}{0.05} & \cellcolor[HTML]{ffffff}{0.05}\\
\cmidrule{1-14}\pagebreak[0]
 & 5 & \cellcolor[HTML]{ffffff}{3.00} & \cellcolor[HTML]{ffffff}{3.00} & \cellcolor[HTML]{ffffff}{3.00} & \cellcolor[HTML]{ffffff}{3.00} & \cellcolor[HTML]{ffffff}{3.00} & \cellcolor[HTML]{ffffff}{3.00} & \cellcolor[HTML]{8cb9e3}{0.05} & \cellcolor[HTML]{8cb9e3}{0.05} & \cellcolor[HTML]{8cb9e3}{0.05} & \cellcolor[HTML]{8cb9e3}{0.05} & \cellcolor[HTML]{8cb9e3}{0.05} & \cellcolor[HTML]{ffffff}{0.05}\\
\nopagebreak
 & 7 & \cellcolor[HTML]{ffffff}{3.00} & \cellcolor[HTML]{ffffff}{3.00} & \cellcolor[HTML]{ffffff}{3.00} & \cellcolor[HTML]{ffffff}{3.00} & \cellcolor[HTML]{ffffff}{3.00} & \cellcolor[HTML]{ffffff}{3.00} & \cellcolor[HTML]{8cb9e3}{0.05} & \cellcolor[HTML]{8cb9e3}{0.05} & \cellcolor[HTML]{8cb9e3}{0.05} & \cellcolor[HTML]{8cb9e3}{0.05} & \cellcolor[HTML]{ffffff}{0.05} & \cellcolor[HTML]{ffffff}{0.05}\\
\nopagebreak
 & 9 & \cellcolor[HTML]{ffffff}{3.00} & \cellcolor[HTML]{ffffff}{3.00} & \cellcolor[HTML]{ffffff}{3.00} & \cellcolor[HTML]{ffffff}{3.00} & \cellcolor[HTML]{ffffff}{3.00} & \cellcolor[HTML]{ffffff}{3.00} & \cellcolor[HTML]{8cb9e3}{0.05} & \cellcolor[HTML]{8cb9e3}{0.05} & \cellcolor[HTML]{8cb9e3}{0.05} & \cellcolor[HTML]{8cb9e3}{0.05} & \cellcolor[HTML]{ffffff}{0.05} & \cellcolor[HTML]{ffffff}{0.05}\\
\nopagebreak
\multirow{-4}{3cm}{\raggedright\arraybackslash Time unstructured (slow response)} & 11 & \cellcolor[HTML]{ffffff}{3.00} & \cellcolor[HTML]{ffffff}{3.00} & \cellcolor[HTML]{ffffff}{3.00} & \cellcolor[HTML]{ffffff}{3.00} & \cellcolor[HTML]{ffffff}{3.00} & \cellcolor[HTML]{ffffff}{3.00} & \cellcolor[HTML]{8cb9e3}{0.05} & \cellcolor[HTML]{8cb9e3}{0.05} & \cellcolor[HTML]{8cb9e3}{0.05} & \cellcolor[HTML]{8cb9e3}{0.05} & \cellcolor[HTML]{ffffff}{0.05} & \cellcolor[HTML]{ffffff}{0.05}\\
\cmidrule{1-14}\pagebreak[0]
 & 5 & \cellcolor[HTML]{ffffff}{3.00} & \cellcolor[HTML]{ffffff}{3.00} & \cellcolor[HTML]{ffffff}{3.00} & \cellcolor[HTML]{ffffff}{3.00} & \cellcolor[HTML]{ffffff}{3.00} & \cellcolor[HTML]{ffffff}{3.00} & \cellcolor[HTML]{8cb9e3}{0.05} & \cellcolor[HTML]{8cb9e3}{0.05} & \cellcolor[HTML]{8cb9e3}{0.05} & \cellcolor[HTML]{8cb9e3}{0.05} & \cellcolor[HTML]{ffffff}{0.05} & \cellcolor[HTML]{ffffff}{0.05}\\
\nopagebreak
 & 7 & \cellcolor[HTML]{ffffff}{3.00} & \cellcolor[HTML]{ffffff}{3.00} & \cellcolor[HTML]{ffffff}{3.00} & \cellcolor[HTML]{ffffff}{3.00} & \cellcolor[HTML]{ffffff}{3.00} & \cellcolor[HTML]{ffffff}{3.00} & \cellcolor[HTML]{8cb9e3}{0.05} & \cellcolor[HTML]{8cb9e3}{0.05} & \cellcolor[HTML]{8cb9e3}{0.05} & \cellcolor[HTML]{8cb9e3}{0.05} & \cellcolor[HTML]{ffffff}{0.05} & \cellcolor[HTML]{ffffff}{0.05}\\
\nopagebreak
 & 9 & \cellcolor[HTML]{ffffff}{3.00} & \cellcolor[HTML]{ffffff}{3.00} & \cellcolor[HTML]{ffffff}{3.00} & \cellcolor[HTML]{ffffff}{3.00} & \cellcolor[HTML]{ffffff}{3.00} & \cellcolor[HTML]{ffffff}{3.00} & \cellcolor[HTML]{8cb9e3}{0.05} & \cellcolor[HTML]{8cb9e3}{0.05} & \cellcolor[HTML]{8cb9e3}{0.05} & \cellcolor[HTML]{8cb9e3}{0.05} & \cellcolor[HTML]{ffffff}{0.05} & \cellcolor[HTML]{ffffff}{0.05}\\
\nopagebreak
\multirow{-4}{3cm}{\raggedright\arraybackslash Time unstructured (slow response) with definition variables} & 11 & \cellcolor[HTML]{ffffff}{3.00} & \cellcolor[HTML]{ffffff}{3.00} & \cellcolor[HTML]{ffffff}{3.00} & \cellcolor[HTML]{ffffff}{3.00} & \cellcolor[HTML]{ffffff}{3.00} & \cellcolor[HTML]{ffffff}{3.00} & \cellcolor[HTML]{8cb9e3}{0.05} & \cellcolor[HTML]{8cb9e3}{0.05} & \cellcolor[HTML]{8cb9e3}{0.05} & \cellcolor[HTML]{8cb9e3}{0.05} & \cellcolor[HTML]{ffffff}{0.05} & \cellcolor[HTML]{ffffff}{0.05}\\
\bottomrule
\insertTableNotes
\end{longtable}
\end{ThreePartTable}
\addtocounter{table}{-1}
\begin{ThreePartTable}
\begin{TableNotes}
\item 
\end{TableNotes}
\begin{longtable}[l]{>{\raggedright\arraybackslash}p{3cm}>{\raggedright\arraybackslash}p{3cm}cccccccccccc}
\caption[]{Parameter Values Estimated for Day- and Likert-Unit Parameters in Experiment 3 (continued)}\\
\toprule
\multicolumn{1}{c}{} & \multicolumn{1}{c}{} & \multicolumn{6}{c}{\thead{$\upalpha_{fixed}$ (Maximal elevation) \\ Pop value = 3.32}} & \multicolumn{6}{c}{\thead{$\upalpha_{random}$ (Maximal elevation) \\ Pop value = 0.05}} \\
\cmidrule(l{3pt}r{3pt}){3-8} \cmidrule(l{3pt}r{3pt}){9-14}
Time Structuredness & Number of Measurements & 30 & 50 & 100 & 200 & 500 & 1000 & 30 & 50 & 100 & 200 & 500 & 1000\\
\midrule
 & 5 & \cellcolor[HTML]{ffffff}{3.32} & \cellcolor[HTML]{ffffff}{3.32} & \cellcolor[HTML]{ffffff}{3.32} & \cellcolor[HTML]{ffffff}{3.32} & \cellcolor[HTML]{ffffff}{3.32} & \cellcolor[HTML]{ffffff}{3.32} & \cellcolor[HTML]{8cb9e3}{0.05} & \cellcolor[HTML]{8cb9e3}{0.05} & \cellcolor[HTML]{8cb9e3}{0.05} & \cellcolor[HTML]{8cb9e3}{0.05} & \cellcolor[HTML]{ffffff}{0.05} & \cellcolor[HTML]{ffffff}{0.05}\\
\nopagebreak
 & 7 & \cellcolor[HTML]{ffffff}{3.32} & \cellcolor[HTML]{ffffff}{3.32} & \cellcolor[HTML]{ffffff}{3.32} & \cellcolor[HTML]{ffffff}{3.32} & \cellcolor[HTML]{ffffff}{3.32} & \cellcolor[HTML]{ffffff}{3.32} & \cellcolor[HTML]{8cb9e3}{0.05} & \cellcolor[HTML]{8cb9e3}{0.05} & \cellcolor[HTML]{8cb9e3}{0.05} & \cellcolor[HTML]{8cb9e3}{0.05} & \cellcolor[HTML]{ffffff}{0.05} & \cellcolor[HTML]{ffffff}{0.05}\\
\nopagebreak
 & 9 & \cellcolor[HTML]{ffffff}{3.32} & \cellcolor[HTML]{ffffff}{3.32} & \cellcolor[HTML]{ffffff}{3.32} & \cellcolor[HTML]{ffffff}{3.32} & \cellcolor[HTML]{ffffff}{3.32} & \cellcolor[HTML]{ffffff}{3.32} & \cellcolor[HTML]{8cb9e3}{0.05} & \cellcolor[HTML]{8cb9e3}{0.05} & \cellcolor[HTML]{8cb9e3}{0.05} & \cellcolor[HTML]{8cb9e3}{0.05} & \cellcolor[HTML]{ffffff}{0.05} & \cellcolor[HTML]{ffffff}{0.05}\\
\nopagebreak
\multirow{-4}{3cm}{\raggedright\arraybackslash Time structured} & 11 & \cellcolor[HTML]{ffffff}{3.32} & \cellcolor[HTML]{ffffff}{3.32} & \cellcolor[HTML]{ffffff}{3.32} & \cellcolor[HTML]{ffffff}{3.32} & \cellcolor[HTML]{ffffff}{3.32} & \cellcolor[HTML]{ffffff}{3.32} & \cellcolor[HTML]{8cb9e3}{0.05} & \cellcolor[HTML]{8cb9e3}{0.05} & \cellcolor[HTML]{8cb9e3}{0.05} & \cellcolor[HTML]{8cb9e3}{0.05} & \cellcolor[HTML]{ffffff}{0.05} & \cellcolor[HTML]{ffffff}{0.05}\\
\cmidrule{1-14}\pagebreak[0]
 & 5 & \cellcolor[HTML]{ffffff}{3.32} & \cellcolor[HTML]{ffffff}{3.32} & \cellcolor[HTML]{ffffff}{3.32} & \cellcolor[HTML]{ffffff}{3.32} & \cellcolor[HTML]{ffffff}{3.32} & \cellcolor[HTML]{ffffff}{3.32} & \cellcolor[HTML]{8cb9e3}{0.05} & \cellcolor[HTML]{8cb9e3}{0.05} & \cellcolor[HTML]{8cb9e3}{0.05} & \cellcolor[HTML]{8cb9e3}{0.05} & \cellcolor[HTML]{ffffff}{0.05} & \cellcolor[HTML]{ffffff}{0.05}\\
\nopagebreak
 & 7 & \cellcolor[HTML]{ffffff}{3.32} & \cellcolor[HTML]{ffffff}{3.32} & \cellcolor[HTML]{ffffff}{3.32} & \cellcolor[HTML]{ffffff}{3.32} & \cellcolor[HTML]{ffffff}{3.32} & \cellcolor[HTML]{ffffff}{3.32} & \cellcolor[HTML]{8cb9e3}{0.05} & \cellcolor[HTML]{8cb9e3}{0.05} & \cellcolor[HTML]{8cb9e3}{0.05} & \cellcolor[HTML]{8cb9e3}{0.05} & \cellcolor[HTML]{ffffff}{0.05} & \cellcolor[HTML]{ffffff}{0.05}\\
\nopagebreak
 & 9 & \cellcolor[HTML]{ffffff}{3.32} & \cellcolor[HTML]{ffffff}{3.32} & \cellcolor[HTML]{ffffff}{3.32} & \cellcolor[HTML]{ffffff}{3.32} & \cellcolor[HTML]{ffffff}{3.32} & \cellcolor[HTML]{ffffff}{3.32} & \cellcolor[HTML]{8cb9e3}{0.05} & \cellcolor[HTML]{8cb9e3}{0.05} & \cellcolor[HTML]{8cb9e3}{0.05} & \cellcolor[HTML]{8cb9e3}{0.05} & \cellcolor[HTML]{ffffff}{0.05} & \cellcolor[HTML]{ffffff}{0.05}\\
\nopagebreak
\multirow{-4}{3cm}{\raggedright\arraybackslash Time unstructured (fast response)} & 11 & \cellcolor[HTML]{ffffff}{3.32} & \cellcolor[HTML]{ffffff}{3.32} & \cellcolor[HTML]{ffffff}{3.32} & \cellcolor[HTML]{ffffff}{3.32} & \cellcolor[HTML]{ffffff}{3.32} & \cellcolor[HTML]{ffffff}{3.32} & \cellcolor[HTML]{8cb9e3}{0.05} & \cellcolor[HTML]{8cb9e3}{0.05} & \cellcolor[HTML]{8cb9e3}{0.05} & \cellcolor[HTML]{8cb9e3}{0.05} & \cellcolor[HTML]{ffffff}{0.05} & \cellcolor[HTML]{ffffff}{0.05}\\
\cmidrule{1-14}\pagebreak[0]
 & 5 & \cellcolor[HTML]{ffffff}{3.32} & \cellcolor[HTML]{ffffff}{3.32} & \cellcolor[HTML]{ffffff}{3.32} & \cellcolor[HTML]{ffffff}{3.32} & \cellcolor[HTML]{ffffff}{3.32} & \cellcolor[HTML]{ffffff}{3.32} & \cellcolor[HTML]{8cb9e3}{0.05} & \cellcolor[HTML]{8cb9e3}{0.05} & \cellcolor[HTML]{8cb9e3}{0.05} & \cellcolor[HTML]{8cb9e3}{0.05} & \cellcolor[HTML]{ffffff}{0.05} & \cellcolor[HTML]{ffffff}{0.05}\\
\nopagebreak
 & 7 & \cellcolor[HTML]{ffffff}{3.32} & \cellcolor[HTML]{ffffff}{3.32} & \cellcolor[HTML]{ffffff}{3.32} & \cellcolor[HTML]{ffffff}{3.32} & \cellcolor[HTML]{ffffff}{3.32} & \cellcolor[HTML]{ffffff}{3.32} & \cellcolor[HTML]{8cb9e3}{0.05} & \cellcolor[HTML]{8cb9e3}{0.05} & \cellcolor[HTML]{8cb9e3}{0.05} & \cellcolor[HTML]{8cb9e3}{0.05} & \cellcolor[HTML]{ffffff}{0.05} & \cellcolor[HTML]{ffffff}{0.05}\\
\nopagebreak
 & 9 & \cellcolor[HTML]{ffffff}{3.32} & \cellcolor[HTML]{ffffff}{3.32} & \cellcolor[HTML]{ffffff}{3.32} & \cellcolor[HTML]{ffffff}{3.32} & \cellcolor[HTML]{ffffff}{3.32} & \cellcolor[HTML]{ffffff}{3.32} & \cellcolor[HTML]{8cb9e3}{0.05} & \cellcolor[HTML]{8cb9e3}{0.05} & \cellcolor[HTML]{8cb9e3}{0.05} & \cellcolor[HTML]{8cb9e3}{0.05} & \cellcolor[HTML]{ffffff}{0.05} & \cellcolor[HTML]{ffffff}{0.05}\\
\nopagebreak
\multirow{-4}{3cm}{\raggedright\arraybackslash Time unstructured (slow response)} & 11 & \cellcolor[HTML]{ffffff}{3.32} & \cellcolor[HTML]{ffffff}{3.32} & \cellcolor[HTML]{ffffff}{3.32} & \cellcolor[HTML]{ffffff}{3.32} & \cellcolor[HTML]{ffffff}{3.32} & \cellcolor[HTML]{ffffff}{3.32} & \cellcolor[HTML]{8cb9e3}{0.05} & \cellcolor[HTML]{8cb9e3}{0.05} & \cellcolor[HTML]{8cb9e3}{0.05} & \cellcolor[HTML]{8cb9e3}{0.05} & \cellcolor[HTML]{ffffff}{0.05} & \cellcolor[HTML]{ffffff}{0.05}\\
\cmidrule{1-14}\pagebreak[0]
 & 5 & \cellcolor[HTML]{ffffff}{3.32} & \cellcolor[HTML]{ffffff}{3.32} & \cellcolor[HTML]{ffffff}{3.32} & \cellcolor[HTML]{ffffff}{3.32} & \cellcolor[HTML]{ffffff}{3.32} & \cellcolor[HTML]{ffffff}{3.32} & \cellcolor[HTML]{8cb9e3}{0.05} & \cellcolor[HTML]{8cb9e3}{0.05} & \cellcolor[HTML]{8cb9e3}{0.05} & \cellcolor[HTML]{8cb9e3}{0.05} & \cellcolor[HTML]{ffffff}{0.05} & \cellcolor[HTML]{ffffff}{0.05}\\
\nopagebreak
 & 7 & \cellcolor[HTML]{ffffff}{3.32} & \cellcolor[HTML]{ffffff}{3.32} & \cellcolor[HTML]{ffffff}{3.32} & \cellcolor[HTML]{ffffff}{3.32} & \cellcolor[HTML]{ffffff}{3.32} & \cellcolor[HTML]{ffffff}{3.32} & \cellcolor[HTML]{8cb9e3}{0.05} & \cellcolor[HTML]{8cb9e3}{0.05} & \cellcolor[HTML]{8cb9e3}{0.05} & \cellcolor[HTML]{8cb9e3}{0.05} & \cellcolor[HTML]{ffffff}{0.05} & \cellcolor[HTML]{ffffff}{0.05}\\
\nopagebreak
 & 9 & \cellcolor[HTML]{ffffff}{3.32} & \cellcolor[HTML]{ffffff}{3.32} & \cellcolor[HTML]{ffffff}{3.32} & \cellcolor[HTML]{ffffff}{3.32} & \cellcolor[HTML]{ffffff}{3.32} & \cellcolor[HTML]{ffffff}{3.32} & \cellcolor[HTML]{8cb9e3}{0.05} & \cellcolor[HTML]{8cb9e3}{0.05} & \cellcolor[HTML]{8cb9e3}{0.05} & \cellcolor[HTML]{8cb9e3}{0.05} & \cellcolor[HTML]{ffffff}{0.05} & \cellcolor[HTML]{ffffff}{0.05}\\
\nopagebreak
\multirow{-4}{3cm}{\raggedright\arraybackslash Time unstructured (slow response) with definition variables} & 11 & \cellcolor[HTML]{ffffff}{3.32} & \cellcolor[HTML]{ffffff}{3.32} & \cellcolor[HTML]{ffffff}{3.32} & \cellcolor[HTML]{ffffff}{3.32} & \cellcolor[HTML]{ffffff}{3.32} & \cellcolor[HTML]{ffffff}{3.32} & \cellcolor[HTML]{8cb9e3}{0.05} & \cellcolor[HTML]{8cb9e3}{0.05} & \cellcolor[HTML]{8cb9e3}{0.05} & \cellcolor[HTML]{8cb9e3}{0.05} & \cellcolor[HTML]{ffffff}{0.05} & \cellcolor[HTML]{ffffff}{0.05}\\
\bottomrule
\insertTableNotes
\end{longtable}
\end{ThreePartTable}
\addtocounter{table}{-1}

\begin{ThreePartTable}
\begin{TableNotes}
\item \textit{Note. }Cells shaded in light blue indicate cells where estimation is imprecise (i.e., lower and/or upper whisker lengths exceeding 10\% of the parameter's population value. Empty superscript squares ($^{\square}$) indicate biased estimates (i.e., bias exceeding 10\% of parameter's population value).
\end{TableNotes}
\begin{longtable}[l]{>{\raggedright\arraybackslash}p{3cm}>{\raggedright\arraybackslash}p{3cm}cccccc}
\caption[]{Parameter Values Estimated for Day- and Likert-Unit Parameters in Experiment 3 (continued)}\\
\toprule
\multicolumn{1}{c}{} & \multicolumn{1}{c}{} & \multicolumn{6}{c}{\thead{$\upepsilon$(error) \\ Pop value = 0.03}} \\
\cmidrule(l{3pt}r{3pt}){3-8}
Time Structuredness & Number of Measurements & 30 & 50 & 100 & 200 & 500 & 1000\\
\midrule
 & 5 & \cellcolor[HTML]{8cb9e3}{0.05} & \cellcolor[HTML]{8cb9e3}{0.05} & \cellcolor[HTML]{8cb9e3}{0.05} & \cellcolor[HTML]{ffffff}{0.05} & \cellcolor[HTML]{ffffff}{0.05} & \cellcolor[HTML]{ffffff}{0.05}\\
\nopagebreak
 & 7 & \cellcolor[HTML]{8cb9e3}{0.05} & \cellcolor[HTML]{ffffff}{0.05} & \cellcolor[HTML]{ffffff}{0.05} & \cellcolor[HTML]{ffffff}{0.05} & \cellcolor[HTML]{ffffff}{0.05} & \cellcolor[HTML]{ffffff}{0.05}\\
\nopagebreak
 & 9 & \cellcolor[HTML]{8cb9e3}{0.05} & \cellcolor[HTML]{ffffff}{0.05} & \cellcolor[HTML]{ffffff}{0.05} & \cellcolor[HTML]{ffffff}{0.05} & \cellcolor[HTML]{ffffff}{0.05} & \cellcolor[HTML]{ffffff}{0.05}\\
\nopagebreak
\multirow{-4}{3cm}{\raggedright\arraybackslash Time structured} & 11 & \cellcolor[HTML]{ffffff}{0.05} & \cellcolor[HTML]{ffffff}{0.05} & \cellcolor[HTML]{ffffff}{0.05} & \cellcolor[HTML]{ffffff}{0.05} & \cellcolor[HTML]{ffffff}{0.05} & \cellcolor[HTML]{ffffff}{0.05}\\
\cmidrule{1-8}\pagebreak[0]
 & 5 & \cellcolor[HTML]{8cb9e3}{0.05} & \cellcolor[HTML]{8cb9e3}{0.05} & \cellcolor[HTML]{8cb9e3}{0.05} & \cellcolor[HTML]{ffffff}{0.05} & \cellcolor[HTML]{ffffff}{0.05} & \cellcolor[HTML]{ffffff}{0.05}\\
\nopagebreak
 & 7 & \cellcolor[HTML]{8cb9e3}{0.05} & \cellcolor[HTML]{8cb9e3}{0.05} & \cellcolor[HTML]{ffffff}{0.05} & \cellcolor[HTML]{ffffff}{0.05} & \cellcolor[HTML]{ffffff}{0.05} & \cellcolor[HTML]{ffffff}{0.05}\\
\nopagebreak
 & 9 & \cellcolor[HTML]{8cb9e3}{0.05} & \cellcolor[HTML]{ffffff}{0.05} & \cellcolor[HTML]{ffffff}{0.05} & \cellcolor[HTML]{ffffff}{0.05} & \cellcolor[HTML]{ffffff}{0.05} & \cellcolor[HTML]{ffffff}{0.05}\\
\nopagebreak
\multirow{-4}{3cm}{\raggedright\arraybackslash Time unstructured (fast response)} & 11 & \cellcolor[HTML]{ffffff}{0.05} & \cellcolor[HTML]{ffffff}{0.05} & \cellcolor[HTML]{ffffff}{0.05} & \cellcolor[HTML]{ffffff}{0.05} & \cellcolor[HTML]{ffffff}{0.05} & \cellcolor[HTML]{ffffff}{0.05}\\
\cmidrule{1-8}\pagebreak[0]
 & 5 & \cellcolor[HTML]{8cb9e3}{0.05} & \cellcolor[HTML]{8cb9e3}{0.05} & \cellcolor[HTML]{8cb9e3}{0.05} & \cellcolor[HTML]{ffffff}{0.05} & \cellcolor[HTML]{ffffff}{0.05} & \cellcolor[HTML]{ffffff}{0.05}\\
\nopagebreak
 & 7 & \cellcolor[HTML]{8cb9e3}{0.05} & \cellcolor[HTML]{ffffff}{0.05} & \cellcolor[HTML]{ffffff}{0.05} & \cellcolor[HTML]{ffffff}{0.05} & \cellcolor[HTML]{ffffff}{0.05} & \cellcolor[HTML]{ffffff}{0.05}\\
\nopagebreak
 & 9 & \cellcolor[HTML]{8cb9e3}{0.05} & \cellcolor[HTML]{ffffff}{0.05} & \cellcolor[HTML]{ffffff}{0.05} & \cellcolor[HTML]{ffffff}{0.05} & \cellcolor[HTML]{ffffff}{0.05} & \cellcolor[HTML]{ffffff}{0.05}\\
\nopagebreak
\multirow{-4}{3cm}{\raggedright\arraybackslash Time unstructured (slow response)} & 11 & \cellcolor[HTML]{8cb9e3}{0.05} & \cellcolor[HTML]{ffffff}{0.05} & \cellcolor[HTML]{ffffff}{0.05} & \cellcolor[HTML]{ffffff}{0.05} & \cellcolor[HTML]{ffffff}{0.05} & \cellcolor[HTML]{ffffff}{0.05}\\
\cmidrule{1-8}\pagebreak[0]
 & 5 & \cellcolor[HTML]{8cb9e3}{0.05} & \cellcolor[HTML]{8cb9e3}{0.05} & \cellcolor[HTML]{8cb9e3}{0.05} & \cellcolor[HTML]{ffffff}{0.05} & \cellcolor[HTML]{ffffff}{0.05} & \cellcolor[HTML]{ffffff}{0.05}\\
\nopagebreak
 & 7 & \cellcolor[HTML]{8cb9e3}{0.05} & \cellcolor[HTML]{8cb9e3}{0.05} & \cellcolor[HTML]{ffffff}{0.05} & \cellcolor[HTML]{ffffff}{0.05} & \cellcolor[HTML]{ffffff}{0.05} & \cellcolor[HTML]{ffffff}{0.05}\\
\nopagebreak
 & 9 & \cellcolor[HTML]{8cb9e3}{0.05} & \cellcolor[HTML]{ffffff}{0.05} & \cellcolor[HTML]{ffffff}{0.05} & \cellcolor[HTML]{ffffff}{0.05} & \cellcolor[HTML]{ffffff}{0.05} & \cellcolor[HTML]{ffffff}{0.05}\\
\nopagebreak
\multirow{-4}{3cm}{\raggedright\arraybackslash Time unstructured (slow response) with definition variables} & 11 & \cellcolor[HTML]{8cb9e3}{0.05} & \cellcolor[HTML]{ffffff}{0.05} & \cellcolor[HTML]{ffffff}{0.05} & \cellcolor[HTML]{ffffff}{0.05} & \cellcolor[HTML]{ffffff}{0.05} & \cellcolor[HTML]{ffffff}{0.05}\\
\bottomrule
\insertTableNotes
\end{longtable}
\end{ThreePartTable}
\end{landscape}
\restoregeometry

\app{OpenMx Code for Structured Latent Growth Curve Model With Definition Variables}

\label{def-model-code}

The code that I used to model logistic pattern of change using definition variables (see \protect\hyperlink{definition-variables}{definition variables}) is shown in Code Block \ref{definition-model}. Note that, the code is largely excerpted from the \texttt{run\_exp\_simulations()} and \texttt{create\_definition\_model()} functions from the \texttt{nonlinSims} package, and so readers interested in obtaining more information should consult the source code of this package. One important point to mention is that the model specified in Code Block \ref{definition-model} can accurately model time-unstructured data because it uses definition variables.

\captionof{chunk}{OpenMx Code for Structured Latent Growth Curve Model With Definition Variables}\restoreparindent\label{definition-model}
\begin{Shaded}
\begin{Highlighting}[numbers=left,,]
\CommentTok{\#Now convert data to wide format (needed for OpenMx)}
\NormalTok{data\_wide }\OtherTok{\textless{}{-}}\NormalTok{ data[ , }\FunctionTok{c}\NormalTok{(}\DecValTok{1}\SpecialCharTok{:}\DecValTok{3}\NormalTok{, }\DecValTok{5}\NormalTok{)] }\SpecialCharTok{\%\textgreater{}\%} 
    \FunctionTok{pivot\_wider}\NormalTok{(}\AttributeTok{names\_from =}\NormalTok{ measurement\_day, }\AttributeTok{values\_from =} \FunctionTok{c}\NormalTok{(obs\_score, actual\_measurement\_day))}

\CommentTok{\#Definition variable (data. prefix tells OpenMx to use recorded time of observation for each person\textquotesingle{}s data)}
\NormalTok{obs\_score\_days }\OtherTok{\textless{}{-}} \FunctionTok{paste}\NormalTok{(}\StringTok{\textquotesingle{}data.\textquotesingle{}}\NormalTok{, }\FunctionTok{extract\_obs\_score\_days}\NormalTok{(}\AttributeTok{data =}\NormalTok{ data\_wide), }\AttributeTok{sep =} \StringTok{\textquotesingle{}\textquotesingle{}}\NormalTok{) }

\CommentTok{\#Remove . from column names so that OpenMx does not run into error (this occurs because, with some spacing schedules, measurement days are not integer values.) }
\FunctionTok{names}\NormalTok{(data\_wide) }\OtherTok{\textless{}{-}} \FunctionTok{str\_replace}\NormalTok{(}\AttributeTok{string =} \FunctionTok{names}\NormalTok{(data\_wide), }\AttributeTok{pattern =} \StringTok{\textquotesingle{}}\SpecialCharTok{\textbackslash{}\textbackslash{}}\StringTok{.\textquotesingle{}}\NormalTok{, }\AttributeTok{replacement =} \StringTok{\textquotesingle{}\_\textquotesingle{}}\NormalTok{)}

\CommentTok{\#Latent variable names (theta = baseline, alpha = maximal elevation, beta = days{-}to{-}halfway elevation, gamma = triquarter{-}halfway elevation)}
\NormalTok{latent\_vars }\OtherTok{\textless{}{-}} \FunctionTok{c}\NormalTok{(}\StringTok{\textquotesingle{}theta\textquotesingle{}}\NormalTok{, }\StringTok{\textquotesingle{}alpha\textquotesingle{}}\NormalTok{, }\StringTok{\textquotesingle{}beta\textquotesingle{}}\NormalTok{, }\StringTok{\textquotesingle{}gamma\textquotesingle{}}\NormalTok{) }

\NormalTok{def\_growth\_curve\_model }\OtherTok{\textless{}{-}} \FunctionTok{mxModel}\NormalTok{(}
  \AttributeTok{model =}\NormalTok{ model\_name,}
  \AttributeTok{type =} \StringTok{\textquotesingle{}RAM\textquotesingle{}}\NormalTok{, }\AttributeTok{independent =}\NormalTok{ T,}
  \FunctionTok{mxData}\NormalTok{(}\AttributeTok{observed =}\NormalTok{ data\_wide, }\AttributeTok{type =} \StringTok{\textquotesingle{}raw\textquotesingle{}}\NormalTok{),}
  
  \AttributeTok{manifestVars =}\NormalTok{ manifest\_vars,}
  \AttributeTok{latentVars =}\NormalTok{ latent\_vars,}
  
  \CommentTok{\#Residual variances; by using one label, they are assumed to all be equal (homogeneity of variance). That is, there is no complex error structure. }
  \FunctionTok{mxPath}\NormalTok{(}\AttributeTok{from =}\NormalTok{ manifest\_vars,}
         \AttributeTok{arrows=}\DecValTok{2}\NormalTok{, }\AttributeTok{free=}\ConstantTok{TRUE}\NormalTok{,  }\AttributeTok{labels=}\StringTok{\textquotesingle{}epsilon\textquotesingle{}}\NormalTok{, }\AttributeTok{values =} \DecValTok{1}\NormalTok{, }\AttributeTok{lbound =} \DecValTok{0}\NormalTok{),}
  
  \CommentTok{\#Latent variable covariances and variances (note that only the variances are estimated. )}
  \FunctionTok{mxPath}\NormalTok{(}\AttributeTok{from =}\NormalTok{ latent\_vars,}
         \AttributeTok{connect=}\StringTok{\textquotesingle{}unique.pairs\textquotesingle{}}\NormalTok{, }\AttributeTok{arrows=}\DecValTok{2}\NormalTok{,}
         \AttributeTok{free =} \FunctionTok{c}\NormalTok{(}\ConstantTok{TRUE}\NormalTok{,}\ConstantTok{FALSE}\NormalTok{, }\ConstantTok{FALSE}\NormalTok{, }\ConstantTok{FALSE}\NormalTok{, }
                  \ConstantTok{TRUE}\NormalTok{, }\ConstantTok{FALSE}\NormalTok{, }\ConstantTok{FALSE}\NormalTok{, }
                  \ConstantTok{TRUE}\NormalTok{, }\ConstantTok{FALSE}\NormalTok{, }
                  \ConstantTok{TRUE}\NormalTok{), }
         \AttributeTok{values=}\FunctionTok{c}\NormalTok{(}\DecValTok{1}\NormalTok{, }\ConstantTok{NA}\NormalTok{, }\ConstantTok{NA}\NormalTok{, }\ConstantTok{NA}\NormalTok{, }
                  \DecValTok{1}\NormalTok{, }\ConstantTok{NA}\NormalTok{, }\ConstantTok{NA}\NormalTok{, }
                  \DecValTok{1}\NormalTok{, }\ConstantTok{NA}\NormalTok{,}
                  \DecValTok{1}\NormalTok{),}
         \AttributeTok{labels=}\FunctionTok{c}\NormalTok{(}\StringTok{\textquotesingle{}theta\_rand\textquotesingle{}}\NormalTok{, }\StringTok{\textquotesingle{}NA(cov\_theta\_alpha)\textquotesingle{}}\NormalTok{, }\StringTok{\textquotesingle{}NA(cov\_theta\_beta)\textquotesingle{}}\NormalTok{, }
                  \StringTok{\textquotesingle{}NA(cov\_theta\_gamma)\textquotesingle{}}\NormalTok{,}
                  \StringTok{\textquotesingle{}alpha\_rand\textquotesingle{}}\NormalTok{,}\StringTok{\textquotesingle{}NA(cov\_alpha\_beta)\textquotesingle{}}\NormalTok{, }\StringTok{\textquotesingle{}NA(cov\_alpha\_gamma)\textquotesingle{}}\NormalTok{, }
                  \StringTok{\textquotesingle{}beta\_rand\textquotesingle{}}\NormalTok{, }\StringTok{\textquotesingle{}NA(cov\_beta\_gamma)\textquotesingle{}}\NormalTok{, }
                  \StringTok{\textquotesingle{}gamma\_rand\textquotesingle{}}\NormalTok{), }
         \AttributeTok{lbound =} \FunctionTok{c}\NormalTok{(}\FloatTok{1e{-}3}\NormalTok{, }\ConstantTok{NA}\NormalTok{, }\ConstantTok{NA}\NormalTok{, }\ConstantTok{NA}\NormalTok{, }
                    \FloatTok{1e{-}3}\NormalTok{, }\ConstantTok{NA}\NormalTok{, }\ConstantTok{NA}\NormalTok{, }
                    \DecValTok{1}\NormalTok{, }\ConstantTok{NA}\NormalTok{,}
                    \DecValTok{1}\NormalTok{), }
         \AttributeTok{ubound =} \FunctionTok{c}\NormalTok{(}\DecValTok{2}\NormalTok{, }\ConstantTok{NA}\NormalTok{, }\ConstantTok{NA}\NormalTok{, }\ConstantTok{NA}\NormalTok{, }
                    \DecValTok{2}\NormalTok{, }\ConstantTok{NA}\NormalTok{, }\ConstantTok{NA}\NormalTok{, }
                    \DecValTok{90}\SpecialCharTok{\^{}}\DecValTok{2}\NormalTok{, }\ConstantTok{NA}\NormalTok{, }
                    \DecValTok{45}\SpecialCharTok{\^{}}\DecValTok{2}\NormalTok{)),}
  
  \CommentTok{\# Latent variable means (linear parameters). Note that the parameters of beta and gamma do not have estimated means because they are nonlinear parameters (i.e., the logistic function\textquotesingle{}s first{-}order partial derivative with respect to each of those two parameters contains those two parameters)}
  \FunctionTok{mxPath}\NormalTok{(}\AttributeTok{from =} \StringTok{\textquotesingle{}one\textquotesingle{}}\NormalTok{, }\AttributeTok{to =} \FunctionTok{c}\NormalTok{(}\StringTok{\textquotesingle{}theta\textquotesingle{}}\NormalTok{, }\StringTok{\textquotesingle{}alpha\textquotesingle{}}\NormalTok{), }\AttributeTok{free =} \FunctionTok{c}\NormalTok{(}\ConstantTok{TRUE}\NormalTok{, }\ConstantTok{TRUE}\NormalTok{), }\AttributeTok{arrows =} \DecValTok{1}\NormalTok{,}
         \AttributeTok{labels =} \FunctionTok{c}\NormalTok{(}\StringTok{\textquotesingle{}theta\_fixed\textquotesingle{}}\NormalTok{, }\StringTok{\textquotesingle{}alpha\_fixed\textquotesingle{}}\NormalTok{), }\AttributeTok{lbound =} \DecValTok{0}\NormalTok{, }\AttributeTok{ubound =} \DecValTok{7}\NormalTok{, }
         \AttributeTok{values =} \FunctionTok{c}\NormalTok{(}\DecValTok{1}\NormalTok{, }\DecValTok{1}\NormalTok{)),}
  
  \CommentTok{\#Functional constraints (needed to estimate mean values of fixed{-}effect parameters)}
  \FunctionTok{mxMatrix}\NormalTok{(}\AttributeTok{type =} \StringTok{\textquotesingle{}Full\textquotesingle{}}\NormalTok{, }\AttributeTok{nrow =} \FunctionTok{length}\NormalTok{(manifest\_vars), }\AttributeTok{ncol =} \DecValTok{1}\NormalTok{, }\AttributeTok{free =} \ConstantTok{TRUE}\NormalTok{, }
           \AttributeTok{labels =} \StringTok{\textquotesingle{}theta\_fixed\textquotesingle{}}\NormalTok{, }\AttributeTok{name =} \StringTok{\textquotesingle{}t\textquotesingle{}}\NormalTok{, }\AttributeTok{values =} \DecValTok{1}\NormalTok{, }\AttributeTok{lbound =} \DecValTok{0}\NormalTok{,  }\AttributeTok{ubound =} \DecValTok{7}\NormalTok{), }
  \FunctionTok{mxMatrix}\NormalTok{(}\AttributeTok{type =} \StringTok{\textquotesingle{}Full\textquotesingle{}}\NormalTok{, }\AttributeTok{nrow =} \FunctionTok{length}\NormalTok{(manifest\_vars), }\AttributeTok{ncol =} \DecValTok{1}\NormalTok{, }\AttributeTok{free =} \ConstantTok{TRUE}\NormalTok{, }
           \AttributeTok{labels =} \StringTok{\textquotesingle{}alpha\_fixed\textquotesingle{}}\NormalTok{, }\AttributeTok{name =} \StringTok{\textquotesingle{}a\textquotesingle{}}\NormalTok{, }\AttributeTok{values =} \DecValTok{1}\NormalTok{, }\AttributeTok{lbound =} \DecValTok{0}\NormalTok{,  }\AttributeTok{ubound =} \DecValTok{7}\NormalTok{), }
  \FunctionTok{mxMatrix}\NormalTok{(}\AttributeTok{type =} \StringTok{\textquotesingle{}Full\textquotesingle{}}\NormalTok{, }\AttributeTok{nrow =} \FunctionTok{length}\NormalTok{(manifest\_vars), }\AttributeTok{ncol =} \DecValTok{1}\NormalTok{, }\AttributeTok{free =} \ConstantTok{TRUE}\NormalTok{, }
           \AttributeTok{labels =} \StringTok{\textquotesingle{}beta\_fixed\textquotesingle{}}\NormalTok{, }\AttributeTok{name =} \StringTok{\textquotesingle{}b\textquotesingle{}}\NormalTok{, }\AttributeTok{values =} \DecValTok{1}\NormalTok{, }\AttributeTok{lbound =} \DecValTok{1}\NormalTok{, }\AttributeTok{ubound =} \DecValTok{360}\NormalTok{),}
  \FunctionTok{mxMatrix}\NormalTok{(}\AttributeTok{type =} \StringTok{\textquotesingle{}Full\textquotesingle{}}\NormalTok{, }\AttributeTok{nrow =} \FunctionTok{length}\NormalTok{(manifest\_vars), }\AttributeTok{ncol =} \DecValTok{1}\NormalTok{, }\AttributeTok{free =} \ConstantTok{TRUE}\NormalTok{, }
           \AttributeTok{labels =} \StringTok{\textquotesingle{}gamma\_fixed\textquotesingle{}}\NormalTok{, }\AttributeTok{name =} \StringTok{\textquotesingle{}g\textquotesingle{}}\NormalTok{, }\AttributeTok{values =} \DecValTok{1}\NormalTok{, }\AttributeTok{lbound =} \DecValTok{1}\NormalTok{, }\AttributeTok{ubound =} \DecValTok{360}\NormalTok{), }

  \CommentTok{\#Definition variables set for loadings (accounts for time{-}unstructured data) }
  \FunctionTok{mxMatrix}\NormalTok{(}\AttributeTok{type =} \StringTok{\textquotesingle{}Full\textquotesingle{}}\NormalTok{, }\AttributeTok{nrow =} \FunctionTok{length}\NormalTok{(obs\_score\_days), }\AttributeTok{ncol =} \DecValTok{1}\NormalTok{, }\AttributeTok{free =} \ConstantTok{FALSE}\NormalTok{, }
  \AttributeTok{labels =}\NormalTok{ obs\_score\_days, }\AttributeTok{name =} \StringTok{\textquotesingle{}time\textquotesingle{}}\NormalTok{),}

  \CommentTok{\#Algebra specifying first{-}order partial derivatives; }
  \FunctionTok{mxAlgebra}\NormalTok{(}\AttributeTok{expression =} \DecValTok{1} \SpecialCharTok{{-}} \DecValTok{1}\SpecialCharTok{/}\NormalTok{(}\DecValTok{1} \SpecialCharTok{+} \FunctionTok{exp}\NormalTok{((b }\SpecialCharTok{{-}}\NormalTok{ time)}\SpecialCharTok{/}\NormalTok{g)), }\AttributeTok{name=}\StringTok{"Tl"}\NormalTok{),}
  \FunctionTok{mxAlgebra}\NormalTok{(}\AttributeTok{expression =} \DecValTok{1}\SpecialCharTok{/}\NormalTok{(}\DecValTok{1} \SpecialCharTok{+} \FunctionTok{exp}\NormalTok{((b }\SpecialCharTok{{-}}\NormalTok{ time)}\SpecialCharTok{/}\NormalTok{g)), }\AttributeTok{name =} \StringTok{\textquotesingle{}Al\textquotesingle{}}\NormalTok{), }
  \FunctionTok{mxAlgebra}\NormalTok{(}\AttributeTok{expression =} \SpecialCharTok{{-}}\NormalTok{((a }\SpecialCharTok{{-}}\NormalTok{ t) }\SpecialCharTok{*}\NormalTok{ (}\FunctionTok{exp}\NormalTok{((b }\SpecialCharTok{{-}}\NormalTok{ time)}\SpecialCharTok{/}\NormalTok{g) }\SpecialCharTok{*}\NormalTok{ (}\DecValTok{1}\SpecialCharTok{/}\NormalTok{g))}\SpecialCharTok{/}\NormalTok{(}\DecValTok{1} \SpecialCharTok{+} \FunctionTok{exp}\NormalTok{((b }\SpecialCharTok{{-}}\NormalTok{ time)}\SpecialCharTok{/}\NormalTok{g))}\SpecialCharTok{\^{}}\DecValTok{2}\NormalTok{), }\AttributeTok{name =} \StringTok{\textquotesingle{}Bl\textquotesingle{}}\NormalTok{),}
  \FunctionTok{mxAlgebra}\NormalTok{(}\AttributeTok{expression =}\NormalTok{  (a }\SpecialCharTok{{-}}\NormalTok{ t) }\SpecialCharTok{*}\NormalTok{ (}\FunctionTok{exp}\NormalTok{((b }\SpecialCharTok{{-}}\NormalTok{ time)}\SpecialCharTok{/}\NormalTok{g) }\SpecialCharTok{*}\NormalTok{ ((b }\SpecialCharTok{{-}}\NormalTok{ time)}\SpecialCharTok{/}\NormalTok{g}\SpecialCharTok{\^{}}\DecValTok{2}\NormalTok{))}\SpecialCharTok{/}\NormalTok{(}\DecValTok{1} \SpecialCharTok{+} \FunctionTok{exp}\NormalTok{((b }\SpecialCharTok{{-}}\NormalTok{time)}\SpecialCharTok{/}\NormalTok{g))}\SpecialCharTok{\^{}}\DecValTok{2}\NormalTok{, }\AttributeTok{name =} \StringTok{\textquotesingle{}Gl\textquotesingle{}}\NormalTok{),}
  
  \CommentTok{\#Factor loadings; all fixed and, importantly, constrained to change according to their partial derivatives (i.e., nonlinear functions) }
  \FunctionTok{mxPath}\NormalTok{(}\AttributeTok{from =} \StringTok{\textquotesingle{}theta\textquotesingle{}}\NormalTok{, }\AttributeTok{to =}\NormalTok{ manifest\_vars, }\AttributeTok{arrows=}\DecValTok{1}\NormalTok{, }\AttributeTok{free=}\ConstantTok{FALSE}\NormalTok{,  }
         \AttributeTok{labels =} \FunctionTok{sprintf}\NormalTok{(}\AttributeTok{fmt =} \StringTok{\textquotesingle{}Tl[\%d,1]\textquotesingle{}}\NormalTok{, }\DecValTok{1}\SpecialCharTok{:}\FunctionTok{length}\NormalTok{(manifest\_vars))),}
  \FunctionTok{mxPath}\NormalTok{(}\AttributeTok{from =} \StringTok{\textquotesingle{}alpha\textquotesingle{}}\NormalTok{, }\AttributeTok{to =}\NormalTok{ manifest\_vars, }\AttributeTok{arrows=}\DecValTok{1}\NormalTok{, }\AttributeTok{free=}\ConstantTok{FALSE}\NormalTok{,  }
         \AttributeTok{labels =} \FunctionTok{sprintf}\NormalTok{(}\AttributeTok{fmt =} \StringTok{\textquotesingle{}Al[\%d,1]\textquotesingle{}}\NormalTok{, }\DecValTok{1}\SpecialCharTok{:}\FunctionTok{length}\NormalTok{(manifest\_vars))), }
  \FunctionTok{mxPath}\NormalTok{(}\AttributeTok{from=}\StringTok{\textquotesingle{}beta\textquotesingle{}}\NormalTok{, }\AttributeTok{to =}\NormalTok{ manifest\_vars, }\AttributeTok{arrows=}\DecValTok{1}\NormalTok{,  }\AttributeTok{free=}\ConstantTok{FALSE}\NormalTok{,}
         \AttributeTok{labels =}  \FunctionTok{sprintf}\NormalTok{(}\AttributeTok{fmt =} \StringTok{\textquotesingle{}Bl[\%d,1]\textquotesingle{}}\NormalTok{, }\DecValTok{1}\SpecialCharTok{:}\FunctionTok{length}\NormalTok{(manifest\_vars))), }
  \FunctionTok{mxPath}\NormalTok{(}\AttributeTok{from=}\StringTok{\textquotesingle{}gamma\textquotesingle{}}\NormalTok{, }\AttributeTok{to =}\NormalTok{ manifest\_vars, }\AttributeTok{arrows=}\DecValTok{1}\NormalTok{,  }\AttributeTok{free=}\ConstantTok{FALSE}\NormalTok{,}
         \AttributeTok{labels =}  \FunctionTok{sprintf}\NormalTok{(}\AttributeTok{fmt =} \StringTok{\textquotesingle{}Gl[\%d,1]\textquotesingle{}}\NormalTok{, }\DecValTok{1}\SpecialCharTok{:}\FunctionTok{length}\NormalTok{(manifest\_vars))), }
  
  \CommentTok{\#Fit function used to estimate free parameter values. }
  \FunctionTok{mxFitFunctionML}\NormalTok{(}\AttributeTok{vector =} \ConstantTok{FALSE}\NormalTok{)}
\NormalTok{)}

\CommentTok{\#Fit model using mxTryHard(). Increases probability of convergence by attempting model convergence by randomly shifting starting values. }
\NormalTok{model\_results }\OtherTok{\textless{}{-}} \FunctionTok{mxTryHard}\NormalTok{(def\_growth\_curve\_model)}
\end{Highlighting}
\end{Shaded}



\end{document}
